\phantomsection\numberedsection{Apartado B}

Hemos considerado diferentes opciones para poder implementar la opción de que un socio que ya tiene un rol
pueda realizar las acciones de otro. Entre estas opciones tenemos:

\numberedsubsection{Herencia simple en Java}
En la Programación Orientada a Objetos con Java, no es posible establecer 
una herencia múltiple ya que una clase en Java solo puede heredar de una 
única clase base. Esto significa que una vez \emph{Socio} se cree como 
instancia de una subclase específica (por ejemplo, \emph{Voluntario}), 
no puede ser también instancia de otra (como \emph{Adoptante} o \emph{Donante}).\par
\vspace{0.15cm}
Dado que las clases \emph{Voluntario}, \emph{Adoptante} y \emph{Donante} 
heredan todas de la misma clase base \emph{Socio}, no es posible que un 
mismo objeto \emph{Socio} asuma múltiples roles simultáneamente.\par

\numberedsubsection{Impacto en la implementación}
El modelo dado se basa en subclases que encapsulan el comportamiento 
específico de cada rol. Si intentamos que un socio tuviera múltiples roles, 
tendríamos que duplicar la información del mismo socio en varias instancias 
lo que rompería la unicidad del objeto. Por ejemplo:
\begin{itemize}
    \item Dos objetos diferentes representarían al mismo socio, pero 
    con roles distintos, lo que podría llevar a datos contradictorios.
    \item La relación entre \emph{Socio} y \emph{Refugio} perdería sentido 
    al no poder asociarse al objeto ambiguo.
\end{itemize}

\newpage