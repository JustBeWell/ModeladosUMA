\phantomsection\numberedsection{Apartado C}

Se nos pide considerar el caso en el que un mismo socio puede desempeñar 
múltiples roles. Tras discutir diferentes opciones, hemos llegado a las siguientes ideas:

\begin{itemize}
    \item \textbf{Composición en lugar de herencia:} 
    Facilita la reutilización al delegar la lógica común a clases auxiliares,
    pero puede aumentar el código necesario para configurar relaciones entre las clases.
    \item \textbf{Interfaces:} 
    Permite a las clases hijas implementar acciones específicas de forma flexible,
    pero puede generar complejidad si hay demasiadas interfaces, ya que cada acción
    requiere una definición separada.
    \item \textbf{Rediseño del modelo con una clase Socio más rica:}
    Centraliza el comportamiento en la clase base, simplificando el diseño y
    evitando la duplicación de lógica, pero puede hacer que la clase base sea
    demasiado compleja.
    \item \textbf{Uso de patrones de diseño como el Singleton:}
    Permite compartir una única instancia de una clase para gestionar lógica o
    estado común entre las clases hijas, asegurando acceso centralizado y consistente.
    Aunque reduce la duplicación de lógica, puede introducir un acoplamiento fuerte
    entre las clases.
\end{itemize}

A continuación, se discute el enfoque que hemos visto más adecuado para resolver 
el caso propuesto justificando las decisiones de diseño que conlleva.

\newpage