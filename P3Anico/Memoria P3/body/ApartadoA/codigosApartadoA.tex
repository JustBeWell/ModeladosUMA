\subsubsection{Clase Socio}\label{codigo:socio}
La clase \texttt{Socio} es abstracta y representa la base para las distintas subclases: 
\texttt{Adoptante}, \texttt{Voluntario}, y \texttt{Donante}. Esta clase asegura que 
cada socio tenga un \texttt{ID} único, una fecha de registro válida y un refugio asociado.

\begin{lstlisting}[style = javaNormal, language=Java] 
    package sistema;

    import java.util.Collections;
    import java.util.Date;
    
    public abstract class Socio {
        private int ID;
        private Date registro;
        private final Refugio refugioAsociado;
    
        public Socio(int ID, Date fechaRegistro, Refugio refugioAsociado) {
            assert ID > 0 : "El ID del socio debe ser valido.";
    
            assert fechaRegistro != null : "La fecha de registro no puede ser nula.";
    
            assert refugioAsociado != null : "El refugio asociado no puede ser nulo.";
    
            this.ID = ID;
            this.registro = fechaRegistro;
            this.refugioAsociado = refugioAsociado;
            refugioAsociado.addSocio(this);
    
            assert refugioAsociado.getSocios().hasMoreElements() && Collections.list(refugioAsociado.getSocios()).contains(this);
        }
    
        public int getID() {
            return this.ID;
        }
    
        private void setID(int ID) {
            assert ID > 0;
    
            assert Collections.list(this.getRefugio().getSocios()).stream().noneMatch((s) -> {
                return s.getID() == ID;
            }) : "El ID ya existe en el refugio";
    
            this.ID = ID;
        }
    
        public Date getRegistro() {
            return this.registro;
        }
    
        public void setRegistro(Date fechaRegistro) {
            assert fechaRegistro != null;
    
            this.registro = fechaRegistro;
        }
    
        public Refugio getRefugio() {
            return this.refugioAsociado;
        }
    
        public boolean equals(Object obj) {
            if (this == obj) {
                return true;
            } else if (obj instanceof Socio) {
                Socio socio = (Socio)obj;
                return this.ID == socio.ID;
            } else {
                return false;
            }
        }
    
        public int hashCode() {
            return Integer.hashCode(this.ID);
        }
    }
\end{lstlisting}



\subsubsection{Clase Donante}\label{codigo:donante}
La clase \texttt{Donante} extiende de \texttt{Socio} y gestiona las donaciones realizadas 
por un socio. Las donaciones se almacenan en un \texttt{HashSet} para asegurar que no hayan elementos repetidos y por eficiencia. (cómo se explicó en \ref{page:Consideraciones}) 
Por otro lado, hemos decidido implementar al crear un nuevo objeto Donante en el sistema,
el constructor llamará directamente al método donar ya que es una condición necesaria para ser donante. Las donaciones que sean modificadas son tratadas desde la clase Donación y
el \texttt{HashSet donaciones} se actualizará.


\begin{lstlisting}[style = javaNormal, language=Java] 
    package sistema;

    import java.time.LocalDate;
    import java.time.ZoneId;
    import java.util.*;
    
    public class Donante extends Socio{
        private Set<Donacion> donaciones;
        public Donante(int ID, Date date,Refugio r, float cantidad) {
            super(ID,date,r);
            assert cantidad > 0 : "La cantidad inicial donada debe ser mayor a cero.";
            donaciones = new HashSet<>();
            donar(cantidad);
        }

        public void donar(float cantidad){
            assert cantidad > 0 : "La cantidad donada debe ser mayor a cero.";
            assert Collections.list(this.getRefugio().getSocios()).contains(this): "El socio debe ser donante antes de poder donar";
            Donacion d = new Donacion(cantidad, Date.from(fechaDonacion.atStartOfDay(ZoneId.systemDefault()).toInstant()));
            addDonacion(d);
            Refugio r = super.getRefugio();
            r.setLiquidez(r.getLiquidez() + cantidad);
            assert donaciones.contains(d);
        }
    
        protected void addDonacion(Donacion donacion){
            assert donacion != null: "La donacion no puede ser nula";
            donaciones.add(donacion);
        }
    
        protected void removeDonacion(Donacion donacion){
            assert donacion != null : "La donacion no puede ser nula.";
            if (donaciones.contains(donacion) && donaciones.size() > 1) {
                donaciones.remove(donacion);
            } else if (donaciones.contains(donacion) && donaciones.size() == 1) {
                System.out.println("Todo donante debe tener al menos una donacion, estas intentando eliminar la unica donacion asociada a este socio donante");
            } else {
                System.out.println("Este socio no ha realizado la donacion que intentas eliminar");
            }
        }
        public Enumeration<Donacion> getDonaciones(){
            return Collections.enumeration(this.donaciones);
        }
    
        @Override
        public String toString() {
            return "Donante " + super.getID();
        }
    }
    
\end{lstlisting}

\subsubsection{Clase Adoptante}\label{codigo:adoptante}
La clase \texttt{Adoptante} extiende de \texttt{Socio} y simula las adopciones realizadas 
por un adoptante. Las adopciones se almacenan en un \texttt{HashSet} para evitar adopciones duplicadas.
\begin{lstlisting}[style = javaNormal, language=Java] 
    package sistema;

    import java.util.*;
    
    public class Adoptante extends Socio {
        private Set<Adopcion> adopciones;
    
        public Adoptante(int ID, Date date, Refugio r) {
            super(ID, date,r);
            adopciones = new HashSet<>();
        }
        public void adoptar(Animal a, Voluntario v) {
            assert a.getEstadoAnimal() == EstadoAnimal.DISPONIBLE: "El animal ya ha sido adoptado";
            Refugio refugioDelVoluntario = v.getRefugio();
            a.setEstadoAnimal(EstadoAnimal.ADOPTADO);
            refugioDelVoluntario.removeAnimalesRefugiados(a);
            v.tramitarAdopcion(a, this);
        }
    
        protected void addAdopcion(Adopcion a){
            this.adopciones.add(a);
        }
    
        protected void removeAdopcion(Adopcion a){
            if (adopciones.contains(a)) adopciones.remove(a);
            else System.out.println("Este animal ya no esta asociado al adoptante");
        }
        public Enumeration<Adopcion> getAdopciones(){
            return Collections.enumeration(adopciones);
        }
        @Override
        public boolean equals(Object obj) {
            if (this == obj) return true;
            if(obj instanceof Adoptante ){
                Adoptante adoptante = (Adoptante) obj;
                return adoptante.getID() == this.getID();
            }
            return false;
        }
        @Override
        public int hashCode() {
            return Integer.hashCode(this.getID());
        }
    
        @Override
        public String toString() {
            return "Adoptante " + super.getID();
        } 
    }    
\end{lstlisting}

\subsubsection{Clase Voluntario}\label{codigo:voluntario}
La clase \texttt{Voluntario} extiende de \texttt{Socio} y gestiona los trámites de 
adopción realizados por un voluntario. Los voluntarios se almacenan en un \texttt{HashSet},
evitando así voluntarios duplicados. 
\begin{lstlisting}[style = javaNormal, language=Java] 
    package sistema;

    import java.time.LocalDate;
    import java.time.ZoneId;
    import java.util.*;
    
    public class Voluntario extends Socio{
        Set<Adopcion> tramites;
        
        public Voluntario(int ID, Date date,Refugio r) {
            super(ID, date,r);
            tramites = new HashSet<>();
        }
        public void tramitarAdopcion(Animal a, Adoptante ad){
            assert ad != null : "El adoptante no puede ser nulo.";
            LocalDate fechaAdopcion = LocalDate.now();
            Adopcion adopcion = new Adopcion(a, ad, this, Date.from(fechaAdopcion.atStartOfDay(ZoneId.systemDefault()).toInstant()));
            addTramite(adopcion);
            ad.addAdopcion(adopcion);
        }
        public void registrar(Animal a){
            Refugio r = super.getRefugio();
            assert r != null : "El refugio asociado no puede ser nulo.";
            assert a != null : "El animal no puede ser nulo.";
            a.setEstadoAnimal(EstadoAnimal.DISPONIBLE);
            r.addAnimalesRefugiados(a);
    
        }
        public Enumeration<Adopcion> getTramites(){
            return Collections.enumeration(tramites);
        }
        protected void addTramite(Adopcion ad){
            assert ad != null : "El tramite de adopcion no puede ser nulo.";
            tramites.add(ad);
    
        }
        protected void removeTramite(Adopcion ad){
            assert  ad != null: "El tramite de adopcion no puede ser nulo.";
            tramites.remove(ad);
        }
    
    
        @Override
        public String toString() {
            return "Voluntario " + super.getID();
        }
    }
    
\end{lstlisting}

\subsubsection{Clase Refugio}\label{codigo:refugio}
La clase \texttt{Refugio} gestiona el conjunto de \texttt{Socios} y \texttt{Animales}.\par
\textbf{liquidez} está declarado como un \texttt{float} por las razones que se exponen en \ref{page:Consideraciones}.


\begin{lstlisting}[style = javaNormal, language=Java] 
package sistema;
import java.util.*;

public class Refugio {
    private float liquidez;
    private Set<Animal> animalesRegistrados;
    private Set<Animal> animalesRefugiados;
    private Set<Socio> socios;

    public Refugio(float liquidez) {
        assert liquidez >= 0 : "La liquidez debe ser no negativa.";
        this.liquidez = liquidez;
        animalesRefugiados = new HashSet<>();
        animalesRegistrados = new HashSet<>();
        socios = new HashSet<>();
    }

    public float getLiquidez() 
        return liquidez;
    }
    public void setLiquidez(float liquidez) {
        assert liquidez >= 0 : "La liquidez debe ser no negativa";
        this.liquidez = liquidez;
    }
    protected void addSocio(Socio s) {
        assert s != null : "El socio no puede ser nulo.";
        if(socios.contains(s)) {
            System.out.println("El socio ya esta registrado.");
            return;
        }
        socios.add(s);
    }
    protected void removeSocio(Socio s) {
        assert s != null : "El socio no puede ser nulo.";
        if (socios.contains(s)) {
            socios.remove(s);
        } else {
            System.out.println("Este socio no esta registrado en el refugio.");
        }
    }

    public Enumeration<Animal> getAnimalesRegistrados() {
        return Collections.enumeration(animalesRegistrados);
    }
    public Enumeration<Animal> getAnimalesRefugiados() {
        return Collections.enumeration(animalesRefugiados);
    }
    public Enumeration<Socio> getSocios() {
        return Collections.enumeration(socios);
    }

    public List<Adoptante> getAdoptantes() {
        List<Adoptante> adoptantes = new ArrayList<>();
        for (Socio s : socios) {
            if (s instanceof Adoptante) {
                adoptantes.add((Adoptante) s);
            }
        }
        return adoptantes;
    }
    public List<Voluntario> getVoluntarios() {
        List<Voluntario> voluntarios = new ArrayList<>();
        for (Socio s : socios) {
            if (s instanceof Voluntario) {
                voluntarios.add((Voluntario) s);
            }
        }
        return voluntarios;
    }
    public List<Donante> getDonantes() {
        List<Donante> donantes = new ArrayList<>();
        for (Socio s : socios) {
            if (s instanceof Donante) {
                donantes.add((Donante) s);
            }
        }
        return donantes;
    }
    public void registrar(Animal a){
        this.addAnimalesRegistrados(a);
    }
    protected void addAnimalesRefugiados(Animal a){
        assert a != null : "El animal no puede ser nulo.";
        if(!animalesRefugiados.contains(a)){
            animalesRefugiados.add(a);
            this.addAnimalesRegistrados(a);
        } else System.out.println("Este animal ya esta en el refugio.");
    }
    private void addAnimalesRegistrados(Animal a){
        assert a != null : "El animal no puede ser nulo.";
        if (!animalesRegistrados.contains(a)) {
            animalesRegistrados.add(a);
        } else {
            System.out.println("El animal ya esta registrado.");
        }
    }

    protected void removeAnimalesRefugiados(Animal a){
        assert a != null : "El animal no puede ser nulo.";
        if (animalesRefugiados.contains(a)) {
            animalesRefugiados.remove(a);
        } else {
            System.out.println("El animal no se encuentra en este Refugio.");
        }
    }
    protected void removeAnimalesRegistrados(Animal a){
        assert a != null : "El animal no puede ser nulo.";
        if (animalesRegistrados.contains(a) && animalesRegistrados.size() > 1) {
            animalesRegistrados.remove(a);
        } else if (animalesRegistrados.contains(a) && animalesRegistrados.size() == 1) {
            System.out.println("Todo refugio debe tener al menos un animal registrado, estas intentando eliminar el unico animal existente.");
        } else {
            System.out.println("El animal no se encuentra en este Refugio.");
        }
    }

    public void mostrarAnimalesRefugiados(){
        System.out.println(animalesRefugiados.toString());
    }
    public void mostrarAnimalesRegistrados(){
        System.out.println(animalesRegistrados.toString());
    }
    public void mostrarSocios() {
        for (Socio s : socios) {
            System.out.println(s);
        }
    }
    public void mostrarSociosPorTipo() {
        System.out.println("Adoptantes: " + getAdoptantes());
        System.out.println("Voluntarios: " + getVoluntarios());
        System.out.println("Donantes: " + getDonantes());
    }

    @Override
    public String toString() {
        StringBuilder sb = new StringBuilder();
        sb.append("Animales Registrados: ").append(animalesRegistrados).append("\n");
        sb.append("Animales Refugiados: ").append(animalesRefugiados).append("\n");
        sb.append("Socios: ").append(socios).append("\n");
        sb.append("Liquidez: ").append(liquidez);
        return sb.toString();
    }
}
\end{lstlisting}



\subsubsection{Clase Donacion}\label{codigo:donacion}
La clase \texttt{Donacion} representa una donación realizada por un \texttt{Donante}. 
Incluye la cantidad que como anteriormente mencionamos en \ref{page:Consideraciones} por temas de eficiencia es un \texttt{float},
la fecha de la donación y el donante asociado. Las validaciones 
aseguran que los valores sean válidos en el momento de la creación de la instancia.

\begin{lstlisting}[style = javaNormal, language=Java] 
    package sistema;

    import java.util.Date;
    import java.util.Objects;
    
    public class Donacion {
        private float cantidad;
        private Date date;
    
        public Donacion(float cantidad, Date date) {
            assert cantidad > 0.0F : "La cantidad debe ser positiva.";
    
            assert date != null && !date.after(new Date()) : "La fecha no puede ser nula ni estar en el futuro.";
    
            this.cantidad = cantidad;
            this.date = date;
        }
    
        public float getCantidad() {
            assert this.cantidad > 0.0F : "La cantidad no puede ser nula.";
    
            return this.cantidad;
        }
    
        public void setCantidad(float cantidad) {
            this.cantidad = cantidad;
        }
    
        public Date getDate() {
            assert this.date != null : "La fecha no puede ser nula.";
    
            return this.date;
        }
    
        public void setDate(Date date) {
            this.date = date;
        }
    
        public String toString() {
            return String.format("Donacion: %.2f, %tY-%tB-%td", this.cantidad, this.date, this.date, this.date);
        }
    
        public boolean equals(Object o) {
            if (this == o) {
                return true;
            } else if (o != null && this.getClass() == o.getClass()) {
                Donacion donacion = (Donacion)o;
                return Float.compare(this.cantidad, donacion.cantidad) == 0 && Objects.equals(this.date, donacion.date);
            } else {
                return false;
            }
        }
    
        public int hashCode() {
            return Objects.hash(new Object[]{this.cantidad, this.date});
        }
    }
\end{lstlisting}



\subsubsection{Clase Adopcion}\label{codigo:adopcion}
La clase \texttt{Adopcion} modela una adopción de un \texttt{Animal} realizada por un 
\texttt{Adoptante}, gestionada por un \texttt{Voluntario}. Implementa la bidireccionalidad 
entre estas entidades para mantener consistencia en las asociaciones.

\begin{lstlisting}[style = javaNormal, language=Java] 
    package sistema;

    import java.util.Date;
    
    public class Adopcion {
        private Date fecha;
        private final Animal animal;
        private final Adoptante adoptante;
        private final Voluntario voluntario;
    
        public Adopcion(Animal a, Adoptante ad, Voluntario v, Date fecha) {
            assert a != null : "El animal no puede ser nulo.";
    
            assert ad != null : "El adoptante no puede ser nulo.";
    
            assert v != null : "El voluntario no puede ser nulo.";
    
            assert fecha != null && !fecha.after(new Date()) : "La fecha no puede ser nula ni estar en el futuro.";
    
            this.animal = a;
            this.adoptante = ad;
            this.voluntario = v;
            this.fecha = fecha;
        }
    
        public Date getFecha() {
            return this.fecha;
        }
    
        public void setFecha(Date fecha) {
            assert fecha != null && !fecha.after(new Date()) : "La fecha no puede ser nula ni estar en el futuro";
    
            this.fecha = fecha;
        }
    
        public Animal getAnimal() {
            return this.animal;
        }
    
        public Voluntario getVoluntario() {
            return this.voluntario;
        }
    
        public Adoptante getAdoptante() {
            return this.adoptante;
        }
    
        public boolean equals(Object obj) {
            if (this == obj) {
                return true;
            } else if (!(obj instanceof Adopcion)) {
                return false;
            } else {
                Adopcion adopcion = (Adopcion)obj;
                boolean ok = this.adoptante.equals(adopcion.adoptante) && this.animal.equals(adopcion.animal);
                return ok;
            }
        }
    
        public int hashCode() {
            return this.adoptante.hashCode() + this.animal.hashCode();
        }
    
        public String toString() {
            return String.format("Adopcion: %tY-%tB-%td, %s, %s", this.fecha, this.fecha, this.fecha, this.animal, this.adoptante);
        }
    }
    
\end{lstlisting}



\subsubsection{Clase Animal}\label{codigo:animal}
La clase \texttt{Animal} modela a un animal registrado en el sistema. Cada animal tiene un 
ID único, una fecha de nacimiento, un estado actual y está asociado a un \texttt{Refugio}.\par

\begin{lstlisting}[style = javaNormal, language=Java] 
    package sistema;
    import java.util.Date;
    
    public class Animal {
        private int ID;
        private Date nacimiento;
        private EstadoAnimal estadoAnimal; 
        private Adopcion adopcion;

        public Animal(int ID, Date nacimiento, EstadoAnimal estadoAnimal) {
            assert ID > 0 : "El ID del animal debe ser valido.";
            assert nacimiento != null : "La fecha de nacimiento no puede ser nula.";
            assert estadoAnimal != null : "El estado del animal debe estar definido.";

            this.ID = ID;
            this.nacimiento = nacimiento;
            this.estadoAnimal = estadoAnimal;
        }
    
        public EstadoAnimal getEstadoAnimal() {
            return estadoAnimal;
        }
        public void setEstadoAnimal(EstadoAnimal estadoAnimal) {
            assert estadoAnimal != null : "El estado del animal debe estar definido.";
            this.estadoAnimal = estadoAnimal;
        }
        public Date getNacimiento() {
            return nacimiento;
        }
        public void setNacimiento(Date nacimiento) {
            assert nacimiento != null : "La fecha de nacimiento no puede ser nula";
            this.nacimiento = nacimiento;
        }
        public Adopcion getAdopcion() {
            return this.adopcion;
        }
        public void setAdopcion(Adopcion adopcion){
            assert  adopcion != null;
            this.adopcion = adopcion;
        }
        public int getID() {
            return ID;
        }

        @Override
        public boolean equals(Object obj) {
            if( this == obj ) return true;
            if(obj instanceof Animal ){
                Animal animal = (Animal) obj;
                return this.ID == animal.ID;
            }
            return false;
        }
        @Override
        public int hashCode() {
            return Integer.hashCode(ID);
        }
        @Override
        public String toString() {
            return String.format("Animal: ID=%d, nacimiento=%tF, estado=%s", ID, nacimiento, estadoAnimal);
        }
    }
\end{lstlisting}

