\section{Diseño del Código de Andamiaje}

\subsection{Introducción}
En este apartado, se describe el diseño del código de andamiaje necesario para implementar el modelo de gestión del refugio. Las clases principales y las estructuras de datos se han implementado para garantizar la correcta funcionalidad y la relación entre las entidades definidas. Se justifica la elección de estructuras de datos como \texttt{Set} para evitar duplicados y se asegura la integridad mediante el uso de aserciones (\texttt{assert}).

\subsection{Implementación del Modelo}

\subsubsection{Clase Socio}
La clase \texttt{Socio} es abstracta y representa la base para las subclases \texttt{Adoptante}, \texttt{Voluntario}, y \texttt{Donante}. Esta clase asegura que cada socio tenga un \texttt{ID} único, una fecha de registro válida y un refugio asociado.

\begin{lstlisting}[language=Java]
public abstract class Socio {
    private int ID; 
    private Date fecha;
    private final Refugio refugioAsociado;

    public Socio(int ID, Date fecha, Refugio refugioAsociado) {
        assert ID > 0 : "El ID del socio debe ser valido.";
        assert fecha != null : "La fecha de registro no puede ser nula.";
        assert refugioAsociado != null : "El refugio asociado no puede ser nulo.";
        this.ID = ID;
        this.fecha = fecha;
        this.refugioAsociado = refugioAsociado;
    }
    public int getID() {
        return ID;
    }
    public Date getDate() {
        return this.fecha;
    }
    public Refugio getRefugio() {
        return this.refugioAsociado;
    }
}
\end{lstlisting}

\subsubsection{Clase Donante}
La clase \texttt{Donante} extiende de \texttt{Socio} y gestiona las donaciones realizadas por un socio. Las donaciones se almacenan en un \texttt{Set} para evitar duplicados.

\begin{lstlisting}[language=Java]
public class Donante extends Socio {
    private Set<Donacion> donaciones;

    public Donante(int ID, Date date, Refugio r, Double cantidad) {
        super(ID, date, r);
        assert cantidad > 0 : "La cantidad inicial donada debe ser mayor a cero.";
        donaciones = new HashSet<>();
        this.donar(cantidad);
    }

    public void donar(Double cantidad) {
        assert cantidad > 0 : "La cantidad donada debe ser mayor a cero.";
        LocalDate fechaDonacion = LocalDate.now();
        Donacion d = new Donacion(cantidad, Date.from(fechaDonacion.atStartOfDay(ZoneId.systemDefault()).toInstant()), this);
        donaciones.add(d);
        Refugio r = super.getRefugio();
        r.setLiquidez(r.getLiquidez() + cantidad);
        r.addSocio(this);
        assert donaciones.contains(d);
    }
}
\end{lstlisting}

\subsubsection{Clase Adoptante}
La clase \texttt{Adoptante} extiende de \texttt{Socio} y gestiona las adopciones realizadas por un adoptante. Las adopciones se almacenan en un \texttt{Set}.

\begin{lstlisting}[language=Java]
public class Adoptante extends Socio {
    private Set<Adopcion> adopciones;

    public Adoptante(int ID, Date date, Refugio r) {
        super(ID, date, r);
        adopciones = new HashSet<>();
    }

    public void adoptar(Animal a, Voluntario v) {
        assert !adopciones.stream().anyMatch(ad -> ad.getAnimal().equals(a)) : "El adoptante ya tiene registrado este animal";
        v.tramitarAdopcion(a, this);
    }

    public void addAdopcion(Adopcion a) {
        adopciones.add(a);
    }
}
\end{lstlisting}

\subsubsection{Clase Voluntario}
La clase \texttt{Voluntario} extiende de \texttt{Socio} y gestiona los trámites de adopción realizados por un voluntario.

\begin{lstlisting}[language=Java]
public class Voluntario extends Socio {
    Set<Adopcion> tramites;

    public Voluntario(int ID, Date date, Refugio r) {
        super(ID, date, r);
        tramites = new HashSet<>();
    }

    public void tramitarAdopcion(Animal a, Adoptante ad) {
        assert a.getEstadoAnimal() == EstadoAnimal.DISPONIBLE : "El animal ya esta adoptado.";
        LocalDate fechaAdopcion = LocalDate.now();
        Adopcion adopcion = new Adopcion(a, ad, this, Date.from(fechaAdopcion.atStartOfDay(ZoneId.systemDefault()).toInstant()));
        tramites.add(adopcion);
    }
}
\end{lstlisting}

\subsubsection{Clase Refugio}
La clase \texttt{Refugio} gestiona el conjunto de \texttt{Socios} y \texttt{Animales}. Las operaciones están centralizadas para simplificar la gestión.

\begin{lstlisting}[language=Java]
public class Refugio {
    private double liquidez;
    private Set<Animal> animalesRegistrados;
    private Set<Socio> socios;

    public Refugio(double liquidez) {
        assert liquidez >= 0 : "La liquidez debe ser no negativa.";
        this.liquidez = liquidez;
        animalesRegistrados = new HashSet<>();
        socios = new HashSet<>();
    }

    public void addSocio(Socio s) {
        assert s != null : "El socio no puede ser nulo.";
        socios.add(s);
    }
}
\end{lstlisting}

\subsubsection{Clase Donacion}
La clase \texttt{Donacion} representa una donación realizada por un \texttt{Donante}. Incluye la cantidad, la fecha de la donación y el donante asociado. Las validaciones aseguran que los valores sean válidos en el momento de la creación de la instancia.

\begin{lstlisting}[language=Java]
public class Donacion {
    private Double cantidad;
    private Date date;
    private final Donante donante;

    public Donacion(Double cantidad, Date date, Donante donante) {
        assert cantidad != null && cantidad > 0 : "La cantidad debe ser positiva.";
        assert date != null && !date.after(new Date()) : "La fecha no puede ser nula ni estar en el futuro.";
        assert donante != null : "El donante no puede ser nulo.";
        this.cantidad = cantidad;
        this.date = date;
        this.donante = donante;
    }

    public Double getCantidad() {
        assert cantidad != null && cantidad > 0 : "La cantidad no puede ser nula.";
        return cantidad;
    }

    public void setCantidad(Double cantidad) {
        this.cantidad = cantidad;
    }

    public Date getDate() {
        assert date != null : "La fecha no puede ser nula.";
        return date;
    }

    public void setDate(Date date) {
        this.date = date;
    }

    public Donante getDonante() {
        return this.donante;
    }

    @Override
    public String toString() {
        return String.format("Donacion: %.2f, %tY-%tB-%td", cantidad, date, date, date);
    }
}
\end{lstlisting}

\subsubsection{Clase Adopcion}
La clase \texttt{Adopcion} modela una adopción de un \texttt{Animal} realizada por un \texttt{Adoptante}, gestionada por un \texttt{Voluntario}. Implementa la bidireccionalidad entre estas entidades para mantener consistencia en las asociaciones.

\begin{lstlisting}[language=Java]
public class Adopcion {
    private Date fecha;
    final private Animal animal;
    final private Adoptante adoptante;
    final private Voluntario voluntario;

    public Adopcion(Animal a, Adoptante ad, Voluntario v, Date fecha) {
        assert a != null : "El animal no puede ser nulo.";
        assert ad != null : "El adoptante no puede ser nulo.";
        assert v != null : "El voluntario no puede ser nulo.";
        assert a.getEstadoAnimal() == EstadoAnimal.DISPONIBLE : "El animal debe estar disponible para adopcion.";
        assert fecha != null && !fecha.after(new Date()) : "La fecha no puede ser nula ni estar en el futuro.";

        this.animal = a;
        this.adoptante = ad;
        this.voluntario = v;
        this.fecha = fecha;

        a.setEstadoAnimal(EstadoAnimal.ADOPTADO);
        ad.addAdopcion(this);
        assert Collections.list(ad.getAdopciones()).contains(this) : 
        "La adopcion no fue anadida correctamente al adoptante.";
        v.addTramite(this);
        assert Collections.list(v.getTramites()).contains(this) : 
        "La adopcion no fue anadida correctamente al voluntario.";
    }

    public Date getFecha() {
        return this.fecha;
    }

    public void setFecha(Date fecha) {
        assert fecha != null && !fecha.after(new Date()) : "La fecha no puede ser nula ni estar en el futuro";
        this.fecha = fecha;
    }

    public Animal getAnimal() {
        return this.animal;
    }

    public Voluntario getVoluntario() {
        return this.voluntario;
    }

    public Adoptante getAdoptante() {
        return this.adoptante;
    }

    @Override
    public String toString() {
        return String.format("Adopcion: %tY-%tB-%td, %s, %s", fecha, fecha, fecha, animal, adoptante);
    }
}
\end{lstlisting}

\subsubsection{Clase Animal}
La clase \texttt{Animal} modela a un animal registrado en el sistema. Cada animal tiene un ID único, una fecha de nacimiento, un estado actual y está asociado a un \texttt{Refugio}.

\begin{lstlisting}[language=Java]
public class Animal {
    private int ID;
    private Date nacimiento;
    private EstadoAnimal estadoAnimal;
    final private Refugio refugio;
    private Adopcion adopcion;

    public Animal(int ID, Date nacimiento, EstadoAnimal estadoAnimal, Refugio refugio, Adopcion adopcion) {
        assert ID > 0 : "El ID del animal debe ser valido.";
        assert nacimiento != null : "La fecha de nacimiento no puede ser nula.";
        assert estadoAnimal != null : "El estado del animal debe estar definido.";
        assert refugio != null : "El refugio debe existir.";

        this.ID = ID;
        this.nacimiento = nacimiento;
        this.estadoAnimal = estadoAnimal;
        this.refugio = refugio;
        this.adopcion = adopcion;
    }

    public EstadoAnimal getEstadoAnimal() {
        return estadoAnimal;
    }

    public void setEstadoAnimal(EstadoAnimal estadoAnimal) {
        assert estadoAnimal != null : "El estado del animal debe estar definido.";
        this.estadoAnimal = estadoAnimal;
    }

    public Date getNacimiento() {
        return nacimiento;
    }

    public void setNacimiento(Date nacimiento) {
        assert nacimiento != null : "La fecha de nacimiento no puede ser nula";
        this.nacimiento = nacimiento;
    }

    public Refugio getRefugio() {
        return refugio;
    }

    public Adopcion getAdopcion() {
        return this.adopcion;
    }

    public void setAdopcion(Adopcion adopcion) {
        assert adopcion != null;
        this.adopcion = adopcion;
    }

    @Override
    public String toString() {
        return String.format("Animal: ID=%d, nacimiento=%tF, estado=%s", ID, nacimiento, estadoAnimal);
    }
}
\end{lstlisting}

\subsection{Conclusión}
El código de andamiaje diseñado utiliza \texttt{Set} para evitar duplicados y asegura la bidireccionalidad de las asociaciones entre clases mediante comprobaciones con \texttt{assert}. Esto garantiza la consistencia e integridad del modelo.
