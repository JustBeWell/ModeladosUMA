\documentclass[12pt, a4paper, titlepage]{article}

\usepackage[utf8]{inputenc}     % Permite el uso de caracteres como ñ y acentos
\usepackage[spanish]{babel}     % Configura el documento en español
\usepackage{graphicx}           % Para manipular gráficos e imágenes en el documento
\usepackage{float}              % Permite forzar una ubicación exacta de imágenes con [H]
\usepackage{listings}           % Permite formato de fragmentos de código de programación
\usepackage[autostyle=false, style=english]{csquotes} % Permite escribir "" con \enquote{}
\usepackage[explicit]{titlesec} % Permite personalizar el estilo de los títulos y secciones
\usepackage{xcolor}             % Para definir y usar colores personalizados en texto
\usepackage{geometry}           % Para configuración de los márgenes y el tamaño de la página
\usepackage{lipsum}             % Para generar texto de relleno ("Lorem ipsum")
\usepackage{tocloft}            % Para personalizar el formato del índice
\usepackage{subfiles}           % Para incluir de otros archivos .tex en el main mediante \include{}
\usepackage[colorlinks=true, allcolors=blue, linktoc=all]{hyperref} % Crea enlaces dentro del documento 
\usepackage{bookmark}           % Mejora la administración de los marcadores (bookmarks) en documentos PDF generados
\usepackage{xr}                 % Permite referenciar elementos de otros documentos .tex
\usepackage{natbib}             % Para gestionar bibliografías y hacer citaciones
\usepackage{varioref}

% Configuración personalizada para las labels en español (por defecto están en inglés)
\renewcommand{\reftextfaceafter}{en la página siguiente}
\renewcommand{\reftextfacebefore}{en la página anterior}
\renewcommand{\reftextafter}{en la página~\thevpagerefnum}
\renewcommand{\reftextbefore}{en la página~\thevpagerefnum}
\renewcommand{\reftextcurrent}{en esta página}
\renewcommand{\reftextfaraway}[1]{en la página~\pageref{#1}}

% Configuración de márgenes
\geometry{
    left=2.5cm,  % Margen izquierdo
    right=2.5cm, % Margen derecho
    top=3cm,     % Margen superior
    bottom=3cm   % Margen inferior
}


% Definimos una nueva forma de referirnos a las \section, \subsection y \subsubsection.
% Ahora en los subarchivos tex al llamarlas de esta forma aseguramos que en la
% Table of Contents (ToC) aparezcan los números de las secciones que seleccionemos
%     \subsection*{} hace que salga el titulo en formato subsection pero no aparece en el ToC
%     \numberedsubsection hará que si salga en la ToC


% ESTA VERSION NO TIENE NÚMEROS EN LOS TÍTULOS     Section, Subsection... 
\newcommand{\numberedsection}[1]{%
  \stepcounter{section}%
  \section*{#1}%
  \addcontentsline{toc}{section}{\protect\numberline{\thesection}#1}%
}
\newcommand{\numberedsubsection}[1]{%
  \stepcounter{subsection}%
  \subsection*{#1}%
  \addcontentsline{toc}{subsection}{\protect\numberline{\thesubsection}#1}%
}
\newcommand{\numberedsubsubsection}[1]{%
  \stepcounter{subsubsection}%
  \subsubsection*{#1}%
  \addcontentsline{toc}{subsubsection}{\protect\numberline{\thesubsubsection}#1}%
}

% ESTA VERSIÓN LE PONE NÚMEROS A LOS TÍTULOS:   1. Section, 1.1 Subsection... 
% \newcommand{\numberedsection}[1]{
%   \stepcounter{section}
%   \section*{\thesection\hspace{0.5em}#1}
%   \addcontentsline{toc}{section}{\protect\numberline{\thesection}#1}
% }

% \newcommand{\numberedsubsection}[1]{
%   \stepcounter{subsection}
%   \subsection*{\thesubsection\hspace{0.5em}#1}
%   \addcontentsline{toc}{subsection}{\protect\numberline{\thesubsection}#1}
% }

% \newcommand{\numberedsubsubsection}[1]{
%   \stepcounter{subsubsection}
%   \subsubsection*{\thesubsubsection\hspace{0.5em}#1}
%   \addcontentsline{toc}{subsubsection}{\protect\numberline{\thesubsubsection}#1}
% }

% Definimos comando para realizar un tab
\newcommand\tab[1][1cm]{\hspace*{#1}} 

% Definimos forma de llamar al listado con letras minúsculas pero sustituye el listado por 
% números de enumerate. Comentar para tener enumerate por default.
% \renewcommand{\theenumi}{\alph{enumi}}
% \begin{enumerate}
%   \item
% \end{enumerate}


% Configuración para sintaxis Java
\lstdefinestyle{javaNormal}{
    language=Java,                        % Especifica Java como el lenguaje
    basicstyle=\ttfamily\small,           % Estilo básico: letra monoespaciada pequeña
    keywordstyle=\color{orange!90!black}, % Color para palabras clave
    commentstyle=\color{gray},            % Color para comentarios
    stringstyle=\color{green!50!black},   % Color para cadenas de texto
    identifierstyle=\color{black},        % Color para identificadores
    tabsize=2,                            % Tamaño del tabulador
    showspaces=false,                     % No mostrar espacios
    showstringspaces=false,               % No mostrar espacios en cadenas de texto
    breaklines=true                       % Ajustar líneas largas
}

% Configuración para sintaxis Java con fondo
\lstdefinestyle{javaEspecifico}{
    language=Java,                        % Especifica Java como el lenguaje
    basicstyle=\ttfamily\small,           % Estilo básico: letra monoespaciada pequeña
    backgroundcolor=\color{gray!10},      % Fondo gris claro
    keywordstyle=\color{orange!90!black}, % Color para palabras clave
    commentstyle=\color{gray},            % Color para comentarios
    stringstyle=\color{green!50!black},   % Color para cadenas de texto
    numberstyle=\tiny\color{gray},        % Color de los números de línea
    stepnumber=1,                         % Mostrar un número de línea en cada línea
    numbersep=10pt,                       % Separación de los números del código
    tabsize=2,                            % Tamaño del tabulador
    showspaces=false,                     % No mostrar espacios
    showstringspaces=false,               % No mostrar espacios en cadenas de texto
    breaklines=true,                      % Ajustar líneas largas
    frame=single                          % Añadir un marco alrededor del código
}

\begin{document}

% PORTADA
\begin{titlepage}
  \centering
  {\bfseries\LARGE Universidad de Málaga\par}
  \vspace{1cm}
  {\scshape\Large ETSI Informática\par}
  \vspace{2cm}
  {\scshape\Huge Patrones de diseño aplicados\par}
  \vspace{0.1cm}
  {\scshape\Huge a un sistema de Alquiler de coches}
  \vspace{2cm}
  \begin{figure}[H]
      \centering
       \includegraphics[width=0.30\linewidth]{assets/umaLogo.png}
  \end{figure}
  \vfill
  {\scshape\Large Modelado y Diseño del Software (2024$-$25)\par}
  \vspace{0.5cm}
  {\Large Daniil Gumeniuk\par}
  {\Large Angel Bayon Pazos\par}
  {\Large Diego Sicre Cortizo\par}
  {\Large Pablo Ortega Serapio\par}
  {\Large Angel Nicolás Escaño López\par}
  {\Large Francisco Javier Jordá Garay\par}
  {\Large Janine Bernadeth Olegario Laguit\par}
  \vspace{1cm}
  {\Large Grupo 1.1}
  \vfill
  {\Large Diciembre 2024}
  
  \end{titlepage}
% FIN PORTADA  

% ÍNDICE
\tableofcontents % Crea el Índice
\thispagestyle{empty} % Quita el número de la primera página
\addtocontents{toc}{\protect\thispagestyle{empty}} % Asegura que cada página del índice sea sin número de página

\newpage

\listoffigures % Crea un Índice de Figuras (registra imágenes)
\thispagestyle{empty}
\addtocontents{lof}{\protect\thispagestyle{empty}}

\newpage
% FIN ÍNDICE


% RESUMEN
\thispagestyle{empty}
\begin{abstract}
  Esta práctica tiene como objetivo el diseño e implementación de un sistema de gestión de alquileres de coches para una 
  empresa. El proyecto se centra en desarrollar las funcionalidades necesarias para gestionar el proceso de alquiler, las 
  restricciones de negocio y las operaciones específicas requeridas asi como aplicar \emph{Patrones de Diseño} presentados 
  en el Tema 6.\par
  \vspace{0.5cm}
  Estructuramos el trabajo de tal forma que para cada uno de los apartados presentamos en esta memoria, se desarrolla y 
  documenta un modelo UML hecho en \emph{Visual Paradigm} detallado ajustándose a las necesidades de cada sección. También 
  se justifican las posibles contradicciones o lagunas encontradas en el enunciado. Además, se explican las decisiones de diseño 
  tomadas para la implementación del código en lenguaje \emph{Java} respetando las restricciones impuestas por el enunciado.\par
  \vspace{0.5cm}
  Todas las implementaciones parten del mismo esqueleto que se presenta en la Introducción de la memoria, detallando en cada apartado 
  las clases que se ven afectadas al implementar el \emph{Patrón de Diseño} escogido (en caso que fuera necesario) y una explicación de 
  como ese patrón enriquece el funcionamiento del sistema original.\par
  \vspace{0.5cm}
  Los metodos protected del sistema asumen que el usuario accedera al main fuera del paquete del código fuente.\par
\end{abstract}


\newpage
% FIN RESUMEN


% CUERPO DEL DOCUMENTO
\setcounter{page}{4} % Inicia a contar las páginas a partir de {}

\section{Apartado A}
\subsection{Diseño del Código de Andamiaje}
\subsection*{Introducción}
El concepto de \enquote{código de andamiaje} en el diseño orientado a objetos, 
particularmente en Java, hace referencia al conjunto de estructuras y métodos 
necesarios para implementar asociaciones entre clases, asegurando la 
consistencia y la integridad del sistema.\par
\vspace{0.15cm}
Su propósito es proporcionar un marco inicial sobre el que los desarrolladores pueden 
construir las funcionalidades particulares de un proyecto. Sin embargo, la línea entre
\enquote{andamiaje} e \enquote{implementación completa} puede ser fina ya que estamos 
añadiendo nuevas funcionalidades que luego se convierten en el nuevo marco inicial para 
futuros cambios que vayamos a realizar en los próximos apartados. A continuación se 
exponen las decisiones diseño que completan el andamiaje inicial.

\subsection{Análisis de opciones de Diseño}
En esta sección se exponen las diferentes opciones de diseño para la implementación 
del sistema de gestión de un refugio de animales, conforme al modelo de clases y 
operaciones proporcionado.\par
\vspace{0.15cm}
El sistema requiere gestionar los socios del refugio (quienes pueden desempeñar 
diferentes roles como voluntarios, donantes y adoptantes), así como el registro, 
adopción y donación de animales. Además, se deben considerar las relaciones entre 
las entidades (socios, animales, refugio, donaciones) y las restricciones del sistema, 
como la consistencia en los datos y las operaciones.\par
\vspace{0.15cm}
Importante mencionar que en nuestro diseño, \textbf{no hemos aplicado un único enfoque de manera 
exclusiva}. Hemos adoptado una combinación de estrategias dependiendo de las necesidades de 
cada relación dentro del sistema justificándolo de forma adecuada.

\subsubsection{Manejo de las Asociaciones}

\begin{description}
    \item[a)] \textbf{Asociación Directa (Sin Reificación\footnote{
        La reificación es una técnica en programación 
        orientada a objetos que se basa en convertir un 
        concepto abstracto, como una relación, en una 
        entidad concreta o clase.})}
\end{description}

\textit{\textbf{Descripción:}}  
En este enfoque, las asociaciones entre clases se implementan directamente como atributos 
en las clases relacionadas.\par
\vspace{0.15cm}
Esta práctica es fácil de implementar porque la cantidad de clases a gestionar 
y el número de clases necesarias para representar las relaciones es menor. No obstante, 
añadir atributos adicionales a las asociaciones (como fechas en el proceso 
de adopción) puede traer problemas de consistencia al manejar relaciones complejas como la 
de un socio con múltiples roles (lo exploraremos en futuras secciones).\par
\newpage % Para que no se ve aglomerada la definición en la footnote
\textbf{Ejemplo: Implementación de \texttt{Refugio} con asociación directa a \texttt{Animal}}\par
La asociación es directa porque \texttt{Refugio} gestiona los \texttt{Animales} mediante un 
\texttt{set}, sin una clase intermedia que relacione ambas entidades.(ver Código~\ref{codigo:refugio})\par
TODO: CAMBIAR EL CODE\par
En nuestra implementación, hemos decidido la utilización de un \texttt{Set} para las estructuras de datos 
en vez de \texttt{List} por varios factores:

\begin{itemize}
    \item \textbf{Control de elementos repetidos:} Los \texttt{Set} usan los métodos de \texttt{hashCode} e \texttt{equals}
    para hacer la comprobación de la existencia de elementos en la colección. Si un nuevo elemento coincide con uno existente, 
    no se inserta, evitando comprobaciones adicionales que haríamos con el uso de \texttt{List}.
    \item \textbf{Complejidad algorítmica:} En los \texttt{HashSet}, la búsqueda y la inserción tienen una complejidad de \(O(1)\), 
    ya que se basan en tablas hash. Por otro lado, en una \texttt{List} (como \texttt{ArrayList}), las operaciones de búsqueda tienen una 
    complejidad de \(O(n)\), porque requiere iterar sobre los elementos.
    \item \textbf{Orden de los elementos:} En nuestra implementación, no es necesario mantener un orden específico en las colecciones. 
    Por esta razón, el uso \texttt{Set} es más adecuado que \texttt{List} teniendo en cuenta los puntos anteriores.
\end{itemize}
\vspace{0.45cm}

\begin{description}
    \item[b)] \textbf{Reificación de la Asociación (Clase de Asociación)}
\end{description}

\textit{\textbf{Descripción:}}  
En este enfoque, las asociaciones complejas entre clases se modelan mediante clases 
intermedias. Por ejemplo, la clase \texttt{Adopcion} representa la relación entre un 
\texttt{Animal}, un \texttt{Adoptante}, y un \texttt{Voluntario}, incluyendo atributos 
como \texttt{fechaAdopcion} para capturar detalles específicos de la relación.\par
\vspace{0.15cm}
Como la reificación nos permiten agregar atributos y métodos específicos a las relaciones,
facilita la implementación de restricciones complejas relacionadas con la asociación.
Sin embargo, aumenta el número de clases y relaciones a gestionar, lo que hace el diseño más denso
ya que debemos implementar y gestionar las clases de asociación, así como los métodos para 
acceder a las relaciones.\par
\vspace{0.15cm}

\textbf{Nuestra Implementación: Uso de \texttt{Adopcion} como clase de asociación:}
\begin{itemize}
    \item Se representa la relación entre \texttt{Animal}, \texttt{Adoptante} y 
    \texttt{Voluntario} mediante una clase intermedia. (ver Código~\ref{codigo:adopcion})
    \item Atributos como \texttt{fecha} añaden flexibilidad al modelo, permitiendo 
    capturar detalles adicionales de la relación.
    \item Se gestionan las relaciones bidireccionales entre las entidades involucradas, 
    asegurando consistencia en los datos.
\end{itemize}

Elegimos este enfoque, ya que proporciona la flexibilidad necesaria para agregar atributos 
y gestionar reglas de negocio específicas. La asociación directa fue descartada porque no 
permitiría capturar detalles adicionales, como la fecha de adopción, ni manejar eficientemente 
las restricciones relacionadas con el proceso de adopción.



\subsubsection{Manejo de Roles de los Socios}

\begin{description}
    \item[a)] \textbf{Subclases Específicas para cada Rol}
\end{description}

\textit{\textbf{Descripción:}}  
Cada rol (\texttt{Voluntario}, \texttt{Donante}, \texttt{Adoptante}) se implementa como una 
subclase de la clase \texttt{Socio}. Esto permite encapsular los atributos y métodos 
específicos de cada rol dentro de su respectiva subclase.
\vspace{0.15cm}


Esto proporciona claridad al diseño, ya que cada rol está claramente representado con 
métodos específicos para su comportamiento. Además, permite encapsular los atributos y 
métodos particulares de cada tipo de socio, lo que mejora la organización y legibilidad 
del código. Aunque tiene una limitación significativa para manejar roles múltiples, ya que no permite que 
un socio asuma más de un rol sin duplicar instancias de las subclases. Esto hace que el 
diseño sea rígido y menos flexible en casos donde los roles pueden cambiar dinámicamente 
o coexistir (volveremos a hablar de esto en los siguientes apartados).\par
\vspace{0.15cm}



\textbf{Ejemplo: Subclases específicas para los roles}\par
El diseño tiene implementadas las subclases \texttt{Donante}, \texttt{Adoptante}, 
y \texttt{Voluntario} como extensiones de la clase \texttt{Socio}.\par
(ver Código~\ref{codigo:donante}, \ref{codigo:adoptante}, \ref{codigo:voluntario})

\begin{description}
    \item[b)] \textbf{Uso de Composición de Roles}
\end{description}

\textit{\textbf{Descripción:}}  
En lugar de modelar cada rol como una subclase de \texttt{Socio}, este enfoque utiliza 
la composición para permitir que un socio tenga múltiples roles simultáneamente. 
Cada rol se modela como una clase independiente que puede ser asociada dinámicamente a 
un \texttt{Socio} mediante una colección de roles.
\vspace{0.15cm}


Este enfoque es mucho más flexible, ya que permite asignar múltiples roles a un socio 
sin necesidad de crear combinaciones de subclases. También simplifica el manejo de 
roles dinámicos y permite cambios en tiempo de ejecución.
Puede reducir la claridad del diseño, ya que no existe una distinción explícita entre 
los diferentes tipos de socios. Además, requiere implementar lógica adicional para 
validar qué operaciones son aplicables para los roles asignados a cada socio.\par
\vspace{0.15cm}


\textbf{Nuestra implementación: Uso de Subclases Específicas para cada Rol:}\par  
\vspace{0.15cm}
Hemos decidido no cambiar como están implementados los roles mediante subclases específicas 
(para este apartado) en lugar de composición. Esto se debe a que en nuestro modelo actual, 
los roles están claramente definidos y no se requiere que un socio tenga múltiples roles de 
manera simultánea. Además:
\begin{itemize}
    \item La claridad y encapsulación que proporciona la herencia permiten manejar las responsabilidades y comportamientos específicos de cada tipo de socio de manera aislada.
    \item Aunque la composición sería más flexible, introducirá complejidad adicional innecesaria para los requisitos actuales del sistema.
\end{itemize}

No obstante, los requisitos del sistema cambiarán en futuros apartados pidiendo que un socio
tenga múltiples roles simultáneamente. En el apartado correspondiente, se discute porque 
la composición sería una solución más adecuada y como se ha implementado.




\subsubsection{Consistencia y Gestión de Datos}

\begin{description}
    \item[a)] \textbf{Encapsulación Estricta}
\end{description}

\textit{\textbf{Descripción:}}  
Este enfoque restringe el acceso directo a los atributos y métodos de las clases mediante 
el uso de visibilidad privada. Para interactuar con los atributos, se proporcionan métodos 
públicos controlados (\texttt{getters} y \texttt{setters}) que incluyen validaciones (mediante 
\texttt{asserts}) para garantizar que los datos se mantengan en un estado consistente.
\vspace{0.15cm}

    Debido a esto nos aseguramos que los datos sean modificados de manera controlada y consistente.
    Facilita la incorporación de validaciones o pruebas unitarias lo que completa el comportamiento del
    esperado del sistema.Dicho esto también se requiere de implementar más métodos, como \texttt{getters}, \texttt{setters} y validaciones 
    necesarias, lo que aumenta la cantidad de código. Además, estas validaciones podrían hacer 
    que el diseño sea más extenso y menos directo.\par
    \vspace{0.15cm}


\textbf{Ejemplo en el sistema: Uso de encapsulación estricta en la clase \texttt{Animal}:}\par  
En nuestro sistema, la clase \texttt{Animal} utiliza atributos privados y métodos públicos 
controlados para garantizar consistencia y validaciones en tiempo de ejecución.Este enfoque 
asegura que cualquier intento de modificar el estado de un \texttt{Animal} pase por 
validaciones definidas en los métodos públicos. (ver Código~\ref{codigo:animal})


\begin{description}
    \item[b)] \textbf{Uso de Colecciones Inmutables}
\end{description}

\textit{\textbf{Descripción:}}  
En este enfoque, las colecciones utilizadas para representar relaciones (por ejemplo, 
listas o conjuntos de \texttt{Animal} en \texttt{Refugio}) son inmutables. Esto garantiza 
que las relaciones no puedan ser modificadas accidentalmente fuera de las clases que las 
gestionan.
\vspace{0.15cm}

    Mejora la integridad del sistema al garantizar que las relaciones no se modifiquen de 
    manera no controlada.
    Por otro lado introduce rigidez ya que no permite realizar cambios dinámicos en las relaciones sin 
    reemplazar completamente la colección. Esto puede dificultar la gestión de operaciones 
    como agregar o eliminar elementos.\par
    \vspace{0.15cm}

\textbf{Ejemplo de Uso de Colecciones Inmutables en la Clase \texttt{Refugio}:}\par  
En nuestro sistema, el método \texttt{getAnimalesRegistrados} de la clase \texttt{Refugio} 
devuelve una vista inmutable de los animales registrados. Esto asegura que las listas no 
puedan modificarse desde fuera de la clase. Además, el uso de \texttt{Collections.}\texttt{enumeration} 
garantiza que la colección de animales no pueda ser alterada fuera de la clase \texttt{Refugio}, 
manteniendo la consistencia de los datos. (ver Código~\ref{codigo:refugio} aunque se implementa en varias clases)

\textbf{Decisión Tomada: Encapsulación Controlada con Enumerations:}\par
En nuestro diseño, optamos por una encapsulación controlada en lugar de colecciones 
completamente inmutables. Esto se debe a que:
\begin{itemize}
    \item Proporciona flexibilidad para realizar cambios dinámicos en las colecciones a 
    través de métodos controlados, lo que es necesario para operaciones como agregar o 
    eliminar animales en un refugio.
    \item Utilizar enumeraciones en los métodos \texttt{get} garantiza que las colecciones 
    no se modifiquen desde fuera de las clases, preservando la integridad de los datos.
\end{itemize}

Este enfoque combina lo mejor de ambos mundos: flexibilidad para realizar cambios controlados 
y protección contra modificaciones accidentales.



\subsubsection{Estrategias para Manejo de Adopciones y Donaciones}

\begin{description}
    \item[a)] \textbf{Operaciones Independientes}
\end{description}

\textit{\textbf{Descripción:}}  
En este enfoque, cada operación (como adopción, registro de animales o donaciones) se 
implementa de forma independiente, sin interacciones entre ellas. Cada acción tiene su 
propio método o flujo lógico separado.
\vspace{0.15cm}

    Este diseño asegura que las operaciones están bien definidas y separadas, lo que 
    facilita su comprensión. Además, la simplicidad del diseño permite que sea directo y 
    fácil de implementar.Sin Embargo, puede llevar a la duplicación de código si varias operaciones comparten lógica similar 
    (por ejemplo, validar la existencia de un animal o donante). También puede ser menos 
    flexible, ya que cualquier cambio en una operación podría requerir modificaciones en 
    múltiples partes del sistema.\par
    \vspace{0.15cm}

\textbf{Ejemplo: Operaciones independientes en nuestra implementación}  
En nuestro diseño, las adopciones y donaciones se gestionan mediante clases específicas 
(\texttt{Adopcion} y \texttt{Donacion}), cada una con su propia lógica y atributos.

\begin{description}
    \item[b)] \textbf{Reutilización de Lógica Compartida entre Operaciones}
\end{description}

\textit{\textbf{Descripción:}}  
En lugar de mantener las operaciones completamente separadas, este enfoque identifica y 
reutiliza lógica común entre las operaciones (como validaciones o actualizaciones de estado). 
Aunque no implementamos este enfoque en nuestra solución actual, sería posible centralizar las 
validaciones comunes mediante una clase auxiliar, como se muestra en el siguiente ejemplo.\par
\vspace{0.15cm}
Esta opción reduce la duplicación de código, ya que la lógica compartida se implementa una sola vez
y facilita la incorporación de nuevas funcionalidades relacionadas con las operaciones existentes. Pero 
puede introducir una dependencia más estrecha entre las clases, lo que podría aumentar 
la complejidad del sistema en caso de cambios importantes.

\textbf{Ejemplo Propuesto: Centralización de Validaciones}  
Aunque nuestra implementación actual gestiona las validaciones directamente en los métodos 
de las clases relevantes (\texttt{Adoptante}, \texttt{Donante}). Por ejemplo, podríamos considerar una 
clase auxiliar para centralizarlas en el futuro:

\begin{lstlisting}[style = javaEspecifico, language=Java, caption={Clase Auxiliar para Validaciones}] 
public class Validacion {
    public static void validarEstadoAnimal(Animal animal, EstadoAnimal estadoEsperado) {
        assert animal.getEstadoAnimal() == estadoEsperado : 
            "El estado del animal no coincide con el esperado.";
    }
}
\end{lstlisting}

En nuestra implementación actual, la validación se realiza directamente dentro de las clases:

\begin{lstlisting}[style = javaEspecifico, language=Java, caption={Manejo de validaciones dentro del método Adoptar}] 
public void adoptar(Animal a, Voluntario v) {
    assert a.getEstadoAnimal() == EstadoAnimal.DISPONIBLE : 
        "El animal no esta disponible.";
    Adopcion adopcion = new Adopcion(a, this, v, new Date());
    adopciones.add(adopcion);
}
\end{lstlisting}
\vspace{0.15cm}
\textbf{Decisión Tomada: Mantener las Validaciones en las Clases Relevantes}\par
En nuestra implementación actual, las validaciones se realizan directamente en las clases 
donde ocurren las operaciones. Esto se alinea con la claridad y simplicidad requeridas por 
el sistema. Sin embargo, reconocemos que la centralización de lógica compartida podría ser 
útil en sistemas más complejos. FIXME: EL GETTER ES MAL EJEMPLO DE ESTO (ver los getters en Código~\ref{codigo:adopcion} por ejemplo) 



\subsubsection{Representación de Relaciones en el Sistema}

\begin{description}
    \item[a)] \textbf{Relaciones Unidireccionales}
\end{description}

\textit{\textbf{Descripción:}}  
En una relación unidireccional, solo una entidad tiene conocimiento de la relación. 
Por ejemplo, un \texttt{Adoptante} puede conocer al \texttt{Animal} que adopta, 
pero el \texttt{Animal} no necesita saber nada sobre el \texttt{Adoptante}.\par
\vspace{0.15cm} 
Estas relaciones gestionan la relación con una clase, lo que reduce la complejidad del sistema.
El problema, es que limita las consultas entre clases relacionadas y puede volverse más complejo 
añadir funcionalidades.\par
\vspace{0.15cm}
En nuestro sistema, todas las relaciones unidireccionales con 1 a muchos, por ejemplo:
TODO: EXPANDIR ESTA IDEA

\begin{description}
    \item[b)] \textbf{Relaciones Bidireccionales}
\end{description}

\textit{\textbf{Descripción:}}  
En una relación bidireccional, ambas entidades conocen y mantienen referencias mutuas. 
Por ejemplo, cuando un \texttt{Adoptante} adopta un \texttt{Animal}, ambos se actualizan 
mutuamente para reflejar la relación.\par
\vspace{0.15cm}
Las relaciones de este estilo garantizan la consistencia de los datos, ya que ambas partes relacionadas están 
sincronizadas al mantener referencias mutuas explícitas. Sin embargo, como hay que tener una sincronización constante 
entre las todas las clases relacionadas, si tuviéramos muchas relaciones bidireccionales puede dificultar el mantenimiento
por que genera un alto nivel de acoplamiento.\par
\vspace{0.15cm}
\textbf{Ejemplo en el Sistema: Relaciones Bidireccionales en \texttt{Adopcion}}\par 
En nuestro diseño, la relación entre \texttt{Animal}, \texttt{Adoptante}, y 
\texttt{Voluntario} es bidireccional y se asegurando consistencia en ambas direcciones
reflejando los cambios realizados en una clase en las demás involucradas.
(ver Código~\ref{codigo:adopcion} como, por ejemplo, se actualiza el estado del animal tras ser adoptado)\par
\vspace{0.15cm}
\textbf{Decisión Tomada: Relaciones Bidireccionales}\par
Hemos implementado relaciones bidireccionales para las asociaciones complejas del sistema, 
como las adopciones, ya que garantizan consistencia y sincronización entre las entidades 
relacionadas. Sin embargo, para relaciones más simples, como la lista de animales en un 
refugio, usamos relaciones unidireccionales para mantener la simplicidad.

\subsection{Diagrama de Diseño}

\begin{figure}[H]
    \centering
     \includegraphics[width=1\linewidth]{assets/umaLogo.png}
     \caption{UMA}
\end{figure}
FALTA HACER Y METER EL DIAGRAMA DE DISEÑO DE NUESTRO SISTEMA.RECOMENDABLE PONER UNA MINI EXPLICACIÓN.

\newpage



\subsection{Implementación del Modelo}
\subsubsection{Clase Socio}\label{codigo:socio}
La clase \texttt{Socio} es abstracta y representa la base para las distintas subclases: 
\texttt{Adoptante}, \texttt{Voluntario}, y \texttt{Donante}. Esta clase asegura que 
cada socio tenga un \texttt{ID} único, una fecha de registro válida y un refugio asociado.

\begin{lstlisting}[style = javaNormal, language=Java] 
    package sistema;

    import java.util.Collections;
    import java.util.Date;
    
    public abstract class Socio {
        private int ID;
        private Date registro;
        private final Refugio refugioAsociado;
    
        public Socio(int ID, Date fechaRegistro, Refugio refugioAsociado) {
            assert ID > 0 : "El ID del socio debe ser valido.";
    
            assert fechaRegistro != null : "La fecha de registro no puede ser nula.";
    
            assert refugioAsociado != null : "El refugio asociado no puede ser nulo.";
    
            this.ID = ID;
            this.registro = fechaRegistro;
            this.refugioAsociado = refugioAsociado;
            refugioAsociado.addSocio(this);
    
            assert refugioAsociado.getSocios().hasMoreElements() && Collections.list(refugioAsociado.getSocios()).contains(this);
        }
    
        public int getID() {
            return this.ID;
        }
    
        private void setID(int ID) {
            assert ID > 0;
    
            assert Collections.list(this.getRefugio().getSocios()).stream().noneMatch((s) -> {
                return s.getID() == ID;
            }) : "El ID ya existe en el refugio";
    
            this.ID = ID;
        }
    
        public Date getRegistro() {
            return this.registro;
        }
    
        public void setRegistro(Date fechaRegistro) {
            assert fechaRegistro != null;
    
            this.registro = fechaRegistro;
        }
    
        public Refugio getRefugio() {
            return this.refugioAsociado;
        }
    
        public boolean equals(Object obj) {
            if (this == obj) {
                return true;
            } else if (obj instanceof Socio) {
                Socio socio = (Socio)obj;
                return this.ID == socio.ID;
            } else {
                return false;
            }
        }
    
        public int hashCode() {
            return Integer.hashCode(this.ID);
        }
    }
\end{lstlisting}



\subsubsection{Clase Donante}\label{codigo:donante}
La clase \texttt{Donante} extiende de \texttt{Socio} y gestiona las donaciones realizadas 
por un socio. Las donaciones se almacenan en un \texttt{HashSet} para asegurar que no hayan elementos repetidos y por eficiencia. (cómo se explicó en \ref{page:Consideraciones}) 
Por otro lado, hemos decidido implementar al crear un nuevo objeto Donante en el sistema,
el constructor llamará directamente al método donar ya que es una condición necesaria para ser donante. Las donaciones que sean modificadas son tratadas desde la clase Donación y
el \texttt{HashSet donaciones} se actualizará.


\begin{lstlisting}[style = javaNormal, language=Java] 
    package sistema;

    import java.time.LocalDate;
    import java.time.ZoneId;
    import java.util.*;
    
    public class Donante extends Socio{
        private Set<Donacion> donaciones;
        public Donante(int ID, Date date,Refugio r, float cantidad) {
            super(ID,date,r);
            assert cantidad > 0 : "La cantidad inicial donada debe ser mayor a cero.";
            donaciones = new HashSet<>();
            donar(cantidad);
        }

        public void donar(float cantidad){
            assert cantidad > 0 : "La cantidad donada debe ser mayor a cero.";
            assert Collections.list(this.getRefugio().getSocios()).contains(this): "El socio debe ser donante antes de poder donar";
            Donacion d = new Donacion(cantidad, Date.from(fechaDonacion.atStartOfDay(ZoneId.systemDefault()).toInstant()));
            addDonacion(d);
            Refugio r = super.getRefugio();
            r.setLiquidez(r.getLiquidez() + cantidad);
            assert donaciones.contains(d);
        }
    
        protected void addDonacion(Donacion donacion){
            assert donacion != null: "La donacion no puede ser nula";
            donaciones.add(donacion);
        }
    
        protected void removeDonacion(Donacion donacion){
            assert donacion != null : "La donacion no puede ser nula.";
            if (donaciones.contains(donacion) && donaciones.size() > 1) {
                donaciones.remove(donacion);
            } else if (donaciones.contains(donacion) && donaciones.size() == 1) {
                System.out.println("Todo donante debe tener al menos una donacion, estas intentando eliminar la unica donacion asociada a este socio donante");
            } else {
                System.out.println("Este socio no ha realizado la donacion que intentas eliminar");
            }
        }
        public Enumeration<Donacion> getDonaciones(){
            return Collections.enumeration(this.donaciones);
        }
    
        @Override
        public String toString() {
            return "Donante " + super.getID();
        }
    }
    
\end{lstlisting}

\subsubsection{Clase Adoptante}\label{codigo:adoptante}
La clase \texttt{Adoptante} extiende de \texttt{Socio} y simula las adopciones realizadas 
por un adoptante. Las adopciones se almacenan en un \texttt{HashSet} para evitar adopciones duplicadas.
\begin{lstlisting}[style = javaNormal, language=Java] 
    package sistema;

    import java.util.*;
    
    public class Adoptante extends Socio {
        private Set<Adopcion> adopciones;
    
        public Adoptante(int ID, Date date, Refugio r) {
            super(ID, date,r);
            adopciones = new HashSet<>();
        }
        public void adoptar(Animal a, Voluntario v) {
            assert a.getEstadoAnimal() == EstadoAnimal.DISPONIBLE: "El animal ya ha sido adoptado";
            Refugio refugioDelVoluntario = v.getRefugio();
            a.setEstadoAnimal(EstadoAnimal.ADOPTADO);
            refugioDelVoluntario.removeAnimalesRefugiados(a);
            v.tramitarAdopcion(a, this);
        }
    
        protected void addAdopcion(Adopcion a){
            this.adopciones.add(a);
        }
    
        protected void removeAdopcion(Adopcion a){
            if (adopciones.contains(a)) adopciones.remove(a);
            else System.out.println("Este animal ya no esta asociado al adoptante");
        }
        public Enumeration<Adopcion> getAdopciones(){
            return Collections.enumeration(adopciones);
        }
        @Override
        public boolean equals(Object obj) {
            if (this == obj) return true;
            if(obj instanceof Adoptante ){
                Adoptante adoptante = (Adoptante) obj;
                return adoptante.getID() == this.getID();
            }
            return false;
        }
        @Override
        public int hashCode() {
            return Integer.hashCode(this.getID());
        }
    
        @Override
        public String toString() {
            return "Adoptante " + super.getID();
        } 
    }    
\end{lstlisting}

\subsubsection{Clase Voluntario}\label{codigo:voluntario}
La clase \texttt{Voluntario} extiende de \texttt{Socio} y gestiona los trámites de 
adopción realizados por un voluntario. Los voluntarios se almacenan en un \texttt{HashSet},
evitando así voluntarios duplicados. 
\begin{lstlisting}[style = javaNormal, language=Java] 
    package sistema;

    import java.time.LocalDate;
    import java.time.ZoneId;
    import java.util.*;
    
    public class Voluntario extends Socio{
        Set<Adopcion> tramites;
        
        public Voluntario(int ID, Date date,Refugio r) {
            super(ID, date,r);
            tramites = new HashSet<>();
        }
        public void tramitarAdopcion(Animal a, Adoptante ad){
            assert ad != null : "El adoptante no puede ser nulo.";
            LocalDate fechaAdopcion = LocalDate.now();
            Adopcion adopcion = new Adopcion(a, ad, this, Date.from(fechaAdopcion.atStartOfDay(ZoneId.systemDefault()).toInstant()));
            addTramite(adopcion);
            ad.addAdopcion(adopcion);
        }
        public void registrar(Animal a){
            Refugio r = super.getRefugio();
            assert r != null : "El refugio asociado no puede ser nulo.";
            assert a != null : "El animal no puede ser nulo.";
            a.setEstadoAnimal(EstadoAnimal.DISPONIBLE);
            r.addAnimalesRefugiados(a);
    
        }
        public Enumeration<Adopcion> getTramites(){
            return Collections.enumeration(tramites);
        }
        protected void addTramite(Adopcion ad){
            assert ad != null : "El tramite de adopcion no puede ser nulo.";
            tramites.add(ad);
    
        }
        protected void removeTramite(Adopcion ad){
            assert  ad != null: "El tramite de adopcion no puede ser nulo.";
            tramites.remove(ad);
        }
    
    
        @Override
        public String toString() {
            return "Voluntario " + super.getID();
        }
    }
    
\end{lstlisting}

\subsubsection{Clase Refugio}\label{codigo:refugio}
La clase \texttt{Refugio} gestiona el conjunto de \texttt{Socios} y \texttt{Animales}.\par
\textbf{liquidez} está declarado como un \texttt{float} por las razones que se exponen en \ref{page:Consideraciones}.


\begin{lstlisting}[style = javaNormal, language=Java] 
package sistema;
import java.util.*;

public class Refugio {
    private float liquidez;
    private Set<Animal> animalesRegistrados;
    private Set<Animal> animalesRefugiados;
    private Set<Socio> socios;

    public Refugio(float liquidez) {
        assert liquidez >= 0 : "La liquidez debe ser no negativa.";
        this.liquidez = liquidez;
        animalesRefugiados = new HashSet<>();
        animalesRegistrados = new HashSet<>();
        socios = new HashSet<>();
    }

    public float getLiquidez() 
        return liquidez;
    }
    public void setLiquidez(float liquidez) {
        assert liquidez >= 0 : "La liquidez debe ser no negativa";
        this.liquidez = liquidez;
    }
    protected void addSocio(Socio s) {
        assert s != null : "El socio no puede ser nulo.";
        if(socios.contains(s)) {
            System.out.println("El socio ya esta registrado.");
            return;
        }
        socios.add(s);
    }
    protected void removeSocio(Socio s) {
        assert s != null : "El socio no puede ser nulo.";
        if (socios.contains(s)) {
            socios.remove(s);
        } else {
            System.out.println("Este socio no esta registrado en el refugio.");
        }
    }

    public Enumeration<Animal> getAnimalesRegistrados() {
        return Collections.enumeration(animalesRegistrados);
    }
    public Enumeration<Animal> getAnimalesRefugiados() {
        return Collections.enumeration(animalesRefugiados);
    }
    public Enumeration<Socio> getSocios() {
        return Collections.enumeration(socios);
    }

    public List<Adoptante> getAdoptantes() {
        List<Adoptante> adoptantes = new ArrayList<>();
        for (Socio s : socios) {
            if (s instanceof Adoptante) {
                adoptantes.add((Adoptante) s);
            }
        }
        return adoptantes;
    }
    public List<Voluntario> getVoluntarios() {
        List<Voluntario> voluntarios = new ArrayList<>();
        for (Socio s : socios) {
            if (s instanceof Voluntario) {
                voluntarios.add((Voluntario) s);
            }
        }
        return voluntarios;
    }
    public List<Donante> getDonantes() {
        List<Donante> donantes = new ArrayList<>();
        for (Socio s : socios) {
            if (s instanceof Donante) {
                donantes.add((Donante) s);
            }
        }
        return donantes;
    }
    public void registrar(Animal a){
        this.addAnimalesRegistrados(a);
    }
    protected void addAnimalesRefugiados(Animal a){
        assert a != null : "El animal no puede ser nulo.";
        if(!animalesRefugiados.contains(a)){
            animalesRefugiados.add(a);
            this.addAnimalesRegistrados(a);
        } else System.out.println("Este animal ya esta en el refugio.");
    }
    private void addAnimalesRegistrados(Animal a){
        assert a != null : "El animal no puede ser nulo.";
        if (!animalesRegistrados.contains(a)) {
            animalesRegistrados.add(a);
        } else {
            System.out.println("El animal ya esta registrado.");
        }
    }

    protected void removeAnimalesRefugiados(Animal a){
        assert a != null : "El animal no puede ser nulo.";
        if (animalesRefugiados.contains(a)) {
            animalesRefugiados.remove(a);
        } else {
            System.out.println("El animal no se encuentra en este Refugio.");
        }
    }
    protected void removeAnimalesRegistrados(Animal a){
        assert a != null : "El animal no puede ser nulo.";
        if (animalesRegistrados.contains(a) && animalesRegistrados.size() > 1) {
            animalesRegistrados.remove(a);
        } else if (animalesRegistrados.contains(a) && animalesRegistrados.size() == 1) {
            System.out.println("Todo refugio debe tener al menos un animal registrado, estas intentando eliminar el unico animal existente.");
        } else {
            System.out.println("El animal no se encuentra en este Refugio.");
        }
    }

    public void mostrarAnimalesRefugiados(){
        System.out.println(animalesRefugiados.toString());
    }
    public void mostrarAnimalesRegistrados(){
        System.out.println(animalesRegistrados.toString());
    }
    public void mostrarSocios() {
        for (Socio s : socios) {
            System.out.println(s);
        }
    }
    public void mostrarSociosPorTipo() {
        System.out.println("Adoptantes: " + getAdoptantes());
        System.out.println("Voluntarios: " + getVoluntarios());
        System.out.println("Donantes: " + getDonantes());
    }

    @Override
    public String toString() {
        StringBuilder sb = new StringBuilder();
        sb.append("Animales Registrados: ").append(animalesRegistrados).append("\n");
        sb.append("Animales Refugiados: ").append(animalesRefugiados).append("\n");
        sb.append("Socios: ").append(socios).append("\n");
        sb.append("Liquidez: ").append(liquidez);
        return sb.toString();
    }
}
\end{lstlisting}



\subsubsection{Clase Donacion}\label{codigo:donacion}
La clase \texttt{Donacion} representa una donación realizada por un \texttt{Donante}. 
Incluye la cantidad que como anteriormente mencionamos en \ref{page:Consideraciones} por temas de eficiencia es un \texttt{float},
la fecha de la donación y el donante asociado. Las validaciones 
aseguran que los valores sean válidos en el momento de la creación de la instancia.

\begin{lstlisting}[style = javaNormal, language=Java] 
    package sistema;

    import java.util.Date;
    import java.util.Objects;
    
    public class Donacion {
        private float cantidad;
        private Date date;
    
        public Donacion(float cantidad, Date date) {
            assert cantidad > 0.0F : "La cantidad debe ser positiva.";
    
            assert date != null && !date.after(new Date()) : "La fecha no puede ser nula ni estar en el futuro.";
    
            this.cantidad = cantidad;
            this.date = date;
        }
    
        public float getCantidad() {
            assert this.cantidad > 0.0F : "La cantidad no puede ser nula.";
    
            return this.cantidad;
        }
    
        public void setCantidad(float cantidad) {
            this.cantidad = cantidad;
        }
    
        public Date getDate() {
            assert this.date != null : "La fecha no puede ser nula.";
    
            return this.date;
        }
    
        public void setDate(Date date) {
            this.date = date;
        }
    
        public String toString() {
            return String.format("Donacion: %.2f, %tY-%tB-%td", this.cantidad, this.date, this.date, this.date);
        }
    
        public boolean equals(Object o) {
            if (this == o) {
                return true;
            } else if (o != null && this.getClass() == o.getClass()) {
                Donacion donacion = (Donacion)o;
                return Float.compare(this.cantidad, donacion.cantidad) == 0 && Objects.equals(this.date, donacion.date);
            } else {
                return false;
            }
        }
    
        public int hashCode() {
            return Objects.hash(new Object[]{this.cantidad, this.date});
        }
    }
\end{lstlisting}



\subsubsection{Clase Adopcion}\label{codigo:adopcion}
La clase \texttt{Adopcion} modela una adopción de un \texttt{Animal} realizada por un 
\texttt{Adoptante}, gestionada por un \texttt{Voluntario}. Implementa la bidireccionalidad 
entre estas entidades para mantener consistencia en las asociaciones.

\begin{lstlisting}[style = javaNormal, language=Java] 
    package sistema;

    import java.util.Date;
    
    public class Adopcion {
        private Date fecha;
        private final Animal animal;
        private final Adoptante adoptante;
        private final Voluntario voluntario;
    
        public Adopcion(Animal a, Adoptante ad, Voluntario v, Date fecha) {
            assert a != null : "El animal no puede ser nulo.";
    
            assert ad != null : "El adoptante no puede ser nulo.";
    
            assert v != null : "El voluntario no puede ser nulo.";
    
            assert fecha != null && !fecha.after(new Date()) : "La fecha no puede ser nula ni estar en el futuro.";
    
            this.animal = a;
            this.adoptante = ad;
            this.voluntario = v;
            this.fecha = fecha;
        }
    
        public Date getFecha() {
            return this.fecha;
        }
    
        public void setFecha(Date fecha) {
            assert fecha != null && !fecha.after(new Date()) : "La fecha no puede ser nula ni estar en el futuro";
    
            this.fecha = fecha;
        }
    
        public Animal getAnimal() {
            return this.animal;
        }
    
        public Voluntario getVoluntario() {
            return this.voluntario;
        }
    
        public Adoptante getAdoptante() {
            return this.adoptante;
        }
    
        public boolean equals(Object obj) {
            if (this == obj) {
                return true;
            } else if (!(obj instanceof Adopcion)) {
                return false;
            } else {
                Adopcion adopcion = (Adopcion)obj;
                boolean ok = this.adoptante.equals(adopcion.adoptante) && this.animal.equals(adopcion.animal);
                return ok;
            }
        }
    
        public int hashCode() {
            return this.adoptante.hashCode() + this.animal.hashCode();
        }
    
        public String toString() {
            return String.format("Adopcion: %tY-%tB-%td, %s, %s", this.fecha, this.fecha, this.fecha, this.animal, this.adoptante);
        }
    }
    
\end{lstlisting}



\subsubsection{Clase Animal}\label{codigo:animal}
La clase \texttt{Animal} modela a un animal registrado en el sistema. Cada animal tiene un 
ID único, una fecha de nacimiento, un estado actual y está asociado a un \texttt{Refugio}.\par

\begin{lstlisting}[style = javaNormal, language=Java] 
    package sistema;
    import java.util.Date;
    
    public class Animal {
        private int ID;
        private Date nacimiento;
        private EstadoAnimal estadoAnimal; 
        private Adopcion adopcion;

        public Animal(int ID, Date nacimiento, EstadoAnimal estadoAnimal) {
            assert ID > 0 : "El ID del animal debe ser valido.";
            assert nacimiento != null : "La fecha de nacimiento no puede ser nula.";
            assert estadoAnimal != null : "El estado del animal debe estar definido.";

            this.ID = ID;
            this.nacimiento = nacimiento;
            this.estadoAnimal = estadoAnimal;
        }
    
        public EstadoAnimal getEstadoAnimal() {
            return estadoAnimal;
        }
        public void setEstadoAnimal(EstadoAnimal estadoAnimal) {
            assert estadoAnimal != null : "El estado del animal debe estar definido.";
            this.estadoAnimal = estadoAnimal;
        }
        public Date getNacimiento() {
            return nacimiento;
        }
        public void setNacimiento(Date nacimiento) {
            assert nacimiento != null : "La fecha de nacimiento no puede ser nula";
            this.nacimiento = nacimiento;
        }
        public Adopcion getAdopcion() {
            return this.adopcion;
        }
        public void setAdopcion(Adopcion adopcion){
            assert  adopcion != null;
            this.adopcion = adopcion;
        }
        public int getID() {
            return ID;
        }

        @Override
        public boolean equals(Object obj) {
            if( this == obj ) return true;
            if(obj instanceof Animal ){
                Animal animal = (Animal) obj;
                return this.ID == animal.ID;
            }
            return false;
        }
        @Override
        public int hashCode() {
            return Integer.hashCode(ID);
        }
        @Override
        public String toString() {
            return String.format("Animal: ID=%d, nacimiento=%tF, estado=%s", ID, nacimiento, estadoAnimal);
        }
    }
\end{lstlisting}



\subsection{Conclusión}

El diseño e implementación del código de andamiaje para el sistema se realizó siguiendo 
los principios fundamentales del diseño orientado a objetos, adaptados a los requerimientos 
específicos de este apartado. Se tomaron la decisiones de diseño adecuadas, como la gestión 
de asociaciones entre clases, la encapsulación de datos y la validación restricciones con \texttt{assert}, 
proporcionando un modelo consistente y flexible.\par
\vspace{0.15cm}
Una de las decisiones clave fue el uso combinado de asociaciones directas para relaciones 
simples y la reificación de asociaciones para relaciones más complejas junto con \texttt{get}
con conjuntos inmutables. Esto permitió mantener un equilibrio entre la simplicidad de las 
implementaciones directas, como la gestión de animales en el refugio, y la flexibilidad 
de las relaciones complejas, como las adopciones, donde se requieren atributos adicionales 
y validaciones específicas mientras protegíamos las listas de cada objeto en el sistema.\par
\vspace{0.15cm}
Además, la bidireccionalidad en relaciones como las adopciones, garantizó la 
consistencia del modelo al sincronizar automáticamente los datos entre entidades 
relacionadas.\par
TODO: AÑADIR COMO RESUMEN LAS DESICIONES QUE NO SE HAYANN  MENCIONADO DE LO QUE DIEGO A RECOPILADO EN LLAMADA

\newpage

\section{Apartado A}
\subsection{Diseño del Código de Andamiaje}
\subsection*{Introducción}
El concepto de \enquote{código de andamiaje} en el diseño orientado a objetos, 
particularmente en Java, hace referencia al conjunto de estructuras y métodos 
necesarios para implementar asociaciones entre clases, asegurando la 
consistencia y la integridad del sistema.\par
\vspace{0.15cm}
Su propósito es proporcionar un marco inicial sobre el que los desarrolladores pueden 
construir las funcionalidades particulares de un proyecto. Sin embargo, la línea entre
\enquote{andamiaje} e \enquote{implementación completa} puede ser fina ya que estamos 
añadiendo nuevas funcionalidades que luego se convierten en el nuevo marco inicial para 
futuros cambios que vayamos a realizar en los próximos apartados. A continuación se 
exponen las decisiones diseño que completan el andamiaje inicial.

\subsection{Análisis de opciones de Diseño}
En esta sección se exponen las diferentes opciones de diseño para la implementación 
del sistema de gestión de un refugio de animales, conforme al modelo de clases y 
operaciones proporcionado.\par
\vspace{0.15cm}
El sistema requiere gestionar los socios del refugio (quienes pueden desempeñar 
diferentes roles como voluntarios, donantes y adoptantes), así como el registro, 
adopción y donación de animales. Además, se deben considerar las relaciones entre 
las entidades (socios, animales, refugio, donaciones) y las restricciones del sistema, 
como la consistencia en los datos y las operaciones.\par
\vspace{0.15cm}
Importante mencionar que en nuestro diseño, \textbf{no hemos aplicado un único enfoque de manera 
exclusiva}. Hemos adoptado una combinación de estrategias dependiendo de las necesidades de 
cada relación dentro del sistema justificándolo de forma adecuada.

\subsubsection{Manejo de las Asociaciones}

\begin{description}
    \item[a)] \textbf{Asociación Directa (Sin Reificación\footnote{
        La reificación es una técnica en programación 
        orientada a objetos que se basa en convertir un 
        concepto abstracto, como una relación, en una 
        entidad concreta o clase.})}
\end{description}

\textit{\textbf{Descripción:}}  
En este enfoque, las asociaciones entre clases se implementan directamente como atributos 
en las clases relacionadas.\par
\vspace{0.15cm}
Esta práctica es fácil de implementar porque la cantidad de clases a gestionar 
y el número de clases necesarias para representar las relaciones es menor. No obstante, 
añadir atributos adicionales a las asociaciones (como fechas en el proceso 
de adopción) puede traer problemas de consistencia al manejar relaciones complejas como la 
de un socio con múltiples roles (lo exploraremos en futuras secciones).\par
\newpage % Para que no se ve aglomerada la definición en la footnote
\textbf{Ejemplo: Implementación de \texttt{Refugio} con asociación directa a \texttt{Animal}}\par
La asociación es directa porque \texttt{Refugio} gestiona los \texttt{Animales} mediante un 
\texttt{set}, sin una clase intermedia que relacione ambas entidades.(ver Código~\ref{codigo:refugio})\par
TODO: CAMBIAR EL CODE\par
En nuestra implementación, hemos decidido la utilización de un \texttt{Set} para las estructuras de datos 
en vez de \texttt{List} por varios factores:

\begin{itemize}
    \item \textbf{Control de elementos repetidos:} Los \texttt{Set} usan los métodos de \texttt{hashCode} e \texttt{equals}
    para hacer la comprobación de la existencia de elementos en la colección. Si un nuevo elemento coincide con uno existente, 
    no se inserta, evitando comprobaciones adicionales que haríamos con el uso de \texttt{List}.
    \item \textbf{Complejidad algorítmica:} En los \texttt{HashSet}, la búsqueda y la inserción tienen una complejidad de \(O(1)\), 
    ya que se basan en tablas hash. Por otro lado, en una \texttt{List} (como \texttt{ArrayList}), las operaciones de búsqueda tienen una 
    complejidad de \(O(n)\), porque requiere iterar sobre los elementos.
    \item \textbf{Orden de los elementos:} En nuestra implementación, no es necesario mantener un orden específico en las colecciones. 
    Por esta razón, el uso \texttt{Set} es más adecuado que \texttt{List} teniendo en cuenta los puntos anteriores.
\end{itemize}
\vspace{0.45cm}

\begin{description}
    \item[b)] \textbf{Reificación de la Asociación (Clase de Asociación)}
\end{description}

\textit{\textbf{Descripción:}}  
En este enfoque, las asociaciones complejas entre clases se modelan mediante clases 
intermedias. Por ejemplo, la clase \texttt{Adopcion} representa la relación entre un 
\texttt{Animal}, un \texttt{Adoptante}, y un \texttt{Voluntario}, incluyendo atributos 
como \texttt{fechaAdopcion} para capturar detalles específicos de la relación.\par
\vspace{0.15cm}
Como la reificación nos permiten agregar atributos y métodos específicos a las relaciones,
facilita la implementación de restricciones complejas relacionadas con la asociación.
Sin embargo, aumenta el número de clases y relaciones a gestionar, lo que hace el diseño más denso
ya que debemos implementar y gestionar las clases de asociación, así como los métodos para 
acceder a las relaciones.\par
\vspace{0.15cm}

\textbf{Nuestra Implementación: Uso de \texttt{Adopcion} como clase de asociación:}
\begin{itemize}
    \item Se representa la relación entre \texttt{Animal}, \texttt{Adoptante} y 
    \texttt{Voluntario} mediante una clase intermedia. (ver Código~\ref{codigo:adopcion})
    \item Atributos como \texttt{fecha} añaden flexibilidad al modelo, permitiendo 
    capturar detalles adicionales de la relación.
    \item Se gestionan las relaciones bidireccionales entre las entidades involucradas, 
    asegurando consistencia en los datos.
\end{itemize}

Elegimos este enfoque, ya que proporciona la flexibilidad necesaria para agregar atributos 
y gestionar reglas de negocio específicas. La asociación directa fue descartada porque no 
permitiría capturar detalles adicionales, como la fecha de adopción, ni manejar eficientemente 
las restricciones relacionadas con el proceso de adopción.



\subsubsection{Manejo de Roles de los Socios}

\begin{description}
    \item[a)] \textbf{Subclases Específicas para cada Rol}
\end{description}

\textit{\textbf{Descripción:}}  
Cada rol (\texttt{Voluntario}, \texttt{Donante}, \texttt{Adoptante}) se implementa como una 
subclase de la clase \texttt{Socio}. Esto permite encapsular los atributos y métodos 
específicos de cada rol dentro de su respectiva subclase.
\vspace{0.15cm}


Esto proporciona claridad al diseño, ya que cada rol está claramente representado con 
métodos específicos para su comportamiento. Además, permite encapsular los atributos y 
métodos particulares de cada tipo de socio, lo que mejora la organización y legibilidad 
del código. Aunque tiene una limitación significativa para manejar roles múltiples, ya que no permite que 
un socio asuma más de un rol sin duplicar instancias de las subclases. Esto hace que el 
diseño sea rígido y menos flexible en casos donde los roles pueden cambiar dinámicamente 
o coexistir (volveremos a hablar de esto en los siguientes apartados).\par
\vspace{0.15cm}



\textbf{Ejemplo: Subclases específicas para los roles}\par
El diseño tiene implementadas las subclases \texttt{Donante}, \texttt{Adoptante}, 
y \texttt{Voluntario} como extensiones de la clase \texttt{Socio}.\par
(ver Código~\ref{codigo:donante}, \ref{codigo:adoptante}, \ref{codigo:voluntario})

\begin{description}
    \item[b)] \textbf{Uso de Composición de Roles}
\end{description}

\textit{\textbf{Descripción:}}  
En lugar de modelar cada rol como una subclase de \texttt{Socio}, este enfoque utiliza 
la composición para permitir que un socio tenga múltiples roles simultáneamente. 
Cada rol se modela como una clase independiente que puede ser asociada dinámicamente a 
un \texttt{Socio} mediante una colección de roles.
\vspace{0.15cm}


Este enfoque es mucho más flexible, ya que permite asignar múltiples roles a un socio 
sin necesidad de crear combinaciones de subclases. También simplifica el manejo de 
roles dinámicos y permite cambios en tiempo de ejecución.
Puede reducir la claridad del diseño, ya que no existe una distinción explícita entre 
los diferentes tipos de socios. Además, requiere implementar lógica adicional para 
validar qué operaciones son aplicables para los roles asignados a cada socio.\par
\vspace{0.15cm}


\textbf{Nuestra implementación: Uso de Subclases Específicas para cada Rol:}\par  
\vspace{0.15cm}
Hemos decidido no cambiar como están implementados los roles mediante subclases específicas 
(para este apartado) en lugar de composición. Esto se debe a que en nuestro modelo actual, 
los roles están claramente definidos y no se requiere que un socio tenga múltiples roles de 
manera simultánea. Además:
\begin{itemize}
    \item La claridad y encapsulación que proporciona la herencia permiten manejar las responsabilidades y comportamientos específicos de cada tipo de socio de manera aislada.
    \item Aunque la composición sería más flexible, introducirá complejidad adicional innecesaria para los requisitos actuales del sistema.
\end{itemize}

No obstante, los requisitos del sistema cambiarán en futuros apartados pidiendo que un socio
tenga múltiples roles simultáneamente. En el apartado correspondiente, se discute porque 
la composición sería una solución más adecuada y como se ha implementado.




\subsubsection{Consistencia y Gestión de Datos}

\begin{description}
    \item[a)] \textbf{Encapsulación Estricta}
\end{description}

\textit{\textbf{Descripción:}}  
Este enfoque restringe el acceso directo a los atributos y métodos de las clases mediante 
el uso de visibilidad privada. Para interactuar con los atributos, se proporcionan métodos 
públicos controlados (\texttt{getters} y \texttt{setters}) que incluyen validaciones (mediante 
\texttt{asserts}) para garantizar que los datos se mantengan en un estado consistente.
\vspace{0.15cm}

    Debido a esto nos aseguramos que los datos sean modificados de manera controlada y consistente.
    Facilita la incorporación de validaciones o pruebas unitarias lo que completa el comportamiento del
    esperado del sistema.Dicho esto también se requiere de implementar más métodos, como \texttt{getters}, \texttt{setters} y validaciones 
    necesarias, lo que aumenta la cantidad de código. Además, estas validaciones podrían hacer 
    que el diseño sea más extenso y menos directo.\par
    \vspace{0.15cm}


\textbf{Ejemplo en el sistema: Uso de encapsulación estricta en la clase \texttt{Animal}:}\par  
En nuestro sistema, la clase \texttt{Animal} utiliza atributos privados y métodos públicos 
controlados para garantizar consistencia y validaciones en tiempo de ejecución.Este enfoque 
asegura que cualquier intento de modificar el estado de un \texttt{Animal} pase por 
validaciones definidas en los métodos públicos. (ver Código~\ref{codigo:animal})


\begin{description}
    \item[b)] \textbf{Uso de Colecciones Inmutables}
\end{description}

\textit{\textbf{Descripción:}}  
En este enfoque, las colecciones utilizadas para representar relaciones (por ejemplo, 
listas o conjuntos de \texttt{Animal} en \texttt{Refugio}) son inmutables. Esto garantiza 
que las relaciones no puedan ser modificadas accidentalmente fuera de las clases que las 
gestionan.
\vspace{0.15cm}

    Mejora la integridad del sistema al garantizar que las relaciones no se modifiquen de 
    manera no controlada.
    Por otro lado introduce rigidez ya que no permite realizar cambios dinámicos en las relaciones sin 
    reemplazar completamente la colección. Esto puede dificultar la gestión de operaciones 
    como agregar o eliminar elementos.\par
    \vspace{0.15cm}

\textbf{Ejemplo de Uso de Colecciones Inmutables en la Clase \texttt{Refugio}:}\par  
En nuestro sistema, el método \texttt{getAnimalesRegistrados} de la clase \texttt{Refugio} 
devuelve una vista inmutable de los animales registrados. Esto asegura que las listas no 
puedan modificarse desde fuera de la clase. Además, el uso de \texttt{Collections.}\texttt{enumeration} 
garantiza que la colección de animales no pueda ser alterada fuera de la clase \texttt{Refugio}, 
manteniendo la consistencia de los datos. (ver Código~\ref{codigo:refugio} aunque se implementa en varias clases)

\textbf{Decisión Tomada: Encapsulación Controlada con Enumerations:}\par
En nuestro diseño, optamos por una encapsulación controlada en lugar de colecciones 
completamente inmutables. Esto se debe a que:
\begin{itemize}
    \item Proporciona flexibilidad para realizar cambios dinámicos en las colecciones a 
    través de métodos controlados, lo que es necesario para operaciones como agregar o 
    eliminar animales en un refugio.
    \item Utilizar enumeraciones en los métodos \texttt{get} garantiza que las colecciones 
    no se modifiquen desde fuera de las clases, preservando la integridad de los datos.
\end{itemize}

Este enfoque combina lo mejor de ambos mundos: flexibilidad para realizar cambios controlados 
y protección contra modificaciones accidentales.



\subsubsection{Estrategias para Manejo de Adopciones y Donaciones}

\begin{description}
    \item[a)] \textbf{Operaciones Independientes}
\end{description}

\textit{\textbf{Descripción:}}  
En este enfoque, cada operación (como adopción, registro de animales o donaciones) se 
implementa de forma independiente, sin interacciones entre ellas. Cada acción tiene su 
propio método o flujo lógico separado.
\vspace{0.15cm}

    Este diseño asegura que las operaciones están bien definidas y separadas, lo que 
    facilita su comprensión. Además, la simplicidad del diseño permite que sea directo y 
    fácil de implementar.Sin Embargo, puede llevar a la duplicación de código si varias operaciones comparten lógica similar 
    (por ejemplo, validar la existencia de un animal o donante). También puede ser menos 
    flexible, ya que cualquier cambio en una operación podría requerir modificaciones en 
    múltiples partes del sistema.\par
    \vspace{0.15cm}

\textbf{Ejemplo: Operaciones independientes en nuestra implementación}  
En nuestro diseño, las adopciones y donaciones se gestionan mediante clases específicas 
(\texttt{Adopcion} y \texttt{Donacion}), cada una con su propia lógica y atributos.

\begin{description}
    \item[b)] \textbf{Reutilización de Lógica Compartida entre Operaciones}
\end{description}

\textit{\textbf{Descripción:}}  
En lugar de mantener las operaciones completamente separadas, este enfoque identifica y 
reutiliza lógica común entre las operaciones (como validaciones o actualizaciones de estado). 
Aunque no implementamos este enfoque en nuestra solución actual, sería posible centralizar las 
validaciones comunes mediante una clase auxiliar, como se muestra en el siguiente ejemplo.\par
\vspace{0.15cm}
Esta opción reduce la duplicación de código, ya que la lógica compartida se implementa una sola vez
y facilita la incorporación de nuevas funcionalidades relacionadas con las operaciones existentes. Pero 
puede introducir una dependencia más estrecha entre las clases, lo que podría aumentar 
la complejidad del sistema en caso de cambios importantes.

\textbf{Ejemplo Propuesto: Centralización de Validaciones}  
Aunque nuestra implementación actual gestiona las validaciones directamente en los métodos 
de las clases relevantes (\texttt{Adoptante}, \texttt{Donante}). Por ejemplo, podríamos considerar una 
clase auxiliar para centralizarlas en el futuro:

\begin{lstlisting}[style = javaEspecifico, language=Java, caption={Clase Auxiliar para Validaciones}] 
public class Validacion {
    public static void validarEstadoAnimal(Animal animal, EstadoAnimal estadoEsperado) {
        assert animal.getEstadoAnimal() == estadoEsperado : 
            "El estado del animal no coincide con el esperado.";
    }
}
\end{lstlisting}

En nuestra implementación actual, la validación se realiza directamente dentro de las clases:

\begin{lstlisting}[style = javaEspecifico, language=Java, caption={Manejo de validaciones dentro del método Adoptar}] 
public void adoptar(Animal a, Voluntario v) {
    assert a.getEstadoAnimal() == EstadoAnimal.DISPONIBLE : 
        "El animal no esta disponible.";
    Adopcion adopcion = new Adopcion(a, this, v, new Date());
    adopciones.add(adopcion);
}
\end{lstlisting}
\vspace{0.15cm}
\textbf{Decisión Tomada: Mantener las Validaciones en las Clases Relevantes}\par
En nuestra implementación actual, las validaciones se realizan directamente en las clases 
donde ocurren las operaciones. Esto se alinea con la claridad y simplicidad requeridas por 
el sistema. Sin embargo, reconocemos que la centralización de lógica compartida podría ser 
útil en sistemas más complejos. FIXME: EL GETTER ES MAL EJEMPLO DE ESTO (ver los getters en Código~\ref{codigo:adopcion} por ejemplo) 



\subsubsection{Representación de Relaciones en el Sistema}

\begin{description}
    \item[a)] \textbf{Relaciones Unidireccionales}
\end{description}

\textit{\textbf{Descripción:}}  
En una relación unidireccional, solo una entidad tiene conocimiento de la relación. 
Por ejemplo, un \texttt{Adoptante} puede conocer al \texttt{Animal} que adopta, 
pero el \texttt{Animal} no necesita saber nada sobre el \texttt{Adoptante}.\par
\vspace{0.15cm} 
Estas relaciones gestionan la relación con una clase, lo que reduce la complejidad del sistema.
El problema, es que limita las consultas entre clases relacionadas y puede volverse más complejo 
añadir funcionalidades.\par
\vspace{0.15cm}
En nuestro sistema, todas las relaciones unidireccionales con 1 a muchos, por ejemplo:
TODO: EXPANDIR ESTA IDEA

\begin{description}
    \item[b)] \textbf{Relaciones Bidireccionales}
\end{description}

\textit{\textbf{Descripción:}}  
En una relación bidireccional, ambas entidades conocen y mantienen referencias mutuas. 
Por ejemplo, cuando un \texttt{Adoptante} adopta un \texttt{Animal}, ambos se actualizan 
mutuamente para reflejar la relación.\par
\vspace{0.15cm}
Las relaciones de este estilo garantizan la consistencia de los datos, ya que ambas partes relacionadas están 
sincronizadas al mantener referencias mutuas explícitas. Sin embargo, como hay que tener una sincronización constante 
entre las todas las clases relacionadas, si tuviéramos muchas relaciones bidireccionales puede dificultar el mantenimiento
por que genera un alto nivel de acoplamiento.\par
\vspace{0.15cm}
\textbf{Ejemplo en el Sistema: Relaciones Bidireccionales en \texttt{Adopcion}}\par 
En nuestro diseño, la relación entre \texttt{Animal}, \texttt{Adoptante}, y 
\texttt{Voluntario} es bidireccional y se asegurando consistencia en ambas direcciones
reflejando los cambios realizados en una clase en las demás involucradas.
(ver Código~\ref{codigo:adopcion} como, por ejemplo, se actualiza el estado del animal tras ser adoptado)\par
\vspace{0.15cm}
\textbf{Decisión Tomada: Relaciones Bidireccionales}\par
Hemos implementado relaciones bidireccionales para las asociaciones complejas del sistema, 
como las adopciones, ya que garantizan consistencia y sincronización entre las entidades 
relacionadas. Sin embargo, para relaciones más simples, como la lista de animales en un 
refugio, usamos relaciones unidireccionales para mantener la simplicidad.

\subsection{Diagrama de Diseño}

\begin{figure}[H]
    \centering
     \includegraphics[width=1\linewidth]{assets/umaLogo.png}
     \caption{UMA}
\end{figure}
FALTA HACER Y METER EL DIAGRAMA DE DISEÑO DE NUESTRO SISTEMA.RECOMENDABLE PONER UNA MINI EXPLICACIÓN.

\newpage



\subsection{Implementación del Modelo}
\subsubsection{Clase Socio}\label{codigo:socio}
La clase \texttt{Socio} es abstracta y representa la base para las distintas subclases: 
\texttt{Adoptante}, \texttt{Voluntario}, y \texttt{Donante}. Esta clase asegura que 
cada socio tenga un \texttt{ID} único, una fecha de registro válida y un refugio asociado.

\begin{lstlisting}[style = javaNormal, language=Java] 
    package sistema;

    import java.util.Collections;
    import java.util.Date;
    
    public abstract class Socio {
        private int ID;
        private Date registro;
        private final Refugio refugioAsociado;
    
        public Socio(int ID, Date fechaRegistro, Refugio refugioAsociado) {
            assert ID > 0 : "El ID del socio debe ser valido.";
    
            assert fechaRegistro != null : "La fecha de registro no puede ser nula.";
    
            assert refugioAsociado != null : "El refugio asociado no puede ser nulo.";
    
            this.ID = ID;
            this.registro = fechaRegistro;
            this.refugioAsociado = refugioAsociado;
            refugioAsociado.addSocio(this);
    
            assert refugioAsociado.getSocios().hasMoreElements() && Collections.list(refugioAsociado.getSocios()).contains(this);
        }
    
        public int getID() {
            return this.ID;
        }
    
        private void setID(int ID) {
            assert ID > 0;
    
            assert Collections.list(this.getRefugio().getSocios()).stream().noneMatch((s) -> {
                return s.getID() == ID;
            }) : "El ID ya existe en el refugio";
    
            this.ID = ID;
        }
    
        public Date getRegistro() {
            return this.registro;
        }
    
        public void setRegistro(Date fechaRegistro) {
            assert fechaRegistro != null;
    
            this.registro = fechaRegistro;
        }
    
        public Refugio getRefugio() {
            return this.refugioAsociado;
        }
    
        public boolean equals(Object obj) {
            if (this == obj) {
                return true;
            } else if (obj instanceof Socio) {
                Socio socio = (Socio)obj;
                return this.ID == socio.ID;
            } else {
                return false;
            }
        }
    
        public int hashCode() {
            return Integer.hashCode(this.ID);
        }
    }
\end{lstlisting}



\subsubsection{Clase Donante}\label{codigo:donante}
La clase \texttt{Donante} extiende de \texttt{Socio} y gestiona las donaciones realizadas 
por un socio. Las donaciones se almacenan en un \texttt{HashSet} para asegurar que no hayan elementos repetidos y por eficiencia. (cómo se explicó en \ref{page:Consideraciones}) 
Por otro lado, hemos decidido implementar al crear un nuevo objeto Donante en el sistema,
el constructor llamará directamente al método donar ya que es una condición necesaria para ser donante. Las donaciones que sean modificadas son tratadas desde la clase Donación y
el \texttt{HashSet donaciones} se actualizará.


\begin{lstlisting}[style = javaNormal, language=Java] 
    package sistema;

    import java.time.LocalDate;
    import java.time.ZoneId;
    import java.util.*;
    
    public class Donante extends Socio{
        private Set<Donacion> donaciones;
        public Donante(int ID, Date date,Refugio r, float cantidad) {
            super(ID,date,r);
            assert cantidad > 0 : "La cantidad inicial donada debe ser mayor a cero.";
            donaciones = new HashSet<>();
            donar(cantidad);
        }

        public void donar(float cantidad){
            assert cantidad > 0 : "La cantidad donada debe ser mayor a cero.";
            assert Collections.list(this.getRefugio().getSocios()).contains(this): "El socio debe ser donante antes de poder donar";
            Donacion d = new Donacion(cantidad, Date.from(fechaDonacion.atStartOfDay(ZoneId.systemDefault()).toInstant()));
            addDonacion(d);
            Refugio r = super.getRefugio();
            r.setLiquidez(r.getLiquidez() + cantidad);
            assert donaciones.contains(d);
        }
    
        protected void addDonacion(Donacion donacion){
            assert donacion != null: "La donacion no puede ser nula";
            donaciones.add(donacion);
        }
    
        protected void removeDonacion(Donacion donacion){
            assert donacion != null : "La donacion no puede ser nula.";
            if (donaciones.contains(donacion) && donaciones.size() > 1) {
                donaciones.remove(donacion);
            } else if (donaciones.contains(donacion) && donaciones.size() == 1) {
                System.out.println("Todo donante debe tener al menos una donacion, estas intentando eliminar la unica donacion asociada a este socio donante");
            } else {
                System.out.println("Este socio no ha realizado la donacion que intentas eliminar");
            }
        }
        public Enumeration<Donacion> getDonaciones(){
            return Collections.enumeration(this.donaciones);
        }
    
        @Override
        public String toString() {
            return "Donante " + super.getID();
        }
    }
    
\end{lstlisting}

\subsubsection{Clase Adoptante}\label{codigo:adoptante}
La clase \texttt{Adoptante} extiende de \texttt{Socio} y simula las adopciones realizadas 
por un adoptante. Las adopciones se almacenan en un \texttt{HashSet} para evitar adopciones duplicadas.
\begin{lstlisting}[style = javaNormal, language=Java] 
    package sistema;

    import java.util.*;
    
    public class Adoptante extends Socio {
        private Set<Adopcion> adopciones;
    
        public Adoptante(int ID, Date date, Refugio r) {
            super(ID, date,r);
            adopciones = new HashSet<>();
        }
        public void adoptar(Animal a, Voluntario v) {
            assert a.getEstadoAnimal() == EstadoAnimal.DISPONIBLE: "El animal ya ha sido adoptado";
            Refugio refugioDelVoluntario = v.getRefugio();
            a.setEstadoAnimal(EstadoAnimal.ADOPTADO);
            refugioDelVoluntario.removeAnimalesRefugiados(a);
            v.tramitarAdopcion(a, this);
        }
    
        protected void addAdopcion(Adopcion a){
            this.adopciones.add(a);
        }
    
        protected void removeAdopcion(Adopcion a){
            if (adopciones.contains(a)) adopciones.remove(a);
            else System.out.println("Este animal ya no esta asociado al adoptante");
        }
        public Enumeration<Adopcion> getAdopciones(){
            return Collections.enumeration(adopciones);
        }
        @Override
        public boolean equals(Object obj) {
            if (this == obj) return true;
            if(obj instanceof Adoptante ){
                Adoptante adoptante = (Adoptante) obj;
                return adoptante.getID() == this.getID();
            }
            return false;
        }
        @Override
        public int hashCode() {
            return Integer.hashCode(this.getID());
        }
    
        @Override
        public String toString() {
            return "Adoptante " + super.getID();
        } 
    }    
\end{lstlisting}

\subsubsection{Clase Voluntario}\label{codigo:voluntario}
La clase \texttt{Voluntario} extiende de \texttt{Socio} y gestiona los trámites de 
adopción realizados por un voluntario. Los voluntarios se almacenan en un \texttt{HashSet},
evitando así voluntarios duplicados. 
\begin{lstlisting}[style = javaNormal, language=Java] 
    package sistema;

    import java.time.LocalDate;
    import java.time.ZoneId;
    import java.util.*;
    
    public class Voluntario extends Socio{
        Set<Adopcion> tramites;
        
        public Voluntario(int ID, Date date,Refugio r) {
            super(ID, date,r);
            tramites = new HashSet<>();
        }
        public void tramitarAdopcion(Animal a, Adoptante ad){
            assert ad != null : "El adoptante no puede ser nulo.";
            LocalDate fechaAdopcion = LocalDate.now();
            Adopcion adopcion = new Adopcion(a, ad, this, Date.from(fechaAdopcion.atStartOfDay(ZoneId.systemDefault()).toInstant()));
            addTramite(adopcion);
            ad.addAdopcion(adopcion);
        }
        public void registrar(Animal a){
            Refugio r = super.getRefugio();
            assert r != null : "El refugio asociado no puede ser nulo.";
            assert a != null : "El animal no puede ser nulo.";
            a.setEstadoAnimal(EstadoAnimal.DISPONIBLE);
            r.addAnimalesRefugiados(a);
    
        }
        public Enumeration<Adopcion> getTramites(){
            return Collections.enumeration(tramites);
        }
        protected void addTramite(Adopcion ad){
            assert ad != null : "El tramite de adopcion no puede ser nulo.";
            tramites.add(ad);
    
        }
        protected void removeTramite(Adopcion ad){
            assert  ad != null: "El tramite de adopcion no puede ser nulo.";
            tramites.remove(ad);
        }
    
    
        @Override
        public String toString() {
            return "Voluntario " + super.getID();
        }
    }
    
\end{lstlisting}

\subsubsection{Clase Refugio}\label{codigo:refugio}
La clase \texttt{Refugio} gestiona el conjunto de \texttt{Socios} y \texttt{Animales}.\par
\textbf{liquidez} está declarado como un \texttt{float} por las razones que se exponen en \ref{page:Consideraciones}.


\begin{lstlisting}[style = javaNormal, language=Java] 
package sistema;
import java.util.*;

public class Refugio {
    private float liquidez;
    private Set<Animal> animalesRegistrados;
    private Set<Animal> animalesRefugiados;
    private Set<Socio> socios;

    public Refugio(float liquidez) {
        assert liquidez >= 0 : "La liquidez debe ser no negativa.";
        this.liquidez = liquidez;
        animalesRefugiados = new HashSet<>();
        animalesRegistrados = new HashSet<>();
        socios = new HashSet<>();
    }

    public float getLiquidez() 
        return liquidez;
    }
    public void setLiquidez(float liquidez) {
        assert liquidez >= 0 : "La liquidez debe ser no negativa";
        this.liquidez = liquidez;
    }
    protected void addSocio(Socio s) {
        assert s != null : "El socio no puede ser nulo.";
        if(socios.contains(s)) {
            System.out.println("El socio ya esta registrado.");
            return;
        }
        socios.add(s);
    }
    protected void removeSocio(Socio s) {
        assert s != null : "El socio no puede ser nulo.";
        if (socios.contains(s)) {
            socios.remove(s);
        } else {
            System.out.println("Este socio no esta registrado en el refugio.");
        }
    }

    public Enumeration<Animal> getAnimalesRegistrados() {
        return Collections.enumeration(animalesRegistrados);
    }
    public Enumeration<Animal> getAnimalesRefugiados() {
        return Collections.enumeration(animalesRefugiados);
    }
    public Enumeration<Socio> getSocios() {
        return Collections.enumeration(socios);
    }

    public List<Adoptante> getAdoptantes() {
        List<Adoptante> adoptantes = new ArrayList<>();
        for (Socio s : socios) {
            if (s instanceof Adoptante) {
                adoptantes.add((Adoptante) s);
            }
        }
        return adoptantes;
    }
    public List<Voluntario> getVoluntarios() {
        List<Voluntario> voluntarios = new ArrayList<>();
        for (Socio s : socios) {
            if (s instanceof Voluntario) {
                voluntarios.add((Voluntario) s);
            }
        }
        return voluntarios;
    }
    public List<Donante> getDonantes() {
        List<Donante> donantes = new ArrayList<>();
        for (Socio s : socios) {
            if (s instanceof Donante) {
                donantes.add((Donante) s);
            }
        }
        return donantes;
    }
    public void registrar(Animal a){
        this.addAnimalesRegistrados(a);
    }
    protected void addAnimalesRefugiados(Animal a){
        assert a != null : "El animal no puede ser nulo.";
        if(!animalesRefugiados.contains(a)){
            animalesRefugiados.add(a);
            this.addAnimalesRegistrados(a);
        } else System.out.println("Este animal ya esta en el refugio.");
    }
    private void addAnimalesRegistrados(Animal a){
        assert a != null : "El animal no puede ser nulo.";
        if (!animalesRegistrados.contains(a)) {
            animalesRegistrados.add(a);
        } else {
            System.out.println("El animal ya esta registrado.");
        }
    }

    protected void removeAnimalesRefugiados(Animal a){
        assert a != null : "El animal no puede ser nulo.";
        if (animalesRefugiados.contains(a)) {
            animalesRefugiados.remove(a);
        } else {
            System.out.println("El animal no se encuentra en este Refugio.");
        }
    }
    protected void removeAnimalesRegistrados(Animal a){
        assert a != null : "El animal no puede ser nulo.";
        if (animalesRegistrados.contains(a) && animalesRegistrados.size() > 1) {
            animalesRegistrados.remove(a);
        } else if (animalesRegistrados.contains(a) && animalesRegistrados.size() == 1) {
            System.out.println("Todo refugio debe tener al menos un animal registrado, estas intentando eliminar el unico animal existente.");
        } else {
            System.out.println("El animal no se encuentra en este Refugio.");
        }
    }

    public void mostrarAnimalesRefugiados(){
        System.out.println(animalesRefugiados.toString());
    }
    public void mostrarAnimalesRegistrados(){
        System.out.println(animalesRegistrados.toString());
    }
    public void mostrarSocios() {
        for (Socio s : socios) {
            System.out.println(s);
        }
    }
    public void mostrarSociosPorTipo() {
        System.out.println("Adoptantes: " + getAdoptantes());
        System.out.println("Voluntarios: " + getVoluntarios());
        System.out.println("Donantes: " + getDonantes());
    }

    @Override
    public String toString() {
        StringBuilder sb = new StringBuilder();
        sb.append("Animales Registrados: ").append(animalesRegistrados).append("\n");
        sb.append("Animales Refugiados: ").append(animalesRefugiados).append("\n");
        sb.append("Socios: ").append(socios).append("\n");
        sb.append("Liquidez: ").append(liquidez);
        return sb.toString();
    }
}
\end{lstlisting}



\subsubsection{Clase Donacion}\label{codigo:donacion}
La clase \texttt{Donacion} representa una donación realizada por un \texttt{Donante}. 
Incluye la cantidad que como anteriormente mencionamos en \ref{page:Consideraciones} por temas de eficiencia es un \texttt{float},
la fecha de la donación y el donante asociado. Las validaciones 
aseguran que los valores sean válidos en el momento de la creación de la instancia.

\begin{lstlisting}[style = javaNormal, language=Java] 
    package sistema;

    import java.util.Date;
    import java.util.Objects;
    
    public class Donacion {
        private float cantidad;
        private Date date;
    
        public Donacion(float cantidad, Date date) {
            assert cantidad > 0.0F : "La cantidad debe ser positiva.";
    
            assert date != null && !date.after(new Date()) : "La fecha no puede ser nula ni estar en el futuro.";
    
            this.cantidad = cantidad;
            this.date = date;
        }
    
        public float getCantidad() {
            assert this.cantidad > 0.0F : "La cantidad no puede ser nula.";
    
            return this.cantidad;
        }
    
        public void setCantidad(float cantidad) {
            this.cantidad = cantidad;
        }
    
        public Date getDate() {
            assert this.date != null : "La fecha no puede ser nula.";
    
            return this.date;
        }
    
        public void setDate(Date date) {
            this.date = date;
        }
    
        public String toString() {
            return String.format("Donacion: %.2f, %tY-%tB-%td", this.cantidad, this.date, this.date, this.date);
        }
    
        public boolean equals(Object o) {
            if (this == o) {
                return true;
            } else if (o != null && this.getClass() == o.getClass()) {
                Donacion donacion = (Donacion)o;
                return Float.compare(this.cantidad, donacion.cantidad) == 0 && Objects.equals(this.date, donacion.date);
            } else {
                return false;
            }
        }
    
        public int hashCode() {
            return Objects.hash(new Object[]{this.cantidad, this.date});
        }
    }
\end{lstlisting}



\subsubsection{Clase Adopcion}\label{codigo:adopcion}
La clase \texttt{Adopcion} modela una adopción de un \texttt{Animal} realizada por un 
\texttt{Adoptante}, gestionada por un \texttt{Voluntario}. Implementa la bidireccionalidad 
entre estas entidades para mantener consistencia en las asociaciones.

\begin{lstlisting}[style = javaNormal, language=Java] 
    package sistema;

    import java.util.Date;
    
    public class Adopcion {
        private Date fecha;
        private final Animal animal;
        private final Adoptante adoptante;
        private final Voluntario voluntario;
    
        public Adopcion(Animal a, Adoptante ad, Voluntario v, Date fecha) {
            assert a != null : "El animal no puede ser nulo.";
    
            assert ad != null : "El adoptante no puede ser nulo.";
    
            assert v != null : "El voluntario no puede ser nulo.";
    
            assert fecha != null && !fecha.after(new Date()) : "La fecha no puede ser nula ni estar en el futuro.";
    
            this.animal = a;
            this.adoptante = ad;
            this.voluntario = v;
            this.fecha = fecha;
        }
    
        public Date getFecha() {
            return this.fecha;
        }
    
        public void setFecha(Date fecha) {
            assert fecha != null && !fecha.after(new Date()) : "La fecha no puede ser nula ni estar en el futuro";
    
            this.fecha = fecha;
        }
    
        public Animal getAnimal() {
            return this.animal;
        }
    
        public Voluntario getVoluntario() {
            return this.voluntario;
        }
    
        public Adoptante getAdoptante() {
            return this.adoptante;
        }
    
        public boolean equals(Object obj) {
            if (this == obj) {
                return true;
            } else if (!(obj instanceof Adopcion)) {
                return false;
            } else {
                Adopcion adopcion = (Adopcion)obj;
                boolean ok = this.adoptante.equals(adopcion.adoptante) && this.animal.equals(adopcion.animal);
                return ok;
            }
        }
    
        public int hashCode() {
            return this.adoptante.hashCode() + this.animal.hashCode();
        }
    
        public String toString() {
            return String.format("Adopcion: %tY-%tB-%td, %s, %s", this.fecha, this.fecha, this.fecha, this.animal, this.adoptante);
        }
    }
    
\end{lstlisting}



\subsubsection{Clase Animal}\label{codigo:animal}
La clase \texttt{Animal} modela a un animal registrado en el sistema. Cada animal tiene un 
ID único, una fecha de nacimiento, un estado actual y está asociado a un \texttt{Refugio}.\par

\begin{lstlisting}[style = javaNormal, language=Java] 
    package sistema;
    import java.util.Date;
    
    public class Animal {
        private int ID;
        private Date nacimiento;
        private EstadoAnimal estadoAnimal; 
        private Adopcion adopcion;

        public Animal(int ID, Date nacimiento, EstadoAnimal estadoAnimal) {
            assert ID > 0 : "El ID del animal debe ser valido.";
            assert nacimiento != null : "La fecha de nacimiento no puede ser nula.";
            assert estadoAnimal != null : "El estado del animal debe estar definido.";

            this.ID = ID;
            this.nacimiento = nacimiento;
            this.estadoAnimal = estadoAnimal;
        }
    
        public EstadoAnimal getEstadoAnimal() {
            return estadoAnimal;
        }
        public void setEstadoAnimal(EstadoAnimal estadoAnimal) {
            assert estadoAnimal != null : "El estado del animal debe estar definido.";
            this.estadoAnimal = estadoAnimal;
        }
        public Date getNacimiento() {
            return nacimiento;
        }
        public void setNacimiento(Date nacimiento) {
            assert nacimiento != null : "La fecha de nacimiento no puede ser nula";
            this.nacimiento = nacimiento;
        }
        public Adopcion getAdopcion() {
            return this.adopcion;
        }
        public void setAdopcion(Adopcion adopcion){
            assert  adopcion != null;
            this.adopcion = adopcion;
        }
        public int getID() {
            return ID;
        }

        @Override
        public boolean equals(Object obj) {
            if( this == obj ) return true;
            if(obj instanceof Animal ){
                Animal animal = (Animal) obj;
                return this.ID == animal.ID;
            }
            return false;
        }
        @Override
        public int hashCode() {
            return Integer.hashCode(ID);
        }
        @Override
        public String toString() {
            return String.format("Animal: ID=%d, nacimiento=%tF, estado=%s", ID, nacimiento, estadoAnimal);
        }
    }
\end{lstlisting}



\subsection{Conclusión}

El diseño e implementación del código de andamiaje para el sistema se realizó siguiendo 
los principios fundamentales del diseño orientado a objetos, adaptados a los requerimientos 
específicos de este apartado. Se tomaron la decisiones de diseño adecuadas, como la gestión 
de asociaciones entre clases, la encapsulación de datos y la validación restricciones con \texttt{assert}, 
proporcionando un modelo consistente y flexible.\par
\vspace{0.15cm}
Una de las decisiones clave fue el uso combinado de asociaciones directas para relaciones 
simples y la reificación de asociaciones para relaciones más complejas junto con \texttt{get}
con conjuntos inmutables. Esto permitió mantener un equilibrio entre la simplicidad de las 
implementaciones directas, como la gestión de animales en el refugio, y la flexibilidad 
de las relaciones complejas, como las adopciones, donde se requieren atributos adicionales 
y validaciones específicas mientras protegíamos las listas de cada objeto en el sistema.\par
\vspace{0.15cm}
Además, la bidireccionalidad en relaciones como las adopciones, garantizó la 
consistencia del modelo al sincronizar automáticamente los datos entre entidades 
relacionadas.\par
TODO: AÑADIR COMO RESUMEN LAS DESICIONES QUE NO SE HAYANN  MENCIONADO DE LO QUE DIEGO A RECOPILADO EN LLAMADA

\newpage

\section{Apartado A}
\subsection{Diseño del Código de Andamiaje}
\subsection*{Introducción}
El concepto de \enquote{código de andamiaje} en el diseño orientado a objetos, 
particularmente en Java, hace referencia al conjunto de estructuras y métodos 
necesarios para implementar asociaciones entre clases, asegurando la 
consistencia y la integridad del sistema.\par
\vspace{0.15cm}
Su propósito es proporcionar un marco inicial sobre el que los desarrolladores pueden 
construir las funcionalidades particulares de un proyecto. Sin embargo, la línea entre
\enquote{andamiaje} e \enquote{implementación completa} puede ser fina ya que estamos 
añadiendo nuevas funcionalidades que luego se convierten en el nuevo marco inicial para 
futuros cambios que vayamos a realizar en los próximos apartados. A continuación se 
exponen las decisiones diseño que completan el andamiaje inicial.

\subsection{Análisis de opciones de Diseño}
En esta sección se exponen las diferentes opciones de diseño para la implementación 
del sistema de gestión de un refugio de animales, conforme al modelo de clases y 
operaciones proporcionado.\par
\vspace{0.15cm}
El sistema requiere gestionar los socios del refugio (quienes pueden desempeñar 
diferentes roles como voluntarios, donantes y adoptantes), así como el registro, 
adopción y donación de animales. Además, se deben considerar las relaciones entre 
las entidades (socios, animales, refugio, donaciones) y las restricciones del sistema, 
como la consistencia en los datos y las operaciones.\par
\vspace{0.15cm}
Importante mencionar que en nuestro diseño, \textbf{no hemos aplicado un único enfoque de manera 
exclusiva}. Hemos adoptado una combinación de estrategias dependiendo de las necesidades de 
cada relación dentro del sistema justificándolo de forma adecuada.

\subsubsection{Manejo de las Asociaciones}

\begin{description}
    \item[a)] \textbf{Asociación Directa (Sin Reificación\footnote{
        La reificación es una técnica en programación 
        orientada a objetos que se basa en convertir un 
        concepto abstracto, como una relación, en una 
        entidad concreta o clase.})}
\end{description}

\textit{\textbf{Descripción:}}  
En este enfoque, las asociaciones entre clases se implementan directamente como atributos 
en las clases relacionadas.\par
\vspace{0.15cm}
Esta práctica es fácil de implementar porque la cantidad de clases a gestionar 
y el número de clases necesarias para representar las relaciones es menor. No obstante, 
añadir atributos adicionales a las asociaciones (como fechas en el proceso 
de adopción) puede traer problemas de consistencia al manejar relaciones complejas como la 
de un socio con múltiples roles (lo exploraremos en futuras secciones).\par
\newpage % Para que no se ve aglomerada la definición en la footnote
\textbf{Ejemplo: Implementación de \texttt{Refugio} con asociación directa a \texttt{Animal}}\par
La asociación es directa porque \texttt{Refugio} gestiona los \texttt{Animales} mediante un 
\texttt{set}, sin una clase intermedia que relacione ambas entidades.(ver Código~\ref{codigo:refugio})\par
TODO: CAMBIAR EL CODE\par
En nuestra implementación, hemos decidido la utilización de un \texttt{Set} para las estructuras de datos 
en vez de \texttt{List} por varios factores:

\begin{itemize}
    \item \textbf{Control de elementos repetidos:} Los \texttt{Set} usan los métodos de \texttt{hashCode} e \texttt{equals}
    para hacer la comprobación de la existencia de elementos en la colección. Si un nuevo elemento coincide con uno existente, 
    no se inserta, evitando comprobaciones adicionales que haríamos con el uso de \texttt{List}.
    \item \textbf{Complejidad algorítmica:} En los \texttt{HashSet}, la búsqueda y la inserción tienen una complejidad de \(O(1)\), 
    ya que se basan en tablas hash. Por otro lado, en una \texttt{List} (como \texttt{ArrayList}), las operaciones de búsqueda tienen una 
    complejidad de \(O(n)\), porque requiere iterar sobre los elementos.
    \item \textbf{Orden de los elementos:} En nuestra implementación, no es necesario mantener un orden específico en las colecciones. 
    Por esta razón, el uso \texttt{Set} es más adecuado que \texttt{List} teniendo en cuenta los puntos anteriores.
\end{itemize}
\vspace{0.45cm}

\begin{description}
    \item[b)] \textbf{Reificación de la Asociación (Clase de Asociación)}
\end{description}

\textit{\textbf{Descripción:}}  
En este enfoque, las asociaciones complejas entre clases se modelan mediante clases 
intermedias. Por ejemplo, la clase \texttt{Adopcion} representa la relación entre un 
\texttt{Animal}, un \texttt{Adoptante}, y un \texttt{Voluntario}, incluyendo atributos 
como \texttt{fechaAdopcion} para capturar detalles específicos de la relación.\par
\vspace{0.15cm}
Como la reificación nos permiten agregar atributos y métodos específicos a las relaciones,
facilita la implementación de restricciones complejas relacionadas con la asociación.
Sin embargo, aumenta el número de clases y relaciones a gestionar, lo que hace el diseño más denso
ya que debemos implementar y gestionar las clases de asociación, así como los métodos para 
acceder a las relaciones.\par
\vspace{0.15cm}

\textbf{Nuestra Implementación: Uso de \texttt{Adopcion} como clase de asociación:}
\begin{itemize}
    \item Se representa la relación entre \texttt{Animal}, \texttt{Adoptante} y 
    \texttt{Voluntario} mediante una clase intermedia. (ver Código~\ref{codigo:adopcion})
    \item Atributos como \texttt{fecha} añaden flexibilidad al modelo, permitiendo 
    capturar detalles adicionales de la relación.
    \item Se gestionan las relaciones bidireccionales entre las entidades involucradas, 
    asegurando consistencia en los datos.
\end{itemize}

Elegimos este enfoque, ya que proporciona la flexibilidad necesaria para agregar atributos 
y gestionar reglas de negocio específicas. La asociación directa fue descartada porque no 
permitiría capturar detalles adicionales, como la fecha de adopción, ni manejar eficientemente 
las restricciones relacionadas con el proceso de adopción.



\subsubsection{Manejo de Roles de los Socios}

\begin{description}
    \item[a)] \textbf{Subclases Específicas para cada Rol}
\end{description}

\textit{\textbf{Descripción:}}  
Cada rol (\texttt{Voluntario}, \texttt{Donante}, \texttt{Adoptante}) se implementa como una 
subclase de la clase \texttt{Socio}. Esto permite encapsular los atributos y métodos 
específicos de cada rol dentro de su respectiva subclase.
\vspace{0.15cm}


Esto proporciona claridad al diseño, ya que cada rol está claramente representado con 
métodos específicos para su comportamiento. Además, permite encapsular los atributos y 
métodos particulares de cada tipo de socio, lo que mejora la organización y legibilidad 
del código. Aunque tiene una limitación significativa para manejar roles múltiples, ya que no permite que 
un socio asuma más de un rol sin duplicar instancias de las subclases. Esto hace que el 
diseño sea rígido y menos flexible en casos donde los roles pueden cambiar dinámicamente 
o coexistir (volveremos a hablar de esto en los siguientes apartados).\par
\vspace{0.15cm}



\textbf{Ejemplo: Subclases específicas para los roles}\par
El diseño tiene implementadas las subclases \texttt{Donante}, \texttt{Adoptante}, 
y \texttt{Voluntario} como extensiones de la clase \texttt{Socio}.\par
(ver Código~\ref{codigo:donante}, \ref{codigo:adoptante}, \ref{codigo:voluntario})

\begin{description}
    \item[b)] \textbf{Uso de Composición de Roles}
\end{description}

\textit{\textbf{Descripción:}}  
En lugar de modelar cada rol como una subclase de \texttt{Socio}, este enfoque utiliza 
la composición para permitir que un socio tenga múltiples roles simultáneamente. 
Cada rol se modela como una clase independiente que puede ser asociada dinámicamente a 
un \texttt{Socio} mediante una colección de roles.
\vspace{0.15cm}


Este enfoque es mucho más flexible, ya que permite asignar múltiples roles a un socio 
sin necesidad de crear combinaciones de subclases. También simplifica el manejo de 
roles dinámicos y permite cambios en tiempo de ejecución.
Puede reducir la claridad del diseño, ya que no existe una distinción explícita entre 
los diferentes tipos de socios. Además, requiere implementar lógica adicional para 
validar qué operaciones son aplicables para los roles asignados a cada socio.\par
\vspace{0.15cm}


\textbf{Nuestra implementación: Uso de Subclases Específicas para cada Rol:}\par  
\vspace{0.15cm}
Hemos decidido no cambiar como están implementados los roles mediante subclases específicas 
(para este apartado) en lugar de composición. Esto se debe a que en nuestro modelo actual, 
los roles están claramente definidos y no se requiere que un socio tenga múltiples roles de 
manera simultánea. Además:
\begin{itemize}
    \item La claridad y encapsulación que proporciona la herencia permiten manejar las responsabilidades y comportamientos específicos de cada tipo de socio de manera aislada.
    \item Aunque la composición sería más flexible, introducirá complejidad adicional innecesaria para los requisitos actuales del sistema.
\end{itemize}

No obstante, los requisitos del sistema cambiarán en futuros apartados pidiendo que un socio
tenga múltiples roles simultáneamente. En el apartado correspondiente, se discute porque 
la composición sería una solución más adecuada y como se ha implementado.




\subsubsection{Consistencia y Gestión de Datos}

\begin{description}
    \item[a)] \textbf{Encapsulación Estricta}
\end{description}

\textit{\textbf{Descripción:}}  
Este enfoque restringe el acceso directo a los atributos y métodos de las clases mediante 
el uso de visibilidad privada. Para interactuar con los atributos, se proporcionan métodos 
públicos controlados (\texttt{getters} y \texttt{setters}) que incluyen validaciones (mediante 
\texttt{asserts}) para garantizar que los datos se mantengan en un estado consistente.
\vspace{0.15cm}

    Debido a esto nos aseguramos que los datos sean modificados de manera controlada y consistente.
    Facilita la incorporación de validaciones o pruebas unitarias lo que completa el comportamiento del
    esperado del sistema.Dicho esto también se requiere de implementar más métodos, como \texttt{getters}, \texttt{setters} y validaciones 
    necesarias, lo que aumenta la cantidad de código. Además, estas validaciones podrían hacer 
    que el diseño sea más extenso y menos directo.\par
    \vspace{0.15cm}


\textbf{Ejemplo en el sistema: Uso de encapsulación estricta en la clase \texttt{Animal}:}\par  
En nuestro sistema, la clase \texttt{Animal} utiliza atributos privados y métodos públicos 
controlados para garantizar consistencia y validaciones en tiempo de ejecución.Este enfoque 
asegura que cualquier intento de modificar el estado de un \texttt{Animal} pase por 
validaciones definidas en los métodos públicos. (ver Código~\ref{codigo:animal})


\begin{description}
    \item[b)] \textbf{Uso de Colecciones Inmutables}
\end{description}

\textit{\textbf{Descripción:}}  
En este enfoque, las colecciones utilizadas para representar relaciones (por ejemplo, 
listas o conjuntos de \texttt{Animal} en \texttt{Refugio}) son inmutables. Esto garantiza 
que las relaciones no puedan ser modificadas accidentalmente fuera de las clases que las 
gestionan.
\vspace{0.15cm}

    Mejora la integridad del sistema al garantizar que las relaciones no se modifiquen de 
    manera no controlada.
    Por otro lado introduce rigidez ya que no permite realizar cambios dinámicos en las relaciones sin 
    reemplazar completamente la colección. Esto puede dificultar la gestión de operaciones 
    como agregar o eliminar elementos.\par
    \vspace{0.15cm}

\textbf{Ejemplo de Uso de Colecciones Inmutables en la Clase \texttt{Refugio}:}\par  
En nuestro sistema, el método \texttt{getAnimalesRegistrados} de la clase \texttt{Refugio} 
devuelve una vista inmutable de los animales registrados. Esto asegura que las listas no 
puedan modificarse desde fuera de la clase. Además, el uso de \texttt{Collections.}\texttt{enumeration} 
garantiza que la colección de animales no pueda ser alterada fuera de la clase \texttt{Refugio}, 
manteniendo la consistencia de los datos. (ver Código~\ref{codigo:refugio} aunque se implementa en varias clases)

\textbf{Decisión Tomada: Encapsulación Controlada con Enumerations:}\par
En nuestro diseño, optamos por una encapsulación controlada en lugar de colecciones 
completamente inmutables. Esto se debe a que:
\begin{itemize}
    \item Proporciona flexibilidad para realizar cambios dinámicos en las colecciones a 
    través de métodos controlados, lo que es necesario para operaciones como agregar o 
    eliminar animales en un refugio.
    \item Utilizar enumeraciones en los métodos \texttt{get} garantiza que las colecciones 
    no se modifiquen desde fuera de las clases, preservando la integridad de los datos.
\end{itemize}

Este enfoque combina lo mejor de ambos mundos: flexibilidad para realizar cambios controlados 
y protección contra modificaciones accidentales.



\subsubsection{Estrategias para Manejo de Adopciones y Donaciones}

\begin{description}
    \item[a)] \textbf{Operaciones Independientes}
\end{description}

\textit{\textbf{Descripción:}}  
En este enfoque, cada operación (como adopción, registro de animales o donaciones) se 
implementa de forma independiente, sin interacciones entre ellas. Cada acción tiene su 
propio método o flujo lógico separado.
\vspace{0.15cm}

    Este diseño asegura que las operaciones están bien definidas y separadas, lo que 
    facilita su comprensión. Además, la simplicidad del diseño permite que sea directo y 
    fácil de implementar.Sin Embargo, puede llevar a la duplicación de código si varias operaciones comparten lógica similar 
    (por ejemplo, validar la existencia de un animal o donante). También puede ser menos 
    flexible, ya que cualquier cambio en una operación podría requerir modificaciones en 
    múltiples partes del sistema.\par
    \vspace{0.15cm}

\textbf{Ejemplo: Operaciones independientes en nuestra implementación}  
En nuestro diseño, las adopciones y donaciones se gestionan mediante clases específicas 
(\texttt{Adopcion} y \texttt{Donacion}), cada una con su propia lógica y atributos.

\begin{description}
    \item[b)] \textbf{Reutilización de Lógica Compartida entre Operaciones}
\end{description}

\textit{\textbf{Descripción:}}  
En lugar de mantener las operaciones completamente separadas, este enfoque identifica y 
reutiliza lógica común entre las operaciones (como validaciones o actualizaciones de estado). 
Aunque no implementamos este enfoque en nuestra solución actual, sería posible centralizar las 
validaciones comunes mediante una clase auxiliar, como se muestra en el siguiente ejemplo.\par
\vspace{0.15cm}
Esta opción reduce la duplicación de código, ya que la lógica compartida se implementa una sola vez
y facilita la incorporación de nuevas funcionalidades relacionadas con las operaciones existentes. Pero 
puede introducir una dependencia más estrecha entre las clases, lo que podría aumentar 
la complejidad del sistema en caso de cambios importantes.

\textbf{Ejemplo Propuesto: Centralización de Validaciones}  
Aunque nuestra implementación actual gestiona las validaciones directamente en los métodos 
de las clases relevantes (\texttt{Adoptante}, \texttt{Donante}). Por ejemplo, podríamos considerar una 
clase auxiliar para centralizarlas en el futuro:

\begin{lstlisting}[style = javaEspecifico, language=Java, caption={Clase Auxiliar para Validaciones}] 
public class Validacion {
    public static void validarEstadoAnimal(Animal animal, EstadoAnimal estadoEsperado) {
        assert animal.getEstadoAnimal() == estadoEsperado : 
            "El estado del animal no coincide con el esperado.";
    }
}
\end{lstlisting}

En nuestra implementación actual, la validación se realiza directamente dentro de las clases:

\begin{lstlisting}[style = javaEspecifico, language=Java, caption={Manejo de validaciones dentro del método Adoptar}] 
public void adoptar(Animal a, Voluntario v) {
    assert a.getEstadoAnimal() == EstadoAnimal.DISPONIBLE : 
        "El animal no esta disponible.";
    Adopcion adopcion = new Adopcion(a, this, v, new Date());
    adopciones.add(adopcion);
}
\end{lstlisting}
\vspace{0.15cm}
\textbf{Decisión Tomada: Mantener las Validaciones en las Clases Relevantes}\par
En nuestra implementación actual, las validaciones se realizan directamente en las clases 
donde ocurren las operaciones. Esto se alinea con la claridad y simplicidad requeridas por 
el sistema. Sin embargo, reconocemos que la centralización de lógica compartida podría ser 
útil en sistemas más complejos. FIXME: EL GETTER ES MAL EJEMPLO DE ESTO (ver los getters en Código~\ref{codigo:adopcion} por ejemplo) 



\subsubsection{Representación de Relaciones en el Sistema}

\begin{description}
    \item[a)] \textbf{Relaciones Unidireccionales}
\end{description}

\textit{\textbf{Descripción:}}  
En una relación unidireccional, solo una entidad tiene conocimiento de la relación. 
Por ejemplo, un \texttt{Adoptante} puede conocer al \texttt{Animal} que adopta, 
pero el \texttt{Animal} no necesita saber nada sobre el \texttt{Adoptante}.\par
\vspace{0.15cm} 
Estas relaciones gestionan la relación con una clase, lo que reduce la complejidad del sistema.
El problema, es que limita las consultas entre clases relacionadas y puede volverse más complejo 
añadir funcionalidades.\par
\vspace{0.15cm}
En nuestro sistema, todas las relaciones unidireccionales con 1 a muchos, por ejemplo:
TODO: EXPANDIR ESTA IDEA

\begin{description}
    \item[b)] \textbf{Relaciones Bidireccionales}
\end{description}

\textit{\textbf{Descripción:}}  
En una relación bidireccional, ambas entidades conocen y mantienen referencias mutuas. 
Por ejemplo, cuando un \texttt{Adoptante} adopta un \texttt{Animal}, ambos se actualizan 
mutuamente para reflejar la relación.\par
\vspace{0.15cm}
Las relaciones de este estilo garantizan la consistencia de los datos, ya que ambas partes relacionadas están 
sincronizadas al mantener referencias mutuas explícitas. Sin embargo, como hay que tener una sincronización constante 
entre las todas las clases relacionadas, si tuviéramos muchas relaciones bidireccionales puede dificultar el mantenimiento
por que genera un alto nivel de acoplamiento.\par
\vspace{0.15cm}
\textbf{Ejemplo en el Sistema: Relaciones Bidireccionales en \texttt{Adopcion}}\par 
En nuestro diseño, la relación entre \texttt{Animal}, \texttt{Adoptante}, y 
\texttt{Voluntario} es bidireccional y se asegurando consistencia en ambas direcciones
reflejando los cambios realizados en una clase en las demás involucradas.
(ver Código~\ref{codigo:adopcion} como, por ejemplo, se actualiza el estado del animal tras ser adoptado)\par
\vspace{0.15cm}
\textbf{Decisión Tomada: Relaciones Bidireccionales}\par
Hemos implementado relaciones bidireccionales para las asociaciones complejas del sistema, 
como las adopciones, ya que garantizan consistencia y sincronización entre las entidades 
relacionadas. Sin embargo, para relaciones más simples, como la lista de animales en un 
refugio, usamos relaciones unidireccionales para mantener la simplicidad.

\subsection{Diagrama de Diseño}

\begin{figure}[H]
    \centering
     \includegraphics[width=1\linewidth]{assets/umaLogo.png}
     \caption{UMA}
\end{figure}
FALTA HACER Y METER EL DIAGRAMA DE DISEÑO DE NUESTRO SISTEMA.RECOMENDABLE PONER UNA MINI EXPLICACIÓN.

\newpage



\subsection{Implementación del Modelo}
\subsubsection{Clase Socio}\label{codigo:socio}
La clase \texttt{Socio} es abstracta y representa la base para las distintas subclases: 
\texttt{Adoptante}, \texttt{Voluntario}, y \texttt{Donante}. Esta clase asegura que 
cada socio tenga un \texttt{ID} único, una fecha de registro válida y un refugio asociado.

\begin{lstlisting}[style = javaNormal, language=Java] 
    package sistema;

    import java.util.Collections;
    import java.util.Date;
    
    public abstract class Socio {
        private int ID;
        private Date registro;
        private final Refugio refugioAsociado;
    
        public Socio(int ID, Date fechaRegistro, Refugio refugioAsociado) {
            assert ID > 0 : "El ID del socio debe ser valido.";
    
            assert fechaRegistro != null : "La fecha de registro no puede ser nula.";
    
            assert refugioAsociado != null : "El refugio asociado no puede ser nulo.";
    
            this.ID = ID;
            this.registro = fechaRegistro;
            this.refugioAsociado = refugioAsociado;
            refugioAsociado.addSocio(this);
    
            assert refugioAsociado.getSocios().hasMoreElements() && Collections.list(refugioAsociado.getSocios()).contains(this);
        }
    
        public int getID() {
            return this.ID;
        }
    
        private void setID(int ID) {
            assert ID > 0;
    
            assert Collections.list(this.getRefugio().getSocios()).stream().noneMatch((s) -> {
                return s.getID() == ID;
            }) : "El ID ya existe en el refugio";
    
            this.ID = ID;
        }
    
        public Date getRegistro() {
            return this.registro;
        }
    
        public void setRegistro(Date fechaRegistro) {
            assert fechaRegistro != null;
    
            this.registro = fechaRegistro;
        }
    
        public Refugio getRefugio() {
            return this.refugioAsociado;
        }
    
        public boolean equals(Object obj) {
            if (this == obj) {
                return true;
            } else if (obj instanceof Socio) {
                Socio socio = (Socio)obj;
                return this.ID == socio.ID;
            } else {
                return false;
            }
        }
    
        public int hashCode() {
            return Integer.hashCode(this.ID);
        }
    }
\end{lstlisting}



\subsubsection{Clase Donante}\label{codigo:donante}
La clase \texttt{Donante} extiende de \texttt{Socio} y gestiona las donaciones realizadas 
por un socio. Las donaciones se almacenan en un \texttt{HashSet} para asegurar que no hayan elementos repetidos y por eficiencia. (cómo se explicó en \ref{page:Consideraciones}) 
Por otro lado, hemos decidido implementar al crear un nuevo objeto Donante en el sistema,
el constructor llamará directamente al método donar ya que es una condición necesaria para ser donante. Las donaciones que sean modificadas son tratadas desde la clase Donación y
el \texttt{HashSet donaciones} se actualizará.


\begin{lstlisting}[style = javaNormal, language=Java] 
    package sistema;

    import java.time.LocalDate;
    import java.time.ZoneId;
    import java.util.*;
    
    public class Donante extends Socio{
        private Set<Donacion> donaciones;
        public Donante(int ID, Date date,Refugio r, float cantidad) {
            super(ID,date,r);
            assert cantidad > 0 : "La cantidad inicial donada debe ser mayor a cero.";
            donaciones = new HashSet<>();
            donar(cantidad);
        }

        public void donar(float cantidad){
            assert cantidad > 0 : "La cantidad donada debe ser mayor a cero.";
            assert Collections.list(this.getRefugio().getSocios()).contains(this): "El socio debe ser donante antes de poder donar";
            Donacion d = new Donacion(cantidad, Date.from(fechaDonacion.atStartOfDay(ZoneId.systemDefault()).toInstant()));
            addDonacion(d);
            Refugio r = super.getRefugio();
            r.setLiquidez(r.getLiquidez() + cantidad);
            assert donaciones.contains(d);
        }
    
        protected void addDonacion(Donacion donacion){
            assert donacion != null: "La donacion no puede ser nula";
            donaciones.add(donacion);
        }
    
        protected void removeDonacion(Donacion donacion){
            assert donacion != null : "La donacion no puede ser nula.";
            if (donaciones.contains(donacion) && donaciones.size() > 1) {
                donaciones.remove(donacion);
            } else if (donaciones.contains(donacion) && donaciones.size() == 1) {
                System.out.println("Todo donante debe tener al menos una donacion, estas intentando eliminar la unica donacion asociada a este socio donante");
            } else {
                System.out.println("Este socio no ha realizado la donacion que intentas eliminar");
            }
        }
        public Enumeration<Donacion> getDonaciones(){
            return Collections.enumeration(this.donaciones);
        }
    
        @Override
        public String toString() {
            return "Donante " + super.getID();
        }
    }
    
\end{lstlisting}

\subsubsection{Clase Adoptante}\label{codigo:adoptante}
La clase \texttt{Adoptante} extiende de \texttt{Socio} y simula las adopciones realizadas 
por un adoptante. Las adopciones se almacenan en un \texttt{HashSet} para evitar adopciones duplicadas.
\begin{lstlisting}[style = javaNormal, language=Java] 
    package sistema;

    import java.util.*;
    
    public class Adoptante extends Socio {
        private Set<Adopcion> adopciones;
    
        public Adoptante(int ID, Date date, Refugio r) {
            super(ID, date,r);
            adopciones = new HashSet<>();
        }
        public void adoptar(Animal a, Voluntario v) {
            assert a.getEstadoAnimal() == EstadoAnimal.DISPONIBLE: "El animal ya ha sido adoptado";
            Refugio refugioDelVoluntario = v.getRefugio();
            a.setEstadoAnimal(EstadoAnimal.ADOPTADO);
            refugioDelVoluntario.removeAnimalesRefugiados(a);
            v.tramitarAdopcion(a, this);
        }
    
        protected void addAdopcion(Adopcion a){
            this.adopciones.add(a);
        }
    
        protected void removeAdopcion(Adopcion a){
            if (adopciones.contains(a)) adopciones.remove(a);
            else System.out.println("Este animal ya no esta asociado al adoptante");
        }
        public Enumeration<Adopcion> getAdopciones(){
            return Collections.enumeration(adopciones);
        }
        @Override
        public boolean equals(Object obj) {
            if (this == obj) return true;
            if(obj instanceof Adoptante ){
                Adoptante adoptante = (Adoptante) obj;
                return adoptante.getID() == this.getID();
            }
            return false;
        }
        @Override
        public int hashCode() {
            return Integer.hashCode(this.getID());
        }
    
        @Override
        public String toString() {
            return "Adoptante " + super.getID();
        } 
    }    
\end{lstlisting}

\subsubsection{Clase Voluntario}\label{codigo:voluntario}
La clase \texttt{Voluntario} extiende de \texttt{Socio} y gestiona los trámites de 
adopción realizados por un voluntario. Los voluntarios se almacenan en un \texttt{HashSet},
evitando así voluntarios duplicados. 
\begin{lstlisting}[style = javaNormal, language=Java] 
    package sistema;

    import java.time.LocalDate;
    import java.time.ZoneId;
    import java.util.*;
    
    public class Voluntario extends Socio{
        Set<Adopcion> tramites;
        
        public Voluntario(int ID, Date date,Refugio r) {
            super(ID, date,r);
            tramites = new HashSet<>();
        }
        public void tramitarAdopcion(Animal a, Adoptante ad){
            assert ad != null : "El adoptante no puede ser nulo.";
            LocalDate fechaAdopcion = LocalDate.now();
            Adopcion adopcion = new Adopcion(a, ad, this, Date.from(fechaAdopcion.atStartOfDay(ZoneId.systemDefault()).toInstant()));
            addTramite(adopcion);
            ad.addAdopcion(adopcion);
        }
        public void registrar(Animal a){
            Refugio r = super.getRefugio();
            assert r != null : "El refugio asociado no puede ser nulo.";
            assert a != null : "El animal no puede ser nulo.";
            a.setEstadoAnimal(EstadoAnimal.DISPONIBLE);
            r.addAnimalesRefugiados(a);
    
        }
        public Enumeration<Adopcion> getTramites(){
            return Collections.enumeration(tramites);
        }
        protected void addTramite(Adopcion ad){
            assert ad != null : "El tramite de adopcion no puede ser nulo.";
            tramites.add(ad);
    
        }
        protected void removeTramite(Adopcion ad){
            assert  ad != null: "El tramite de adopcion no puede ser nulo.";
            tramites.remove(ad);
        }
    
    
        @Override
        public String toString() {
            return "Voluntario " + super.getID();
        }
    }
    
\end{lstlisting}

\subsubsection{Clase Refugio}\label{codigo:refugio}
La clase \texttt{Refugio} gestiona el conjunto de \texttt{Socios} y \texttt{Animales}.\par
\textbf{liquidez} está declarado como un \texttt{float} por las razones que se exponen en \ref{page:Consideraciones}.


\begin{lstlisting}[style = javaNormal, language=Java] 
package sistema;
import java.util.*;

public class Refugio {
    private float liquidez;
    private Set<Animal> animalesRegistrados;
    private Set<Animal> animalesRefugiados;
    private Set<Socio> socios;

    public Refugio(float liquidez) {
        assert liquidez >= 0 : "La liquidez debe ser no negativa.";
        this.liquidez = liquidez;
        animalesRefugiados = new HashSet<>();
        animalesRegistrados = new HashSet<>();
        socios = new HashSet<>();
    }

    public float getLiquidez() 
        return liquidez;
    }
    public void setLiquidez(float liquidez) {
        assert liquidez >= 0 : "La liquidez debe ser no negativa";
        this.liquidez = liquidez;
    }
    protected void addSocio(Socio s) {
        assert s != null : "El socio no puede ser nulo.";
        if(socios.contains(s)) {
            System.out.println("El socio ya esta registrado.");
            return;
        }
        socios.add(s);
    }
    protected void removeSocio(Socio s) {
        assert s != null : "El socio no puede ser nulo.";
        if (socios.contains(s)) {
            socios.remove(s);
        } else {
            System.out.println("Este socio no esta registrado en el refugio.");
        }
    }

    public Enumeration<Animal> getAnimalesRegistrados() {
        return Collections.enumeration(animalesRegistrados);
    }
    public Enumeration<Animal> getAnimalesRefugiados() {
        return Collections.enumeration(animalesRefugiados);
    }
    public Enumeration<Socio> getSocios() {
        return Collections.enumeration(socios);
    }

    public List<Adoptante> getAdoptantes() {
        List<Adoptante> adoptantes = new ArrayList<>();
        for (Socio s : socios) {
            if (s instanceof Adoptante) {
                adoptantes.add((Adoptante) s);
            }
        }
        return adoptantes;
    }
    public List<Voluntario> getVoluntarios() {
        List<Voluntario> voluntarios = new ArrayList<>();
        for (Socio s : socios) {
            if (s instanceof Voluntario) {
                voluntarios.add((Voluntario) s);
            }
        }
        return voluntarios;
    }
    public List<Donante> getDonantes() {
        List<Donante> donantes = new ArrayList<>();
        for (Socio s : socios) {
            if (s instanceof Donante) {
                donantes.add((Donante) s);
            }
        }
        return donantes;
    }
    public void registrar(Animal a){
        this.addAnimalesRegistrados(a);
    }
    protected void addAnimalesRefugiados(Animal a){
        assert a != null : "El animal no puede ser nulo.";
        if(!animalesRefugiados.contains(a)){
            animalesRefugiados.add(a);
            this.addAnimalesRegistrados(a);
        } else System.out.println("Este animal ya esta en el refugio.");
    }
    private void addAnimalesRegistrados(Animal a){
        assert a != null : "El animal no puede ser nulo.";
        if (!animalesRegistrados.contains(a)) {
            animalesRegistrados.add(a);
        } else {
            System.out.println("El animal ya esta registrado.");
        }
    }

    protected void removeAnimalesRefugiados(Animal a){
        assert a != null : "El animal no puede ser nulo.";
        if (animalesRefugiados.contains(a)) {
            animalesRefugiados.remove(a);
        } else {
            System.out.println("El animal no se encuentra en este Refugio.");
        }
    }
    protected void removeAnimalesRegistrados(Animal a){
        assert a != null : "El animal no puede ser nulo.";
        if (animalesRegistrados.contains(a) && animalesRegistrados.size() > 1) {
            animalesRegistrados.remove(a);
        } else if (animalesRegistrados.contains(a) && animalesRegistrados.size() == 1) {
            System.out.println("Todo refugio debe tener al menos un animal registrado, estas intentando eliminar el unico animal existente.");
        } else {
            System.out.println("El animal no se encuentra en este Refugio.");
        }
    }

    public void mostrarAnimalesRefugiados(){
        System.out.println(animalesRefugiados.toString());
    }
    public void mostrarAnimalesRegistrados(){
        System.out.println(animalesRegistrados.toString());
    }
    public void mostrarSocios() {
        for (Socio s : socios) {
            System.out.println(s);
        }
    }
    public void mostrarSociosPorTipo() {
        System.out.println("Adoptantes: " + getAdoptantes());
        System.out.println("Voluntarios: " + getVoluntarios());
        System.out.println("Donantes: " + getDonantes());
    }

    @Override
    public String toString() {
        StringBuilder sb = new StringBuilder();
        sb.append("Animales Registrados: ").append(animalesRegistrados).append("\n");
        sb.append("Animales Refugiados: ").append(animalesRefugiados).append("\n");
        sb.append("Socios: ").append(socios).append("\n");
        sb.append("Liquidez: ").append(liquidez);
        return sb.toString();
    }
}
\end{lstlisting}



\subsubsection{Clase Donacion}\label{codigo:donacion}
La clase \texttt{Donacion} representa una donación realizada por un \texttt{Donante}. 
Incluye la cantidad que como anteriormente mencionamos en \ref{page:Consideraciones} por temas de eficiencia es un \texttt{float},
la fecha de la donación y el donante asociado. Las validaciones 
aseguran que los valores sean válidos en el momento de la creación de la instancia.

\begin{lstlisting}[style = javaNormal, language=Java] 
    package sistema;

    import java.util.Date;
    import java.util.Objects;
    
    public class Donacion {
        private float cantidad;
        private Date date;
    
        public Donacion(float cantidad, Date date) {
            assert cantidad > 0.0F : "La cantidad debe ser positiva.";
    
            assert date != null && !date.after(new Date()) : "La fecha no puede ser nula ni estar en el futuro.";
    
            this.cantidad = cantidad;
            this.date = date;
        }
    
        public float getCantidad() {
            assert this.cantidad > 0.0F : "La cantidad no puede ser nula.";
    
            return this.cantidad;
        }
    
        public void setCantidad(float cantidad) {
            this.cantidad = cantidad;
        }
    
        public Date getDate() {
            assert this.date != null : "La fecha no puede ser nula.";
    
            return this.date;
        }
    
        public void setDate(Date date) {
            this.date = date;
        }
    
        public String toString() {
            return String.format("Donacion: %.2f, %tY-%tB-%td", this.cantidad, this.date, this.date, this.date);
        }
    
        public boolean equals(Object o) {
            if (this == o) {
                return true;
            } else if (o != null && this.getClass() == o.getClass()) {
                Donacion donacion = (Donacion)o;
                return Float.compare(this.cantidad, donacion.cantidad) == 0 && Objects.equals(this.date, donacion.date);
            } else {
                return false;
            }
        }
    
        public int hashCode() {
            return Objects.hash(new Object[]{this.cantidad, this.date});
        }
    }
\end{lstlisting}



\subsubsection{Clase Adopcion}\label{codigo:adopcion}
La clase \texttt{Adopcion} modela una adopción de un \texttt{Animal} realizada por un 
\texttt{Adoptante}, gestionada por un \texttt{Voluntario}. Implementa la bidireccionalidad 
entre estas entidades para mantener consistencia en las asociaciones.

\begin{lstlisting}[style = javaNormal, language=Java] 
    package sistema;

    import java.util.Date;
    
    public class Adopcion {
        private Date fecha;
        private final Animal animal;
        private final Adoptante adoptante;
        private final Voluntario voluntario;
    
        public Adopcion(Animal a, Adoptante ad, Voluntario v, Date fecha) {
            assert a != null : "El animal no puede ser nulo.";
    
            assert ad != null : "El adoptante no puede ser nulo.";
    
            assert v != null : "El voluntario no puede ser nulo.";
    
            assert fecha != null && !fecha.after(new Date()) : "La fecha no puede ser nula ni estar en el futuro.";
    
            this.animal = a;
            this.adoptante = ad;
            this.voluntario = v;
            this.fecha = fecha;
        }
    
        public Date getFecha() {
            return this.fecha;
        }
    
        public void setFecha(Date fecha) {
            assert fecha != null && !fecha.after(new Date()) : "La fecha no puede ser nula ni estar en el futuro";
    
            this.fecha = fecha;
        }
    
        public Animal getAnimal() {
            return this.animal;
        }
    
        public Voluntario getVoluntario() {
            return this.voluntario;
        }
    
        public Adoptante getAdoptante() {
            return this.adoptante;
        }
    
        public boolean equals(Object obj) {
            if (this == obj) {
                return true;
            } else if (!(obj instanceof Adopcion)) {
                return false;
            } else {
                Adopcion adopcion = (Adopcion)obj;
                boolean ok = this.adoptante.equals(adopcion.adoptante) && this.animal.equals(adopcion.animal);
                return ok;
            }
        }
    
        public int hashCode() {
            return this.adoptante.hashCode() + this.animal.hashCode();
        }
    
        public String toString() {
            return String.format("Adopcion: %tY-%tB-%td, %s, %s", this.fecha, this.fecha, this.fecha, this.animal, this.adoptante);
        }
    }
    
\end{lstlisting}



\subsubsection{Clase Animal}\label{codigo:animal}
La clase \texttt{Animal} modela a un animal registrado en el sistema. Cada animal tiene un 
ID único, una fecha de nacimiento, un estado actual y está asociado a un \texttt{Refugio}.\par

\begin{lstlisting}[style = javaNormal, language=Java] 
    package sistema;
    import java.util.Date;
    
    public class Animal {
        private int ID;
        private Date nacimiento;
        private EstadoAnimal estadoAnimal; 
        private Adopcion adopcion;

        public Animal(int ID, Date nacimiento, EstadoAnimal estadoAnimal) {
            assert ID > 0 : "El ID del animal debe ser valido.";
            assert nacimiento != null : "La fecha de nacimiento no puede ser nula.";
            assert estadoAnimal != null : "El estado del animal debe estar definido.";

            this.ID = ID;
            this.nacimiento = nacimiento;
            this.estadoAnimal = estadoAnimal;
        }
    
        public EstadoAnimal getEstadoAnimal() {
            return estadoAnimal;
        }
        public void setEstadoAnimal(EstadoAnimal estadoAnimal) {
            assert estadoAnimal != null : "El estado del animal debe estar definido.";
            this.estadoAnimal = estadoAnimal;
        }
        public Date getNacimiento() {
            return nacimiento;
        }
        public void setNacimiento(Date nacimiento) {
            assert nacimiento != null : "La fecha de nacimiento no puede ser nula";
            this.nacimiento = nacimiento;
        }
        public Adopcion getAdopcion() {
            return this.adopcion;
        }
        public void setAdopcion(Adopcion adopcion){
            assert  adopcion != null;
            this.adopcion = adopcion;
        }
        public int getID() {
            return ID;
        }

        @Override
        public boolean equals(Object obj) {
            if( this == obj ) return true;
            if(obj instanceof Animal ){
                Animal animal = (Animal) obj;
                return this.ID == animal.ID;
            }
            return false;
        }
        @Override
        public int hashCode() {
            return Integer.hashCode(ID);
        }
        @Override
        public String toString() {
            return String.format("Animal: ID=%d, nacimiento=%tF, estado=%s", ID, nacimiento, estadoAnimal);
        }
    }
\end{lstlisting}



\subsection{Conclusión}

El diseño e implementación del código de andamiaje para el sistema se realizó siguiendo 
los principios fundamentales del diseño orientado a objetos, adaptados a los requerimientos 
específicos de este apartado. Se tomaron la decisiones de diseño adecuadas, como la gestión 
de asociaciones entre clases, la encapsulación de datos y la validación restricciones con \texttt{assert}, 
proporcionando un modelo consistente y flexible.\par
\vspace{0.15cm}
Una de las decisiones clave fue el uso combinado de asociaciones directas para relaciones 
simples y la reificación de asociaciones para relaciones más complejas junto con \texttt{get}
con conjuntos inmutables. Esto permitió mantener un equilibrio entre la simplicidad de las 
implementaciones directas, como la gestión de animales en el refugio, y la flexibilidad 
de las relaciones complejas, como las adopciones, donde se requieren atributos adicionales 
y validaciones específicas mientras protegíamos las listas de cada objeto en el sistema.\par
\vspace{0.15cm}
Además, la bidireccionalidad en relaciones como las adopciones, garantizó la 
consistencia del modelo al sincronizar automáticamente los datos entre entidades 
relacionadas.\par
TODO: AÑADIR COMO RESUMEN LAS DESICIONES QUE NO SE HAYANN  MENCIONADO DE LO QUE DIEGO A RECOPILADO EN LLAMADA

\newpage

\section{Apartado A}
\subsection{Diseño del Código de Andamiaje}
\subsection*{Introducción}
El concepto de \enquote{código de andamiaje} en el diseño orientado a objetos, 
particularmente en Java, hace referencia al conjunto de estructuras y métodos 
necesarios para implementar asociaciones entre clases, asegurando la 
consistencia y la integridad del sistema.\par
\vspace{0.15cm}
Su propósito es proporcionar un marco inicial sobre el que los desarrolladores pueden 
construir las funcionalidades particulares de un proyecto. Sin embargo, la línea entre
\enquote{andamiaje} e \enquote{implementación completa} puede ser fina ya que estamos 
añadiendo nuevas funcionalidades que luego se convierten en el nuevo marco inicial para 
futuros cambios que vayamos a realizar en los próximos apartados. A continuación se 
exponen las decisiones diseño que completan el andamiaje inicial.

\subsection{Análisis de opciones de Diseño}
En esta sección se exponen las diferentes opciones de diseño para la implementación 
del sistema de gestión de un refugio de animales, conforme al modelo de clases y 
operaciones proporcionado.\par
\vspace{0.15cm}
El sistema requiere gestionar los socios del refugio (quienes pueden desempeñar 
diferentes roles como voluntarios, donantes y adoptantes), así como el registro, 
adopción y donación de animales. Además, se deben considerar las relaciones entre 
las entidades (socios, animales, refugio, donaciones) y las restricciones del sistema, 
como la consistencia en los datos y las operaciones.\par
\vspace{0.15cm}
Importante mencionar que en nuestro diseño, \textbf{no hemos aplicado un único enfoque de manera 
exclusiva}. Hemos adoptado una combinación de estrategias dependiendo de las necesidades de 
cada relación dentro del sistema justificándolo de forma adecuada.

\subsubsection{Manejo de las Asociaciones}

\begin{description}
    \item[a)] \textbf{Asociación Directa (Sin Reificación\footnote{
        La reificación es una técnica en programación 
        orientada a objetos que se basa en convertir un 
        concepto abstracto, como una relación, en una 
        entidad concreta o clase.})}
\end{description}

\textit{\textbf{Descripción:}}  
En este enfoque, las asociaciones entre clases se implementan directamente como atributos 
en las clases relacionadas.\par
\vspace{0.15cm}
Esta práctica es fácil de implementar porque la cantidad de clases a gestionar 
y el número de clases necesarias para representar las relaciones es menor. No obstante, 
añadir atributos adicionales a las asociaciones (como fechas en el proceso 
de adopción) puede traer problemas de consistencia al manejar relaciones complejas como la 
de un socio con múltiples roles (lo exploraremos en futuras secciones).\par
\newpage % Para que no se ve aglomerada la definición en la footnote
\textbf{Ejemplo: Implementación de \texttt{Refugio} con asociación directa a \texttt{Animal}}\par
La asociación es directa porque \texttt{Refugio} gestiona los \texttt{Animales} mediante un 
\texttt{set}, sin una clase intermedia que relacione ambas entidades.(ver Código~\ref{codigo:refugio})\par
TODO: CAMBIAR EL CODE\par
En nuestra implementación, hemos decidido la utilización de un \texttt{Set} para las estructuras de datos 
en vez de \texttt{List} por varios factores:

\begin{itemize}
    \item \textbf{Control de elementos repetidos:} Los \texttt{Set} usan los métodos de \texttt{hashCode} e \texttt{equals}
    para hacer la comprobación de la existencia de elementos en la colección. Si un nuevo elemento coincide con uno existente, 
    no se inserta, evitando comprobaciones adicionales que haríamos con el uso de \texttt{List}.
    \item \textbf{Complejidad algorítmica:} En los \texttt{HashSet}, la búsqueda y la inserción tienen una complejidad de \(O(1)\), 
    ya que se basan en tablas hash. Por otro lado, en una \texttt{List} (como \texttt{ArrayList}), las operaciones de búsqueda tienen una 
    complejidad de \(O(n)\), porque requiere iterar sobre los elementos.
    \item \textbf{Orden de los elementos:} En nuestra implementación, no es necesario mantener un orden específico en las colecciones. 
    Por esta razón, el uso \texttt{Set} es más adecuado que \texttt{List} teniendo en cuenta los puntos anteriores.
\end{itemize}
\vspace{0.45cm}

\begin{description}
    \item[b)] \textbf{Reificación de la Asociación (Clase de Asociación)}
\end{description}

\textit{\textbf{Descripción:}}  
En este enfoque, las asociaciones complejas entre clases se modelan mediante clases 
intermedias. Por ejemplo, la clase \texttt{Adopcion} representa la relación entre un 
\texttt{Animal}, un \texttt{Adoptante}, y un \texttt{Voluntario}, incluyendo atributos 
como \texttt{fechaAdopcion} para capturar detalles específicos de la relación.\par
\vspace{0.15cm}
Como la reificación nos permiten agregar atributos y métodos específicos a las relaciones,
facilita la implementación de restricciones complejas relacionadas con la asociación.
Sin embargo, aumenta el número de clases y relaciones a gestionar, lo que hace el diseño más denso
ya que debemos implementar y gestionar las clases de asociación, así como los métodos para 
acceder a las relaciones.\par
\vspace{0.15cm}

\textbf{Nuestra Implementación: Uso de \texttt{Adopcion} como clase de asociación:}
\begin{itemize}
    \item Se representa la relación entre \texttt{Animal}, \texttt{Adoptante} y 
    \texttt{Voluntario} mediante una clase intermedia. (ver Código~\ref{codigo:adopcion})
    \item Atributos como \texttt{fecha} añaden flexibilidad al modelo, permitiendo 
    capturar detalles adicionales de la relación.
    \item Se gestionan las relaciones bidireccionales entre las entidades involucradas, 
    asegurando consistencia en los datos.
\end{itemize}

Elegimos este enfoque, ya que proporciona la flexibilidad necesaria para agregar atributos 
y gestionar reglas de negocio específicas. La asociación directa fue descartada porque no 
permitiría capturar detalles adicionales, como la fecha de adopción, ni manejar eficientemente 
las restricciones relacionadas con el proceso de adopción.



\subsubsection{Manejo de Roles de los Socios}

\begin{description}
    \item[a)] \textbf{Subclases Específicas para cada Rol}
\end{description}

\textit{\textbf{Descripción:}}  
Cada rol (\texttt{Voluntario}, \texttt{Donante}, \texttt{Adoptante}) se implementa como una 
subclase de la clase \texttt{Socio}. Esto permite encapsular los atributos y métodos 
específicos de cada rol dentro de su respectiva subclase.
\vspace{0.15cm}


Esto proporciona claridad al diseño, ya que cada rol está claramente representado con 
métodos específicos para su comportamiento. Además, permite encapsular los atributos y 
métodos particulares de cada tipo de socio, lo que mejora la organización y legibilidad 
del código. Aunque tiene una limitación significativa para manejar roles múltiples, ya que no permite que 
un socio asuma más de un rol sin duplicar instancias de las subclases. Esto hace que el 
diseño sea rígido y menos flexible en casos donde los roles pueden cambiar dinámicamente 
o coexistir (volveremos a hablar de esto en los siguientes apartados).\par
\vspace{0.15cm}



\textbf{Ejemplo: Subclases específicas para los roles}\par
El diseño tiene implementadas las subclases \texttt{Donante}, \texttt{Adoptante}, 
y \texttt{Voluntario} como extensiones de la clase \texttt{Socio}.\par
(ver Código~\ref{codigo:donante}, \ref{codigo:adoptante}, \ref{codigo:voluntario})

\begin{description}
    \item[b)] \textbf{Uso de Composición de Roles}
\end{description}

\textit{\textbf{Descripción:}}  
En lugar de modelar cada rol como una subclase de \texttt{Socio}, este enfoque utiliza 
la composición para permitir que un socio tenga múltiples roles simultáneamente. 
Cada rol se modela como una clase independiente que puede ser asociada dinámicamente a 
un \texttt{Socio} mediante una colección de roles.
\vspace{0.15cm}


Este enfoque es mucho más flexible, ya que permite asignar múltiples roles a un socio 
sin necesidad de crear combinaciones de subclases. También simplifica el manejo de 
roles dinámicos y permite cambios en tiempo de ejecución.
Puede reducir la claridad del diseño, ya que no existe una distinción explícita entre 
los diferentes tipos de socios. Además, requiere implementar lógica adicional para 
validar qué operaciones son aplicables para los roles asignados a cada socio.\par
\vspace{0.15cm}


\textbf{Nuestra implementación: Uso de Subclases Específicas para cada Rol:}\par  
\vspace{0.15cm}
Hemos decidido no cambiar como están implementados los roles mediante subclases específicas 
(para este apartado) en lugar de composición. Esto se debe a que en nuestro modelo actual, 
los roles están claramente definidos y no se requiere que un socio tenga múltiples roles de 
manera simultánea. Además:
\begin{itemize}
    \item La claridad y encapsulación que proporciona la herencia permiten manejar las responsabilidades y comportamientos específicos de cada tipo de socio de manera aislada.
    \item Aunque la composición sería más flexible, introducirá complejidad adicional innecesaria para los requisitos actuales del sistema.
\end{itemize}

No obstante, los requisitos del sistema cambiarán en futuros apartados pidiendo que un socio
tenga múltiples roles simultáneamente. En el apartado correspondiente, se discute porque 
la composición sería una solución más adecuada y como se ha implementado.




\subsubsection{Consistencia y Gestión de Datos}

\begin{description}
    \item[a)] \textbf{Encapsulación Estricta}
\end{description}

\textit{\textbf{Descripción:}}  
Este enfoque restringe el acceso directo a los atributos y métodos de las clases mediante 
el uso de visibilidad privada. Para interactuar con los atributos, se proporcionan métodos 
públicos controlados (\texttt{getters} y \texttt{setters}) que incluyen validaciones (mediante 
\texttt{asserts}) para garantizar que los datos se mantengan en un estado consistente.
\vspace{0.15cm}

    Debido a esto nos aseguramos que los datos sean modificados de manera controlada y consistente.
    Facilita la incorporación de validaciones o pruebas unitarias lo que completa el comportamiento del
    esperado del sistema.Dicho esto también se requiere de implementar más métodos, como \texttt{getters}, \texttt{setters} y validaciones 
    necesarias, lo que aumenta la cantidad de código. Además, estas validaciones podrían hacer 
    que el diseño sea más extenso y menos directo.\par
    \vspace{0.15cm}


\textbf{Ejemplo en el sistema: Uso de encapsulación estricta en la clase \texttt{Animal}:}\par  
En nuestro sistema, la clase \texttt{Animal} utiliza atributos privados y métodos públicos 
controlados para garantizar consistencia y validaciones en tiempo de ejecución.Este enfoque 
asegura que cualquier intento de modificar el estado de un \texttt{Animal} pase por 
validaciones definidas en los métodos públicos. (ver Código~\ref{codigo:animal})


\begin{description}
    \item[b)] \textbf{Uso de Colecciones Inmutables}
\end{description}

\textit{\textbf{Descripción:}}  
En este enfoque, las colecciones utilizadas para representar relaciones (por ejemplo, 
listas o conjuntos de \texttt{Animal} en \texttt{Refugio}) son inmutables. Esto garantiza 
que las relaciones no puedan ser modificadas accidentalmente fuera de las clases que las 
gestionan.
\vspace{0.15cm}

    Mejora la integridad del sistema al garantizar que las relaciones no se modifiquen de 
    manera no controlada.
    Por otro lado introduce rigidez ya que no permite realizar cambios dinámicos en las relaciones sin 
    reemplazar completamente la colección. Esto puede dificultar la gestión de operaciones 
    como agregar o eliminar elementos.\par
    \vspace{0.15cm}

\textbf{Ejemplo de Uso de Colecciones Inmutables en la Clase \texttt{Refugio}:}\par  
En nuestro sistema, el método \texttt{getAnimalesRegistrados} de la clase \texttt{Refugio} 
devuelve una vista inmutable de los animales registrados. Esto asegura que las listas no 
puedan modificarse desde fuera de la clase. Además, el uso de \texttt{Collections.}\texttt{enumeration} 
garantiza que la colección de animales no pueda ser alterada fuera de la clase \texttt{Refugio}, 
manteniendo la consistencia de los datos. (ver Código~\ref{codigo:refugio} aunque se implementa en varias clases)

\textbf{Decisión Tomada: Encapsulación Controlada con Enumerations:}\par
En nuestro diseño, optamos por una encapsulación controlada en lugar de colecciones 
completamente inmutables. Esto se debe a que:
\begin{itemize}
    \item Proporciona flexibilidad para realizar cambios dinámicos en las colecciones a 
    través de métodos controlados, lo que es necesario para operaciones como agregar o 
    eliminar animales en un refugio.
    \item Utilizar enumeraciones en los métodos \texttt{get} garantiza que las colecciones 
    no se modifiquen desde fuera de las clases, preservando la integridad de los datos.
\end{itemize}

Este enfoque combina lo mejor de ambos mundos: flexibilidad para realizar cambios controlados 
y protección contra modificaciones accidentales.



\subsubsection{Estrategias para Manejo de Adopciones y Donaciones}

\begin{description}
    \item[a)] \textbf{Operaciones Independientes}
\end{description}

\textit{\textbf{Descripción:}}  
En este enfoque, cada operación (como adopción, registro de animales o donaciones) se 
implementa de forma independiente, sin interacciones entre ellas. Cada acción tiene su 
propio método o flujo lógico separado.
\vspace{0.15cm}

    Este diseño asegura que las operaciones están bien definidas y separadas, lo que 
    facilita su comprensión. Además, la simplicidad del diseño permite que sea directo y 
    fácil de implementar.Sin Embargo, puede llevar a la duplicación de código si varias operaciones comparten lógica similar 
    (por ejemplo, validar la existencia de un animal o donante). También puede ser menos 
    flexible, ya que cualquier cambio en una operación podría requerir modificaciones en 
    múltiples partes del sistema.\par
    \vspace{0.15cm}

\textbf{Ejemplo: Operaciones independientes en nuestra implementación}  
En nuestro diseño, las adopciones y donaciones se gestionan mediante clases específicas 
(\texttt{Adopcion} y \texttt{Donacion}), cada una con su propia lógica y atributos.

\begin{description}
    \item[b)] \textbf{Reutilización de Lógica Compartida entre Operaciones}
\end{description}

\textit{\textbf{Descripción:}}  
En lugar de mantener las operaciones completamente separadas, este enfoque identifica y 
reutiliza lógica común entre las operaciones (como validaciones o actualizaciones de estado). 
Aunque no implementamos este enfoque en nuestra solución actual, sería posible centralizar las 
validaciones comunes mediante una clase auxiliar, como se muestra en el siguiente ejemplo.\par
\vspace{0.15cm}
Esta opción reduce la duplicación de código, ya que la lógica compartida se implementa una sola vez
y facilita la incorporación de nuevas funcionalidades relacionadas con las operaciones existentes. Pero 
puede introducir una dependencia más estrecha entre las clases, lo que podría aumentar 
la complejidad del sistema en caso de cambios importantes.

\textbf{Ejemplo Propuesto: Centralización de Validaciones}  
Aunque nuestra implementación actual gestiona las validaciones directamente en los métodos 
de las clases relevantes (\texttt{Adoptante}, \texttt{Donante}). Por ejemplo, podríamos considerar una 
clase auxiliar para centralizarlas en el futuro:

\begin{lstlisting}[style = javaEspecifico, language=Java, caption={Clase Auxiliar para Validaciones}] 
public class Validacion {
    public static void validarEstadoAnimal(Animal animal, EstadoAnimal estadoEsperado) {
        assert animal.getEstadoAnimal() == estadoEsperado : 
            "El estado del animal no coincide con el esperado.";
    }
}
\end{lstlisting}

En nuestra implementación actual, la validación se realiza directamente dentro de las clases:

\begin{lstlisting}[style = javaEspecifico, language=Java, caption={Manejo de validaciones dentro del método Adoptar}] 
public void adoptar(Animal a, Voluntario v) {
    assert a.getEstadoAnimal() == EstadoAnimal.DISPONIBLE : 
        "El animal no esta disponible.";
    Adopcion adopcion = new Adopcion(a, this, v, new Date());
    adopciones.add(adopcion);
}
\end{lstlisting}
\vspace{0.15cm}
\textbf{Decisión Tomada: Mantener las Validaciones en las Clases Relevantes}\par
En nuestra implementación actual, las validaciones se realizan directamente en las clases 
donde ocurren las operaciones. Esto se alinea con la claridad y simplicidad requeridas por 
el sistema. Sin embargo, reconocemos que la centralización de lógica compartida podría ser 
útil en sistemas más complejos. FIXME: EL GETTER ES MAL EJEMPLO DE ESTO (ver los getters en Código~\ref{codigo:adopcion} por ejemplo) 



\subsubsection{Representación de Relaciones en el Sistema}

\begin{description}
    \item[a)] \textbf{Relaciones Unidireccionales}
\end{description}

\textit{\textbf{Descripción:}}  
En una relación unidireccional, solo una entidad tiene conocimiento de la relación. 
Por ejemplo, un \texttt{Adoptante} puede conocer al \texttt{Animal} que adopta, 
pero el \texttt{Animal} no necesita saber nada sobre el \texttt{Adoptante}.\par
\vspace{0.15cm} 
Estas relaciones gestionan la relación con una clase, lo que reduce la complejidad del sistema.
El problema, es que limita las consultas entre clases relacionadas y puede volverse más complejo 
añadir funcionalidades.\par
\vspace{0.15cm}
En nuestro sistema, todas las relaciones unidireccionales con 1 a muchos, por ejemplo:
TODO: EXPANDIR ESTA IDEA

\begin{description}
    \item[b)] \textbf{Relaciones Bidireccionales}
\end{description}

\textit{\textbf{Descripción:}}  
En una relación bidireccional, ambas entidades conocen y mantienen referencias mutuas. 
Por ejemplo, cuando un \texttt{Adoptante} adopta un \texttt{Animal}, ambos se actualizan 
mutuamente para reflejar la relación.\par
\vspace{0.15cm}
Las relaciones de este estilo garantizan la consistencia de los datos, ya que ambas partes relacionadas están 
sincronizadas al mantener referencias mutuas explícitas. Sin embargo, como hay que tener una sincronización constante 
entre las todas las clases relacionadas, si tuviéramos muchas relaciones bidireccionales puede dificultar el mantenimiento
por que genera un alto nivel de acoplamiento.\par
\vspace{0.15cm}
\textbf{Ejemplo en el Sistema: Relaciones Bidireccionales en \texttt{Adopcion}}\par 
En nuestro diseño, la relación entre \texttt{Animal}, \texttt{Adoptante}, y 
\texttt{Voluntario} es bidireccional y se asegurando consistencia en ambas direcciones
reflejando los cambios realizados en una clase en las demás involucradas.
(ver Código~\ref{codigo:adopcion} como, por ejemplo, se actualiza el estado del animal tras ser adoptado)\par
\vspace{0.15cm}
\textbf{Decisión Tomada: Relaciones Bidireccionales}\par
Hemos implementado relaciones bidireccionales para las asociaciones complejas del sistema, 
como las adopciones, ya que garantizan consistencia y sincronización entre las entidades 
relacionadas. Sin embargo, para relaciones más simples, como la lista de animales en un 
refugio, usamos relaciones unidireccionales para mantener la simplicidad.

\subsection{Diagrama de Diseño}

\begin{figure}[H]
    \centering
     \includegraphics[width=1\linewidth]{assets/umaLogo.png}
     \caption{UMA}
\end{figure}
FALTA HACER Y METER EL DIAGRAMA DE DISEÑO DE NUESTRO SISTEMA.RECOMENDABLE PONER UNA MINI EXPLICACIÓN.

\newpage



\subsection{Implementación del Modelo}
\subsubsection{Clase Socio}\label{codigo:socio}
La clase \texttt{Socio} es abstracta y representa la base para las distintas subclases: 
\texttt{Adoptante}, \texttt{Voluntario}, y \texttt{Donante}. Esta clase asegura que 
cada socio tenga un \texttt{ID} único, una fecha de registro válida y un refugio asociado.

\begin{lstlisting}[style = javaNormal, language=Java] 
    package sistema;

    import java.util.Collections;
    import java.util.Date;
    
    public abstract class Socio {
        private int ID;
        private Date registro;
        private final Refugio refugioAsociado;
    
        public Socio(int ID, Date fechaRegistro, Refugio refugioAsociado) {
            assert ID > 0 : "El ID del socio debe ser valido.";
    
            assert fechaRegistro != null : "La fecha de registro no puede ser nula.";
    
            assert refugioAsociado != null : "El refugio asociado no puede ser nulo.";
    
            this.ID = ID;
            this.registro = fechaRegistro;
            this.refugioAsociado = refugioAsociado;
            refugioAsociado.addSocio(this);
    
            assert refugioAsociado.getSocios().hasMoreElements() && Collections.list(refugioAsociado.getSocios()).contains(this);
        }
    
        public int getID() {
            return this.ID;
        }
    
        private void setID(int ID) {
            assert ID > 0;
    
            assert Collections.list(this.getRefugio().getSocios()).stream().noneMatch((s) -> {
                return s.getID() == ID;
            }) : "El ID ya existe en el refugio";
    
            this.ID = ID;
        }
    
        public Date getRegistro() {
            return this.registro;
        }
    
        public void setRegistro(Date fechaRegistro) {
            assert fechaRegistro != null;
    
            this.registro = fechaRegistro;
        }
    
        public Refugio getRefugio() {
            return this.refugioAsociado;
        }
    
        public boolean equals(Object obj) {
            if (this == obj) {
                return true;
            } else if (obj instanceof Socio) {
                Socio socio = (Socio)obj;
                return this.ID == socio.ID;
            } else {
                return false;
            }
        }
    
        public int hashCode() {
            return Integer.hashCode(this.ID);
        }
    }
\end{lstlisting}



\subsubsection{Clase Donante}\label{codigo:donante}
La clase \texttt{Donante} extiende de \texttt{Socio} y gestiona las donaciones realizadas 
por un socio. Las donaciones se almacenan en un \texttt{HashSet} para asegurar que no hayan elementos repetidos y por eficiencia. (cómo se explicó en \ref{page:Consideraciones}) 
Por otro lado, hemos decidido implementar al crear un nuevo objeto Donante en el sistema,
el constructor llamará directamente al método donar ya que es una condición necesaria para ser donante. Las donaciones que sean modificadas son tratadas desde la clase Donación y
el \texttt{HashSet donaciones} se actualizará.


\begin{lstlisting}[style = javaNormal, language=Java] 
    package sistema;

    import java.time.LocalDate;
    import java.time.ZoneId;
    import java.util.*;
    
    public class Donante extends Socio{
        private Set<Donacion> donaciones;
        public Donante(int ID, Date date,Refugio r, float cantidad) {
            super(ID,date,r);
            assert cantidad > 0 : "La cantidad inicial donada debe ser mayor a cero.";
            donaciones = new HashSet<>();
            donar(cantidad);
        }

        public void donar(float cantidad){
            assert cantidad > 0 : "La cantidad donada debe ser mayor a cero.";
            assert Collections.list(this.getRefugio().getSocios()).contains(this): "El socio debe ser donante antes de poder donar";
            Donacion d = new Donacion(cantidad, Date.from(fechaDonacion.atStartOfDay(ZoneId.systemDefault()).toInstant()));
            addDonacion(d);
            Refugio r = super.getRefugio();
            r.setLiquidez(r.getLiquidez() + cantidad);
            assert donaciones.contains(d);
        }
    
        protected void addDonacion(Donacion donacion){
            assert donacion != null: "La donacion no puede ser nula";
            donaciones.add(donacion);
        }
    
        protected void removeDonacion(Donacion donacion){
            assert donacion != null : "La donacion no puede ser nula.";
            if (donaciones.contains(donacion) && donaciones.size() > 1) {
                donaciones.remove(donacion);
            } else if (donaciones.contains(donacion) && donaciones.size() == 1) {
                System.out.println("Todo donante debe tener al menos una donacion, estas intentando eliminar la unica donacion asociada a este socio donante");
            } else {
                System.out.println("Este socio no ha realizado la donacion que intentas eliminar");
            }
        }
        public Enumeration<Donacion> getDonaciones(){
            return Collections.enumeration(this.donaciones);
        }
    
        @Override
        public String toString() {
            return "Donante " + super.getID();
        }
    }
    
\end{lstlisting}

\subsubsection{Clase Adoptante}\label{codigo:adoptante}
La clase \texttt{Adoptante} extiende de \texttt{Socio} y simula las adopciones realizadas 
por un adoptante. Las adopciones se almacenan en un \texttt{HashSet} para evitar adopciones duplicadas.
\begin{lstlisting}[style = javaNormal, language=Java] 
    package sistema;

    import java.util.*;
    
    public class Adoptante extends Socio {
        private Set<Adopcion> adopciones;
    
        public Adoptante(int ID, Date date, Refugio r) {
            super(ID, date,r);
            adopciones = new HashSet<>();
        }
        public void adoptar(Animal a, Voluntario v) {
            assert a.getEstadoAnimal() == EstadoAnimal.DISPONIBLE: "El animal ya ha sido adoptado";
            Refugio refugioDelVoluntario = v.getRefugio();
            a.setEstadoAnimal(EstadoAnimal.ADOPTADO);
            refugioDelVoluntario.removeAnimalesRefugiados(a);
            v.tramitarAdopcion(a, this);
        }
    
        protected void addAdopcion(Adopcion a){
            this.adopciones.add(a);
        }
    
        protected void removeAdopcion(Adopcion a){
            if (adopciones.contains(a)) adopciones.remove(a);
            else System.out.println("Este animal ya no esta asociado al adoptante");
        }
        public Enumeration<Adopcion> getAdopciones(){
            return Collections.enumeration(adopciones);
        }
        @Override
        public boolean equals(Object obj) {
            if (this == obj) return true;
            if(obj instanceof Adoptante ){
                Adoptante adoptante = (Adoptante) obj;
                return adoptante.getID() == this.getID();
            }
            return false;
        }
        @Override
        public int hashCode() {
            return Integer.hashCode(this.getID());
        }
    
        @Override
        public String toString() {
            return "Adoptante " + super.getID();
        } 
    }    
\end{lstlisting}

\subsubsection{Clase Voluntario}\label{codigo:voluntario}
La clase \texttt{Voluntario} extiende de \texttt{Socio} y gestiona los trámites de 
adopción realizados por un voluntario. Los voluntarios se almacenan en un \texttt{HashSet},
evitando así voluntarios duplicados. 
\begin{lstlisting}[style = javaNormal, language=Java] 
    package sistema;

    import java.time.LocalDate;
    import java.time.ZoneId;
    import java.util.*;
    
    public class Voluntario extends Socio{
        Set<Adopcion> tramites;
        
        public Voluntario(int ID, Date date,Refugio r) {
            super(ID, date,r);
            tramites = new HashSet<>();
        }
        public void tramitarAdopcion(Animal a, Adoptante ad){
            assert ad != null : "El adoptante no puede ser nulo.";
            LocalDate fechaAdopcion = LocalDate.now();
            Adopcion adopcion = new Adopcion(a, ad, this, Date.from(fechaAdopcion.atStartOfDay(ZoneId.systemDefault()).toInstant()));
            addTramite(adopcion);
            ad.addAdopcion(adopcion);
        }
        public void registrar(Animal a){
            Refugio r = super.getRefugio();
            assert r != null : "El refugio asociado no puede ser nulo.";
            assert a != null : "El animal no puede ser nulo.";
            a.setEstadoAnimal(EstadoAnimal.DISPONIBLE);
            r.addAnimalesRefugiados(a);
    
        }
        public Enumeration<Adopcion> getTramites(){
            return Collections.enumeration(tramites);
        }
        protected void addTramite(Adopcion ad){
            assert ad != null : "El tramite de adopcion no puede ser nulo.";
            tramites.add(ad);
    
        }
        protected void removeTramite(Adopcion ad){
            assert  ad != null: "El tramite de adopcion no puede ser nulo.";
            tramites.remove(ad);
        }
    
    
        @Override
        public String toString() {
            return "Voluntario " + super.getID();
        }
    }
    
\end{lstlisting}

\subsubsection{Clase Refugio}\label{codigo:refugio}
La clase \texttt{Refugio} gestiona el conjunto de \texttt{Socios} y \texttt{Animales}.\par
\textbf{liquidez} está declarado como un \texttt{float} por las razones que se exponen en \ref{page:Consideraciones}.


\begin{lstlisting}[style = javaNormal, language=Java] 
package sistema;
import java.util.*;

public class Refugio {
    private float liquidez;
    private Set<Animal> animalesRegistrados;
    private Set<Animal> animalesRefugiados;
    private Set<Socio> socios;

    public Refugio(float liquidez) {
        assert liquidez >= 0 : "La liquidez debe ser no negativa.";
        this.liquidez = liquidez;
        animalesRefugiados = new HashSet<>();
        animalesRegistrados = new HashSet<>();
        socios = new HashSet<>();
    }

    public float getLiquidez() 
        return liquidez;
    }
    public void setLiquidez(float liquidez) {
        assert liquidez >= 0 : "La liquidez debe ser no negativa";
        this.liquidez = liquidez;
    }
    protected void addSocio(Socio s) {
        assert s != null : "El socio no puede ser nulo.";
        if(socios.contains(s)) {
            System.out.println("El socio ya esta registrado.");
            return;
        }
        socios.add(s);
    }
    protected void removeSocio(Socio s) {
        assert s != null : "El socio no puede ser nulo.";
        if (socios.contains(s)) {
            socios.remove(s);
        } else {
            System.out.println("Este socio no esta registrado en el refugio.");
        }
    }

    public Enumeration<Animal> getAnimalesRegistrados() {
        return Collections.enumeration(animalesRegistrados);
    }
    public Enumeration<Animal> getAnimalesRefugiados() {
        return Collections.enumeration(animalesRefugiados);
    }
    public Enumeration<Socio> getSocios() {
        return Collections.enumeration(socios);
    }

    public List<Adoptante> getAdoptantes() {
        List<Adoptante> adoptantes = new ArrayList<>();
        for (Socio s : socios) {
            if (s instanceof Adoptante) {
                adoptantes.add((Adoptante) s);
            }
        }
        return adoptantes;
    }
    public List<Voluntario> getVoluntarios() {
        List<Voluntario> voluntarios = new ArrayList<>();
        for (Socio s : socios) {
            if (s instanceof Voluntario) {
                voluntarios.add((Voluntario) s);
            }
        }
        return voluntarios;
    }
    public List<Donante> getDonantes() {
        List<Donante> donantes = new ArrayList<>();
        for (Socio s : socios) {
            if (s instanceof Donante) {
                donantes.add((Donante) s);
            }
        }
        return donantes;
    }
    public void registrar(Animal a){
        this.addAnimalesRegistrados(a);
    }
    protected void addAnimalesRefugiados(Animal a){
        assert a != null : "El animal no puede ser nulo.";
        if(!animalesRefugiados.contains(a)){
            animalesRefugiados.add(a);
            this.addAnimalesRegistrados(a);
        } else System.out.println("Este animal ya esta en el refugio.");
    }
    private void addAnimalesRegistrados(Animal a){
        assert a != null : "El animal no puede ser nulo.";
        if (!animalesRegistrados.contains(a)) {
            animalesRegistrados.add(a);
        } else {
            System.out.println("El animal ya esta registrado.");
        }
    }

    protected void removeAnimalesRefugiados(Animal a){
        assert a != null : "El animal no puede ser nulo.";
        if (animalesRefugiados.contains(a)) {
            animalesRefugiados.remove(a);
        } else {
            System.out.println("El animal no se encuentra en este Refugio.");
        }
    }
    protected void removeAnimalesRegistrados(Animal a){
        assert a != null : "El animal no puede ser nulo.";
        if (animalesRegistrados.contains(a) && animalesRegistrados.size() > 1) {
            animalesRegistrados.remove(a);
        } else if (animalesRegistrados.contains(a) && animalesRegistrados.size() == 1) {
            System.out.println("Todo refugio debe tener al menos un animal registrado, estas intentando eliminar el unico animal existente.");
        } else {
            System.out.println("El animal no se encuentra en este Refugio.");
        }
    }

    public void mostrarAnimalesRefugiados(){
        System.out.println(animalesRefugiados.toString());
    }
    public void mostrarAnimalesRegistrados(){
        System.out.println(animalesRegistrados.toString());
    }
    public void mostrarSocios() {
        for (Socio s : socios) {
            System.out.println(s);
        }
    }
    public void mostrarSociosPorTipo() {
        System.out.println("Adoptantes: " + getAdoptantes());
        System.out.println("Voluntarios: " + getVoluntarios());
        System.out.println("Donantes: " + getDonantes());
    }

    @Override
    public String toString() {
        StringBuilder sb = new StringBuilder();
        sb.append("Animales Registrados: ").append(animalesRegistrados).append("\n");
        sb.append("Animales Refugiados: ").append(animalesRefugiados).append("\n");
        sb.append("Socios: ").append(socios).append("\n");
        sb.append("Liquidez: ").append(liquidez);
        return sb.toString();
    }
}
\end{lstlisting}



\subsubsection{Clase Donacion}\label{codigo:donacion}
La clase \texttt{Donacion} representa una donación realizada por un \texttt{Donante}. 
Incluye la cantidad que como anteriormente mencionamos en \ref{page:Consideraciones} por temas de eficiencia es un \texttt{float},
la fecha de la donación y el donante asociado. Las validaciones 
aseguran que los valores sean válidos en el momento de la creación de la instancia.

\begin{lstlisting}[style = javaNormal, language=Java] 
    package sistema;

    import java.util.Date;
    import java.util.Objects;
    
    public class Donacion {
        private float cantidad;
        private Date date;
    
        public Donacion(float cantidad, Date date) {
            assert cantidad > 0.0F : "La cantidad debe ser positiva.";
    
            assert date != null && !date.after(new Date()) : "La fecha no puede ser nula ni estar en el futuro.";
    
            this.cantidad = cantidad;
            this.date = date;
        }
    
        public float getCantidad() {
            assert this.cantidad > 0.0F : "La cantidad no puede ser nula.";
    
            return this.cantidad;
        }
    
        public void setCantidad(float cantidad) {
            this.cantidad = cantidad;
        }
    
        public Date getDate() {
            assert this.date != null : "La fecha no puede ser nula.";
    
            return this.date;
        }
    
        public void setDate(Date date) {
            this.date = date;
        }
    
        public String toString() {
            return String.format("Donacion: %.2f, %tY-%tB-%td", this.cantidad, this.date, this.date, this.date);
        }
    
        public boolean equals(Object o) {
            if (this == o) {
                return true;
            } else if (o != null && this.getClass() == o.getClass()) {
                Donacion donacion = (Donacion)o;
                return Float.compare(this.cantidad, donacion.cantidad) == 0 && Objects.equals(this.date, donacion.date);
            } else {
                return false;
            }
        }
    
        public int hashCode() {
            return Objects.hash(new Object[]{this.cantidad, this.date});
        }
    }
\end{lstlisting}



\subsubsection{Clase Adopcion}\label{codigo:adopcion}
La clase \texttt{Adopcion} modela una adopción de un \texttt{Animal} realizada por un 
\texttt{Adoptante}, gestionada por un \texttt{Voluntario}. Implementa la bidireccionalidad 
entre estas entidades para mantener consistencia en las asociaciones.

\begin{lstlisting}[style = javaNormal, language=Java] 
    package sistema;

    import java.util.Date;
    
    public class Adopcion {
        private Date fecha;
        private final Animal animal;
        private final Adoptante adoptante;
        private final Voluntario voluntario;
    
        public Adopcion(Animal a, Adoptante ad, Voluntario v, Date fecha) {
            assert a != null : "El animal no puede ser nulo.";
    
            assert ad != null : "El adoptante no puede ser nulo.";
    
            assert v != null : "El voluntario no puede ser nulo.";
    
            assert fecha != null && !fecha.after(new Date()) : "La fecha no puede ser nula ni estar en el futuro.";
    
            this.animal = a;
            this.adoptante = ad;
            this.voluntario = v;
            this.fecha = fecha;
        }
    
        public Date getFecha() {
            return this.fecha;
        }
    
        public void setFecha(Date fecha) {
            assert fecha != null && !fecha.after(new Date()) : "La fecha no puede ser nula ni estar en el futuro";
    
            this.fecha = fecha;
        }
    
        public Animal getAnimal() {
            return this.animal;
        }
    
        public Voluntario getVoluntario() {
            return this.voluntario;
        }
    
        public Adoptante getAdoptante() {
            return this.adoptante;
        }
    
        public boolean equals(Object obj) {
            if (this == obj) {
                return true;
            } else if (!(obj instanceof Adopcion)) {
                return false;
            } else {
                Adopcion adopcion = (Adopcion)obj;
                boolean ok = this.adoptante.equals(adopcion.adoptante) && this.animal.equals(adopcion.animal);
                return ok;
            }
        }
    
        public int hashCode() {
            return this.adoptante.hashCode() + this.animal.hashCode();
        }
    
        public String toString() {
            return String.format("Adopcion: %tY-%tB-%td, %s, %s", this.fecha, this.fecha, this.fecha, this.animal, this.adoptante);
        }
    }
    
\end{lstlisting}



\subsubsection{Clase Animal}\label{codigo:animal}
La clase \texttt{Animal} modela a un animal registrado en el sistema. Cada animal tiene un 
ID único, una fecha de nacimiento, un estado actual y está asociado a un \texttt{Refugio}.\par

\begin{lstlisting}[style = javaNormal, language=Java] 
    package sistema;
    import java.util.Date;
    
    public class Animal {
        private int ID;
        private Date nacimiento;
        private EstadoAnimal estadoAnimal; 
        private Adopcion adopcion;

        public Animal(int ID, Date nacimiento, EstadoAnimal estadoAnimal) {
            assert ID > 0 : "El ID del animal debe ser valido.";
            assert nacimiento != null : "La fecha de nacimiento no puede ser nula.";
            assert estadoAnimal != null : "El estado del animal debe estar definido.";

            this.ID = ID;
            this.nacimiento = nacimiento;
            this.estadoAnimal = estadoAnimal;
        }
    
        public EstadoAnimal getEstadoAnimal() {
            return estadoAnimal;
        }
        public void setEstadoAnimal(EstadoAnimal estadoAnimal) {
            assert estadoAnimal != null : "El estado del animal debe estar definido.";
            this.estadoAnimal = estadoAnimal;
        }
        public Date getNacimiento() {
            return nacimiento;
        }
        public void setNacimiento(Date nacimiento) {
            assert nacimiento != null : "La fecha de nacimiento no puede ser nula";
            this.nacimiento = nacimiento;
        }
        public Adopcion getAdopcion() {
            return this.adopcion;
        }
        public void setAdopcion(Adopcion adopcion){
            assert  adopcion != null;
            this.adopcion = adopcion;
        }
        public int getID() {
            return ID;
        }

        @Override
        public boolean equals(Object obj) {
            if( this == obj ) return true;
            if(obj instanceof Animal ){
                Animal animal = (Animal) obj;
                return this.ID == animal.ID;
            }
            return false;
        }
        @Override
        public int hashCode() {
            return Integer.hashCode(ID);
        }
        @Override
        public String toString() {
            return String.format("Animal: ID=%d, nacimiento=%tF, estado=%s", ID, nacimiento, estadoAnimal);
        }
    }
\end{lstlisting}



\subsection{Conclusión}

El diseño e implementación del código de andamiaje para el sistema se realizó siguiendo 
los principios fundamentales del diseño orientado a objetos, adaptados a los requerimientos 
específicos de este apartado. Se tomaron la decisiones de diseño adecuadas, como la gestión 
de asociaciones entre clases, la encapsulación de datos y la validación restricciones con \texttt{assert}, 
proporcionando un modelo consistente y flexible.\par
\vspace{0.15cm}
Una de las decisiones clave fue el uso combinado de asociaciones directas para relaciones 
simples y la reificación de asociaciones para relaciones más complejas junto con \texttt{get}
con conjuntos inmutables. Esto permitió mantener un equilibrio entre la simplicidad de las 
implementaciones directas, como la gestión de animales en el refugio, y la flexibilidad 
de las relaciones complejas, como las adopciones, donde se requieren atributos adicionales 
y validaciones específicas mientras protegíamos las listas de cada objeto en el sistema.\par
\vspace{0.15cm}
Además, la bidireccionalidad en relaciones como las adopciones, garantizó la 
consistencia del modelo al sincronizar automáticamente los datos entre entidades 
relacionadas.\par
TODO: AÑADIR COMO RESUMEN LAS DESICIONES QUE NO SE HAYANN  MENCIONADO DE LO QUE DIEGO A RECOPILADO EN LLAMADA

\newpage


% BIBLIOGRAFÍA
% \newpage

% \addcontentsline{toc}{section}{Referencias}  % Agrega "Referencias" al índice
% \bibliographystyle{apalike}                  % ó elsarticle-num.bst
% \bibliography{citas.bib}                     % Nombre del archivo donde tenemos todas las referencias bibliográficas


\end{document}