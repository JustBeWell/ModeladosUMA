\section{Ejercicio 2}
\subsection*{Instrucción}
A menudo, los coches que la empresa de alquiler de coches pone a disposición de sus clientes se tienen que poner
fuera de servicio (por reparaciones, para pasar la ITV, etc.). Queremos representar esta situación en nuestro sistema
para que cuando los coches estén fuera de servicio no puedan ser alquilados aunque registraremos, si hay, un coche
que lo sustituye. Es por eso que nuestro sistema tendrá que proporcionar dos funcionalidades. La funcionalidad (1) poner 
un coche fuera de servicio (si ya está fuera de servicio o si está en servicio pero es sustituto de algún coche
que está fuera de servicio, la funcionalidad no tendrá ningún efecto).\par
\vspace{0.15cm}
Esta funcionalidad pondrá el coche fuera de servicio y registrará la fecha hasta la cual estará fuera de servicio y, si hay, buscará y registrará también un coche
sustituto (será cualquier coche del mismo modelo del coche que se pone fuera de servicio que esté asignado a la
misma oficina y que esté en servicio).\par
\vspace{0.15cm}
La funcionalidad (2) pone un coche que estaba fuera de servicio en servicio.\par
\vspace{0.15cm}
En concreto, \textbf{nos centraremos en implementar la funcionalidad (1)} añadiendo la operación
\texttt{takeOutOfService(backToService : date)} de la clase \texttt{Car}. No hace falta implementar la funcionalidad (2).

\subsection{Patrón de Diseño utilizado}


\subsection{Efectos sobre el Diagrama de Diseño}

\subsection{Implementación de \textit{takeOutOfService : date}}
\begin{lstlisting}[style = javaNormal, language=Java] 

    introducir codigo aqui

\end{lstlisting} % Código en el documento "codigosEj2"

\newpage
