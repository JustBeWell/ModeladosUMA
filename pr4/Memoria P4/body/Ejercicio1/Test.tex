\subsubsection*{Código del tester}
\begin{lstlisting}[style = javaNormal, language=Java] 
    // Crear cliente
    Customer customer2 = new Customer("87654321B", "Bob");
    // Crear modelos de coches
    Model modelA = new Model("Model A", 50);
    Model modelB = new Model("Model B", 70);
    // Crear oficinas de alquiler
    RentalOffice office1 = new RentalOffice("Office 1", 20);
    RentalOffice office2 = new RentalOffice("Office 2", 25);
    // Crear coches
    Car car1 = new Car("ABC-123", modelA, office1);
    Car car2 = new Car("XYZ-789", modelB, office2);
    // Crear fechas
    Date startDate1 = new GregorianCalendar(2024, Calendar.JANUARY, 1).getTime();
    Date endDate1 = new GregorianCalendar(2024, Calendar.JANUARY, 10).getTime();
    Date startDate2 = new GregorianCalendar(2024, Calendar.JANUARY, 5).getTime();
    Date endDate2 = new GregorianCalendar(2024, Calendar.JANUARY, 15).getTime();

    // Oficinas diferentes
    WebRental webRental1 = new WebRental(5, startDate1, endDate1, customer2, car1, office1);
    webRental1.setDeliveryRentalOffice(office2);

    // Oficinas iguales
    WebRental webRental2 = new WebRental(2, startDate1, endDate1, customer2, car2, office2);
    webRental2.setDeliveryRentalOffice(office2);

    // Usar nueva operacion implementada
    int rentalsWithDifferentOffices = customer2.numberOfRentalsWithDifferentOffices();

    System.out.println("Alquileres con oficinas diferentes: " + rentalsWithDifferentOffices);

    if (rentalsWithDifferentOffices == 1) {
        System.out.println("Correcto: El metodo devuelve el numero esperado.");
    } else {
        System.out.println("ERROR: El metodo devuelve un resultado incorrecto.");
    }
\end{lstlisting}


\subsubsection*{Output}

\begin{lstlisting}[style = javaNormal, language=Java] 
    [WebRental: Mon Jan 01 00:00:00 CET 2024 Wed Jan 10 00:00:00 CET 2024
    ; 2, WebRental: Mon Jan 01 00:00:00 CET 2024 Wed Jan 10 00:00:00 CET 2024
    ; 5]

    Alquileres con oficinas diferentes: 1
    Correcto: El metodo devuelve el numero esperado.
\end{lstlisting}

\vspace{1cm}

Los \texttt{toString()} usados para mostrar el Output fueron:


\begin{lstlisting}[style = javaEspecifico, language=Java] 
    // en Rental
    @Override
    public String toString() {
        return this.getClass().getName() + ": " + startDate + " " + endDate+"\n";
    }

    // en WebRental, que llama al de Rental mediante super
    @Override
    public String toString() {
        return super.toString() + " ; " + deliveryTime.toString();
    }
\end{lstlisting}