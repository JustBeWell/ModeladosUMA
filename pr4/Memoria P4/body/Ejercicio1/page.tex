\section{Ejercicio 1}
\subsection*{Instrucción}
Supongamos ahora que queremos definir una operación\par 
\texttt{numberOfRentalsWithDifferentOffices() : Integer} en la clase
\texttt{Customer} que devuelve el número de alquileres web que ha hecho un cliente self donde la oficina de recogida y de
entrega es diferente. En este momento del diseño del sistema todavía no sabemos qué estructura de datos
utilizaremos para guardar los alquileres que ha hecho/hace un cliente.

\subsection{Patrón de Diseño utilizado}
El objetivo de incorporar la nueva operación es introducir al sistema una forma de recuperar información específica acerca de alquileres web. 
Como todos los alquileres son realizados por clientes, necesitaríamos verificar cada uno y encontrar todos aquellos que cumplan que su alquiler 
además de ser mediante la página, cumplan que la oficina de recogida entrega sea diferente.\par
\vspace{0.15cm}
La descripción inicial del sistema no detalla que esta operación sea la única que hace falta o es la primera 
que se implementa referente a los alquileres. Es posible que en un futuro se necesite información de clientes que hayan hecho más de un alquiler 
o se necesite recuperar información de cuantas veces un coche ha sido alquilado.\par
\vspace{0.15cm}
Pensando en todos los casos que se puedan solicitar, es ineficiente simplemente hacer un bucle que recorra toda la colección y filtrar por 
la condición necesaria si el sistema puede escalar. Es por ello que el patrón \texttt{Iterador} es el más adecuado para implementar esta solución.\par
\vspace{0.15cm}
Este patrón de diseño nos permite acceder secuencialmente a los elementos de una colección, separando la lógica de iteración del \emph{cliente}\footnote{\textbf{¿Quiénes son los \emph{clientes}?} Este término no se refiere ni a las clases del sistema ni a los usuarios del programa, sino a las partes del código que utilizan el iterador para recorrer una colección. En este caso, el \emph{cliente} del iterador es el método \texttt{numberOfRentalsWithDifferentOffices()}.} y centralizándola 
en el propio iterador, sin mencionar que permite escalar con facilidad el sistema. Otra ventaja de usar esta patrón es que no expone la estructura interna 
de la colección y permite ser usada por varios \emph{clientes} simultáneamente.

\subsubsection*{Consideraciones}
Como estamos utilizando un \texttt{HashSet} para almacenar los alquileres, somos consientes que esta colección ya implementa el método \texttt{iterator()} de la 
interfaz \texttt{Iterable}. No sería correcto usar este iterator por las mismas razones recién comentadas; el iterador nativo de \texttt{HashSet} simplemente 
recorre los elementos uno por uno, sin ningún tipo de lógica adicional. Como queremos filtrar \texttt{Rentals} para recuperar los que cumplan una condición específica, 
tiene sentido implementar un iterador personalizado con este patrón que encapsule esa lógica en particular.

\newpage % Mostramos el diagrama en una nueva página

\subsection{Efectos sobre el Diagrama de Diseño}
\begin{figure}[H]
    \centering
     \includegraphics[width=0.70\linewidth]{assets/diagramas/UML_Apartado1.png}
     \caption{Diagrama de Diseño modificado para Ejercicio1}
\end{figure}
\vspace{0.50cm}
Hemos tenido que modificar el diseño original para poder implementar este patrón de diseño. Siguiendo la estructura presentada en los apuntes del Tema 6 en la página 108, tenemos:
\begin{itemize}
    \item \textbf{\texttt{Interfaz} RentalIterator:} Define las operaciones necesarias para recorrer una colección de elementos.
    \item \textbf{\texttt{ConcreteIterator} RentalSetIterator:} Contiene la lógica específica para moverse a través de la colección.
    \item \textbf{\texttt{ConcreteCollection } Set$<$Rental$>$ rentals:} Almacena los elementos y devuelve un iterador para recorrerlos.
\end{itemize}
\vspace{0.15cm}
Importante mencionar que en el diseño presentado en los apuntes del Tema 6, la estructura del patrón \texttt{Iterator} incluye una interfaz adicional, 
\texttt{IterableCollection}, que establece una relación entre las clases que desean ser iterables. En nuestra implementación, decidimos no introducir 
esta interfaz adicional porque únicamente la clase \texttt{Customer} requiere un iterador que resuelva la restricción de búsqueda en el método que se presenta 
en el siguiente apartado.

\newpage % Mostramos la implementación en una nueva página

\subsection{Implementación de \textit{numberOfRentalsWithDifferentOffices() : Integer}}
\begin{lstlisting}[style = javaNormal, language=Java] 

    public Integer numberOfRentalsWithDifferentOffices() {
        int count = 0;
        RentalIterator iterator = this.getRentalIterator();

        while (iterator.hasNext()) {
            Rental rental = iterator.next();
            if (rental instanceof WebRental) {
                WebRental webRental = (WebRental) rental;
                if (!webRental.getPickUpRentalOffice().equals(webRental.getDeliveryRentalOffice())) {
                    count++;
                }
            }
        }

        return count;
    }

    public RentalIterator getRentalIterator() {
        return new RentalSetIterator(this.rentals);
    }

\end{lstlisting}
\vspace{1cm} % Código en el documento "codigosEj1"

Con estos dos métodos habilitamos el uso del patrón de diseño \texttt{Iterator}.\par
\texttt{numberOfRentalsWithDifferentOffices()} realiza el recorrido de la colección de alquileres asociados a un cliente y 
calcula cuántos de ellos cumplen con la condición específica de que sean alquileres de \texttt{WebRental} y las oficinas 
de recogida y entrega sean diferentes. Como se explicó antes, \texttt{getRentalIterator()} encapsula la creación del iterador específico (\texttt{RentalSetIterator}) 
que permite recorrer la colección de alquileres sin exponer su estructura interna.

\newpage % Muestra el test

\subsection{Ejemplo de ejecución}
\subsubsection*{Código del tester}
\begin{lstlisting}[style = javaNormal, language=Java] 
    // Crear cliente
    Customer customer2 = new Customer("87654321B", "Bob");
    // Crear modelos de coches
    Model modelA = new Model("Model A", 50);
    Model modelB = new Model("Model B", 70);
    // Crear oficinas de alquiler
    RentalOffice office1 = new RentalOffice("Office 1", 20);
    RentalOffice office2 = new RentalOffice("Office 2", 25);
    // Crear coches
    Car car1 = new Car("ABC-123", modelA, office1);
    Car car2 = new Car("XYZ-789", modelB, office2);
    // Crear fechas
    Date startDate1 = new GregorianCalendar(2024, Calendar.JANUARY, 1).getTime();
    Date endDate1 = new GregorianCalendar(2024, Calendar.JANUARY, 10).getTime();
    Date startDate2 = new GregorianCalendar(2024, Calendar.JANUARY, 5).getTime();
    Date endDate2 = new GregorianCalendar(2024, Calendar.JANUARY, 15).getTime();

    // Oficinas diferentes
    WebRental webRental1 = new WebRental(5, startDate1, endDate1, customer2, car1, office1);
    webRental1.setDeliveryRentalOffice(office2);

    // Oficinas iguales
    WebRental webRental2 = new WebRental(2, startDate1, endDate1, customer2, car2, office2);
    webRental2.setDeliveryRentalOffice(office2);

    // Usar nueva operacion implementada
    int rentalsWithDifferentOffices = customer2.numberOfRentalsWithDifferentOffices();

    System.out.println("Alquileres con oficinas diferentes: " + rentalsWithDifferentOffices);

    if (rentalsWithDifferentOffices == 1) {
        System.out.println("Correcto: El metodo devuelve el numero esperado.");
    } else {
        System.out.println("ERROR: El metodo devuelve un resultado incorrecto.");
    }
\end{lstlisting}


\subsubsection*{Output}

\begin{lstlisting}[style = javaNormal, language=Java] 
    [WebRental: Mon Jan 01 00:00:00 CET 2024 Wed Jan 10 00:00:00 CET 2024
    ; 2, WebRental: Mon Jan 01 00:00:00 CET 2024 Wed Jan 10 00:00:00 CET 2024
    ; 5]

    Alquileres con oficinas diferentes: 1
    Correcto: El metodo devuelve el numero esperado.
\end{lstlisting}

\vspace{1cm}

Los \texttt{toString()} usados para mostrar el Output fueron:


\begin{lstlisting}[style = javaEspecifico, language=Java] 
    // en Rental
    @Override
    public String toString() {
        return this.getClass().getName() + ": " + startDate + " " + endDate+"\n";
    }

    // en WebRental, que llama al de Rental mediante super
    @Override
    public String toString() {
        return super.toString() + " ; " + deliveryTime.toString();
    }
\end{lstlisting}

