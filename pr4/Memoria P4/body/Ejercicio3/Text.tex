\subsubsection*{Código del tester}
\begin{lstlisting}[style = javaNormal, language=Java] 
    // Crear modelos de coches
    Model modelA = new Model("Model A", 50);
    Model modelB = new Model("Model B", 70);

    // Crear oficinas de alquiler
    RentalOffice office1 = new RentalOffice("Office 1", 20);
    RentalOffice office2 = new RentalOffice("Office 2", 25);
    // Crear coches
    Car car1 = new Car("ABC-123", modelA, office1);
    Car car2 = new Car("XYZ-789", modelB, office2);

    // Crear clientes
    Customer customer1 = new Customer("12345678A", "Alice");

    // Alquiler con fechas solapadas
    Date startDate1 = new GregorianCalendar(2024, Calendar.JANUARY, 1).getTime();
    Date endDate1 = new GregorianCalendar(2024, Calendar.JANUARY, 10).getTime();
    Date startDate2 = new GregorianCalendar(2024, Calendar.JANUARY, 5).getTime(); // Solapado
    Date endDate2 = new GregorianCalendar(2024, Calendar.JANUARY, 15).getTime();
    DescuentoPorcentaje descPorcentaje = new DescuentoPorcentaje(50);
    DescuentoPromocion descPromocion = new DescuentoPromocion(100);
    Rental rental1 = new RentalOnSite("First rental", startDate1, endDate1, customer1, car1, office1,descPorcentaje);
    Rental rental2 = new RentalOnSite("Overlapping rental", startDate2, endDate2, customer1, car1, office1,descPromocion);

    // Mete alquileres al cliente
    customer1.addRental(rental1);
    customer1.addRental(rental2); // Esto viola la restriccion de solapamiento

    // Alquiler con fecha de inicio posterior a la fecha de finalizacion
    Date invalidStartDate = new GregorianCalendar(2024, Calendar.FEBRUARY, 10).getTime();
    Date invalidEndDate = new GregorianCalendar(2024, Calendar.FEBRUARY, 5).getTime(); // Invalido

    Rental invalidRental = new WebRental(11, invalidStartDate, invalidEndDate, customer1, car2, office2,null);

    // Alquiler web con hora de entrega despues de las 13:00
    Rental lateDeliveryRental = new WebRental(14, startDate1, endDate1, customer1, car2, office1,null); // Invalido

    // Mete mas alquileres
    customer1.addRental(invalidRental);
    customer1.addRental(lateDeliveryRental);

    // Imprimir resultados
    System.out.println("Customer Rentals:");
    System.out.println(customer1);

    System.out.println("\nOffice Rentals:");
    System.out.println(office1.toString());
    System.out.println(office2.toString());
    //Comprobar que los precios han cambiado en base a la promocion dada en el sistema
    System.out.println("----------------------------------------------------------------");
    System.out.println("Precio tras aplicar un descuento del " + rental1.getDescuento().toString() + " del tipo " + rental1.getDescuento().getClass().getName() + ": " + rental1.getPrice() + "\n");
    System.out.println("Precio tras aplicar un descuento del " + rental2.getDescuento().toString() + " del tipo " + rental2.getDescuento().getClass().getName() + ": " + rental2.getPrice() + "\n");
\end{lstlisting}


\subsubsection*{Output}

\begin{lstlisting}[style = javaNormal, language=Java] 
    Precio tras aplicar un descuento del [50.0] del tipo DescuentoPorcentaje: 225

    Precio tras aplicar un descuento del [100] del tipo DescuentoPromocion: 400
\end{lstlisting}

\vspace{1cm}

\newpage

Los \texttt{toString()} usados para mostrar el Output fueron:

\begin{lstlisting}[style = javaEspecifico, language=Java] 
    // en DescuentoPromocion
    @Override
    public String toString() {
        return "[" + descuento + "]";
    }

    // en DescuentoPorcentaje
    @Override
    public String toString() {
        return "[" + porcentaje + "]";
    }
\end{lstlisting}