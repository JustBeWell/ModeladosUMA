\subsubsection*{Código del tester}
\begin{lstlisting}[style = javaNormal, language=Java] 
    //Comprobar que los precios han cambiado en base a la promocion dada en el sistema
    System.out.println("Precio tras aplicar un descuento del " + rental1.getDescuento().toString() + " del tipo " + rental1.getDescuento().getClass().getName() + ": " + rental1.getPrice() + "\n");
    System.out.println("Precio tras aplicar un descuento del " + rental2.getDescuento().toString() + " del tipo " + rental2.getDescuento().getClass().getName() + ": " + rental2.getPrice() + "\n");
\end{lstlisting}


\subsubsection*{Output}

\begin{lstlisting}[style = javaNormal, language=Java] 
    Precio tras aplicar un descuento del [50.0] del tipo DescuentoPorcentaje: 225

    Precio tras aplicar un descuento del [100] del tipo DescuentoPromocion: 400
\end{lstlisting}

\vspace{1cm}

\newpage

Los \texttt{toString()} usados para mostrar el Output fueron:

\begin{lstlisting}[style = javaEspecifico, language=Java] 
    // en DescuentoPromocion
    @Override
    public String toString() {
        return "[" + descuento + "]";
    }

    // en DescuentoPorcentaje
    @Override
    public String toString() {
        return "[" + porcentaje + "]";
    }
\end{lstlisting}