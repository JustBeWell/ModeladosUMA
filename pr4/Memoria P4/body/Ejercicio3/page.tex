\section{Ejercicio 3}
\subsection*{Instrucción}
Cuando los clientes de la empresa de alquiler de coches alquilan un coche, el sistema que estamos construyendo
tiene que proporcionarles el precio del alquiler. Este precio lo calcula la operación \texttt{getPrice() : Integer} de la siguiente
forma: El precio base será el precio del modelo del vehículo por día.
\begin{center}
    \texttt{ (pricePerDay) * [endDate - startDate]}
\end{center}
Además, la empresa de alquiler de coches puede añadir al cálculo de precios la posibilidad de hacer promociones
que implican descuentos de estos precios. Inicialmente, la empresa ofrecerá dos tipos de promociones: por cantidad
y por porcentaje.
\begin{itemize}
    \item \textbf{Promoción por cantidad:} permitirá decrementar el precio del alquiler en la cantidad indicada en
    la promoción.
    \item \textbf{Promoción por porcentaje:} decrementará el precio del alquiler en el porcentaje indicado en la
    promoción. Las promociones se asignan a los alquileres en el momento de su creación.
\end{itemize}
Evidentemente, es posible
que a algunos alquileres no se les aplique ninguna promoción. Las promociones que se asignan a los alquileres son
determinadas por una política de la empresa que no impacta al diseño de nuestra operación (impactará a la
operación que crea los alquileres).\par
\vspace{0.15cm}
Eso sí, la empresa quiere que mientras no se haga el pago del alquiler, si aparecen
nuevas promociones, se apliquen a los alquileres siempre y cuando sean más favorables (no nos tenemos que
preocupar tampoco de estos cambios, son gestionados por otras operaciones).


\subsection{Patrón de Diseño utilizado}


\subsection{Efectos sobre el Diagrama de Diseño}

\subsection{Implementación de \textit{getPrice() : Integer}}
\begin{lstlisting}[style = javaNormal, language=Java] 

    introducir codigo aqui

\end{lstlisting} % Código en el documento "codigosEj3"

\newpage
