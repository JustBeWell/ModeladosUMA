\begin{lstlisting}[style = javaNormal, language=Java,caption={Clases generadas para el patron Strategy}] 
    //INTERFAZ GENERAL
    public interface DescuentoEstrategia {
    //Metodo para aplicar un descuento
    int aplicarDescuento(int precioBase);
    }
    //TIPO DE DESCUENTO 1 POR PORCENTAJE
    public class DescuentoPorcentaje implements DescuentoEstrategia{
        private final double porcentaje;

        public DescuentoPorcentaje(double porcentaje) {
            assert porcentaje >= 0 && porcentaje <= 100 : "Porcentaje invalido";
            this.porcentaje = porcentaje;
        }

        @Override
        public int aplicarDescuento(int precioBase) {
            double res = precioBase * (1-porcentaje/100);
            int sol = (int)res;
            return sol;
        }

        @Override
        public String toString() {
            return "[" + porcentaje + "]";
        }
    }
    //TIPO DE DESCUENTO 2 POR DESCUENTO 
    public class DescuentoPromocion implements DescuentoEstrategia{
        private final int descuento;
    
        public DescuentoPromocion(int descuento) {
            assert descuento >= 0 : "Descuento no puede ser negativo";
            this.descuento = descuento;
        }
    
        @Override
        public int aplicarDescuento(int precioBase) {
            if(descuento >= precioBase) return 0;
            return precioBase - descuento;
        }
    
        @Override
        public String toString() {
            return "[" + descuento + "]";
        }      
    }
\end{lstlisting}

\begin{lstlisting}[style = javaNormal, language=Java, caption={Metodo getPrice en la clase Rental}] 
    //NUEVO METODO PARA OBTENER PRECIO CON DESCUENTO SI EXISTE
    public int getPrice(){
        LocalDate localStartDate = startDate.toInstant().atZone(ZoneId.systemDefault()).toLocalDate();
        LocalDate localEndDate = endDate.toInstant().atZone(ZoneId.systemDefault()).toLocalDate();
        double diasRental = ChronoUnit.DAYS.between(localStartDate, localEndDate);
        int precioBase = (int) (diasRental * car.getModel().getPricePerDay());
        if(this.descuento != null) return descuento.aplicarDescuento(precioBase);
        return precioBase;
    }
\end{lstlisting}
